
% DOCUMENT CLASS
\documentclass[oneside,12pt]{article}

\usepackage{graphicx}

\usepackage{agda}



% Use T1 font encoding (for output)
\usepackage[T1]{fontenc}

% Enable utf-8 input encoding (n.b. this doesn't mean TeX will know how to print it)
\usepackage[utf8]{inputenc}

% For smaller `⊨' than \models gives
\usepackage{amssymb}

% Use: \mathpzc{abc...z} for lowercase mathscript
\DeclareMathAlphabet{\mathpzc}{OT1}{pzc}{m}{it}


%\usepackage{ucs}

%\usepackage[greek,english]{babel}
%\usepackage{eufrak}


%\DeclareUnicodeCharacter{1D4B9}{\ensuremath\mathfrak{d}}


\usepackage{newunicodechar}

\newunicodechar{≡}{\ensuremath{\mathnormal\equiv}}
\newunicodechar{≠}{\ensuremath{\mathnormal\neq}}
% Line drawing
\newunicodechar{═}{\ensuremath{\mathnormal\blacksquare}}
\newunicodechar{━}{\ensuremath{\mathnormal - -}}
\newunicodechar{Σ}{\ensuremath{\mathnormal\Sigma}}
%\newunicodechar{∈}{\ensuremath{\mathnormal\in}}
\newunicodechar{∀}{\ensuremath{\mathnormal\forall}}
%\newunicodechar{∈}{\ensuremath{\mathnormal\upepsilon}}
\newunicodechar{∃}{\ensuremath{\mathnormal\exists}}
\newunicodechar{ℕ}{\ensuremath{\mathbb{N}}}
\newunicodechar{≥}{\ensuremath{\mathnormal\geq}}
\newunicodechar{≤}{\ensuremath{\mathnormal\leq}}
\newunicodechar{₀}{\ensuremath{\mathnormal{_0}}}
\newunicodechar{₁}{\ensuremath{\mathnormal{_1}}}
\newunicodechar{₂}{\ensuremath{\mathnormal{_2}}}
\newunicodechar{₃}{\ensuremath{\mathnormal{_3}}}
\newunicodechar{₄}{\ensuremath{\mathnormal{_4}}}
\newunicodechar{₅}{\ensuremath{\mathnormal{_5}}}
\newunicodechar{₆}{\ensuremath{\mathnormal{_6}}}
\newunicodechar{ᴺ}{\ensuremath{\mathnormal{^N}}}
\newunicodechar{λ}{\ensuremath{\mathnormal\lambda}}
\newunicodechar{∘}{\ensuremath{\mathnormal\circ}}
\newunicodechar{⊢}{\ensuremath{\mathnormal\vdash}}
\newunicodechar{⊨}{\ensuremath{\mathnormal\vDash}}
\newunicodechar{⊭}{\ensuremath{\mathnormal\not\vDash}}
\newunicodechar{⊬}{\ensuremath{\mathnormal\not\vdash}}
\newunicodechar{⇒}{\ensuremath{\mathnormal\Rightarrow}}
\newunicodechar{⇧}{\ensuremath{\mathnormal\Uparrow}}
\newunicodechar{⇄}{\ensuremath{\mathnormal\rightleftarrows}}
\newunicodechar{∧}{\ensuremath{\mathnormal\wedge}}
\newunicodechar{∨}{\ensuremath{\mathnormal\lor}}
\newunicodechar{;}{\text{\bf{;}}}
\newunicodechar{↪}{\ensuremath{\mathnormal\Delta}}
\newunicodechar{𝑿}{\ensuremath{\mathnormal\mathcal{X}}}
\newunicodechar{𝒀}{\ensuremath{\mathnormal\mathcal{Y}}}
\newunicodechar{𝒙}{\ensuremath{\mathnormal\mathcal{X}}}
\newunicodechar{𝒙}{\codevar{x}}
\newunicodechar{𝒚}{\codevar{y}}
\newunicodechar{𝒛}{\codevar{z}}
\newunicodechar{𝓿}{\ensuremath{\mathnormal_v}}
\newunicodechar{⓪}{\constv{0}}
\newunicodechar{①}{\constv{1}}
\newunicodechar{②}{\constv{2}}
\newunicodechar{ℱ}{\ensuremath{\mathnormal\mathpzc{f}}}
\newunicodechar{⌊}{\ensuremath{\mathnormal\lfloor}}
\newunicodechar{⌋}{\ensuremath{\mathnormal\rfloor}}
\newunicodechar{ᵗ}{\ensuremath{^{t}}}
\newunicodechar{⸴}{\raisebox{-2pt}{,}}
\newunicodechar{″}{\ensuremath{\mathnormal\dagger}}
\newunicodechar{‵}{\ensuremath{\mathnormal\dagger}}
\newunicodechar{ϕ}{\ensuremath{\mathnormal\Phi}}
\newunicodechar{Δ}{\ensuremath{\mathnormal\Delta}}
\newunicodechar{⟪}{\large{\guillemotleft}}
\newunicodechar{⟫}{\large{\guillemotright}}
\newunicodechar{⟦}{\ensuremath{\mathnormal\llbracket}}
\newunicodechar{⟧}{\ensuremath{\mathnormal\rrbracket}}
\newunicodechar{∎}{\ensuremath{\mathnormal\blacksquare}}
\newunicodechar{Ψ}{\ensuremath{\mathnormal\Psi}}
\newunicodechar{Λ}{\ensuremath{\mathnormal\Lambda}}
\newunicodechar{⊥}{\ensuremath{\mathnormal\bot}}
\newunicodechar{⊤}{\ensuremath{\mathnormal\top}}
\newunicodechar{≢}{\ensuremath{\mathnormal\not\equiv}}
\newunicodechar{ₒ}{\ensuremath{\mathnormal{_o}}}
\newunicodechar{𝐶}{\ensuremath{\mathnormal\mathbf{C_1}}}

\newunicodechar{𝔴}{\impcode{w}}
\newunicodechar{𝔥}{\impcode{h}}
\newunicodechar{𝔦}{\impcode{i}}
\newunicodechar{𝔩}{\impcode{l}}
\newunicodechar{𝔢}{\impcode{e}}

%s
\newunicodechar{𝑘}{\agdamath{k}}
\newunicodechar{𝑖}{\agdamath{i}}
\newunicodechar{𝑝}{\agdamath{p}}

% e
% l
\newunicodechar{𝔰}{\impcode{s}}
% e


\newunicodechar{𝒹}{\impcode{d}}
\newunicodechar{ℴ}{\impcode{o}}

\newunicodechar{𝔦}{\impcode{i}}
\newunicodechar{𝔣}{\impcode{f}}

\newunicodechar{𝔱}{\impcode{t}}
\newunicodechar{𝔥}{\impcode{h}}
\newunicodechar{𝔢}{\impcode{e}}
\newunicodechar{𝔫}{\impcode{n}}

\newunicodechar{𝑐}{\agdamath{c}}
\newunicodechar{𝑜}{\agdamath{o}}
\newunicodechar{𝑛}{\agdamath{n}}
\newunicodechar{𝑠}{\agdamath{s}}
\newunicodechar{𝑡}{\agdamath{t}}


\newunicodechar{𝑣}{\agdamath{v}}
\newunicodechar{𝑎}{\agdamath{a}}
\newunicodechar{𝑙}{\agdamath{l}}


\newunicodechar{𝑒}{\agdamath{e}}
\newunicodechar{𝑣}{\agdamath{v}}
% e
\newunicodechar{𝑛}{\agdamath{n}}

% o
\newunicodechar{𝑑}{\agdamath{d}}
% d



%\newunicodechar{⟨}{\ensuremath{\mathnormal\langle}}
%\newunicodechar{⟩}{\ensuremath{\mathnormal\rangle}}
%
%\newunicodechar{π}{\ensuremath{\mathnormal\uppi}}
%\newunicodechar{𝔡}{\ensuremath{\mathfrak{d}}}
%\newunicodechar{𝔖}{\ensuremath{\mathfrak{s}}}
%
%
%% Language Constructs:
%% 𝒹
%\newunicodechar{𝒹}{\ensuremath{\mathfrak{d}}}
%% ℴ
%\newunicodechar{ℴ}{\ensuremath{\mathfrak{o}}}
%
%




%\DeclareUnicodeCharacter{"1D4B9}{\alpha}

% Caption options
\usepackage[font=footnotesize,labelfont=bf]{caption}

\usepackage{lipsum}

% Bold face small caps
% \usepackage{bold-extra}

%
\usepackage{ucs}
\usepackage[utf8]{inputenc}
\usepackage{amssymb}
\usepackage{bbm}
\usepackage[greek,english]{babel}
\usepackage{eufrak}

\DeclareUnicodeCharacter{1D4B9}{\ensuremath\mathfrak{d}}
\DeclareUnicodeCharacter{2261}{\ensuremath\equiv}

%\usepackage{newunicodechar}

%\newunicodechar{Σ}{\ensuremath{\mathnormal\Sigma}}
%\newunicodechar{⟨}{\ensuremath{\mathnormal\langle}}
%\newunicodechar{⟩}{\ensuremath{\mathnormal\rangle}}
%\newunicodechar{∀}{\ensuremath{\mathnormal\forall}}
%\newunicodechar{λ}{\ensuremath{\mathnormal\lambda}}
%
%\newunicodechar{π}{\ensuremath{\mathnormal\uppi}}
%\newunicodechar{∈}{\ensuremath{\mathnormal\upepsilon}}
%\newunicodechar{₀}{\ensuremath{\mathnormal{_0}}}
%\newunicodechar{₁}{\ensuremath{\mathnormal{_1}}}
%\newunicodechar{𝔡}{\ensuremath{\mathfrak{d}}}
%\newunicodechar{𝔖}{\ensuremath{\mathfrak{s}}}
%
%
%% Language Constructs:
%% 𝒹
%\newunicodechar{𝒹}{\ensuremath{\mathfrak{d}}}
%% ℴ
%\newunicodechar{ℴ}{\ensuremath{\mathfrak{o}}}
%
%




%\DeclareUnicodeCharacter{"1D4B9}{\alpha}
%\usepackage{eufrak}
%\usepackage{mathabx}
% Use Chancery Font


%Includes "References" in the table of contents
\usepackage[nottoc]{tocbibind}


\usepackage{ stmaryrd }% For \llbracket \rrbracket
% Used in \tctrip def

% Style for Agda snippet math script replacement
% (emphasised agda definitions)
\newcommand{\agdamath}[1]{\emph{\texttt{\!#1}}}

% style for Mini-Imp construct/mechanism
\newcommand{\impcode}[1]{\textsc{\texttt{#1}}}

\newcommand{\codevar}[1]{#1}


% For constants in agda-snippets
% Place input in circle (arg should be one numeral 1-9)
\newcommand{\constv}[1]{\raisebox{.5pt}{\textcircled{\raisebox{-.9pt} {#1}}}}

% Hoare's original notation for partial correctness
\newcommand{\hpc}[3]{$#1\{\!\!\{#2\}\!\!\}#3$}

% Gries then uses this for total correctness, but confusingly in many
% expositions these days it is used to denote partial correctness!
\newcommand{\gtc}[3]{\{\!\!\{#1\}\!\!\}\,#2\,\{\!\!\{#3\}\!\!\}}

% This reports/project's notation
% Partial Correctness Hoare-Triple
\newcommand{\pctrip}[3]{\large{\guillemotleft}\normalsize{$#1$}\large{\guillemotright}%
  \normalsize{$#2$}\large{\guillemotleft}\normalsize{$#3$}\large{\guillemotright}\normalsize}

% Total Correctness Hoare-Triple
\newcommand{\tctrip}[3]{$\large{\llbracket}\normalsize{#1}\large{\rrbracket}\,%
  \normalsize{#2}\,\large{\llbracket}\normalsize{#3}\large{\rrbracket}\normalsize$}


\newcommand{\wpre}[2]{$\textit{wp}(#1,#2)$}

\newcommand{\wlpre}[2]{$\textit{wlp}(#1,#2)$}

% Agda code state transformer symbol
\newcommand{\stateT}{\ensuremath{S\Delta}}


% def above equal sign
\newcommand{\eqdef}{$\stackrel{\text{\tiny def}}{=}$}

\begin{document}

\begin{titlepage}
  \begin{center}
    \large

    \textbf{\Large A Constructive Formalisation of Hoare Logic within the Interactive Theorem Prover Agda}

    
    \vfill

    Project Report \\
    Word Count XXXX   \\
    Fraser L. Brooks 1680975 \\
    Supervisor: Vincent Rahli
    
    \vfill
    \includegraphics[width=4cm]{Figures/birmingham_shield.png}
    \vfill
       
    Submitted in partial fulfillment \\
    of the requirements for the degree of\\
    Master of Science \\
    (Computer Science) \\
       
    \vfill
       
    at the \\
    University of Birmingham\\
    School of Computer Science\\
    July 2021
       
    \vspace{0.8cm}
       
  \end{center}
\end{titlepage}

\flushbottom

\begin{abstract}


  Program correctness is a perennial problem for software engineers and computer scientists alike. Many methods exist for establishing the correctness of a program and broadly speaking these methods fall into one of two paradigms; a program can be tested or the correctness can be `proved' outright. Due to the sheer complexity of software engineering, testing has reigned supreme in industry as formal techniques for proving correctness, while numerous, have lagged behind practice.
  However, with the advent of higher-order-logic theorem provers and dependently typed programming languages, both operating under the scope of the Curry-Howard correspondence, the gap between practice and theory is shrinking.

  Hoare logic is a formal system in which one can reason rigorously about --- and \emph{prove} --- the correctness of programs while Agda is both a dependently typed programming language \emph{and} an interactive theorem prover in accordance with the Curry-Howard correspondence. Combining the two, this work sets out to formalise the salient rules from Hoare logic within Agda and in doing so, provide a novel library with which a user could reason and prove correct simple imperative-style programs.

  This formalisation was achieved via a deep embedding of both a simple imperative language, dubbed \emph{`Mini-Imp,'} and of the propositional calculus used in the reasoning about programs in the guise of Mini-Imp's expression language. Interfaces were also used to seperate out the concerns of proving program correctness and proving trivial results within the expression language such as conjunction elimination or the distributivity of multiplication over addition.

  The final result is an Agda library that is fit for the purpose of reasoning about and proving correct simple imperative-style programs using the implemented Hoare logic rules. A limitation of the work is the simplicity of the Mini-Imp language and corresponding lack of more sophisticated logical rules meaning there is no facility for reasoning about more complex language constructs like procedures, arrays or pointers. However, more powerful logics such as `separation logic' --- an extension of Hoare logic --- could bridge this gap and owing to the expressive power of HOL, with time, there is no reason why the current library couldn't be expanded to encompass separation logic too.

  \end{abstract}

\pagebreak
  
\tableofcontents

\pagebreak

\raggedbottom

\section{Introduction}


This is some text. That will not be in my report. Lorem  ipsum  dolor  sit  amet,  consectetuer  adipiscing  
elit.   Etiam  lobortisfacilisis sem.  Nullam nec mi et 
neque pharetra sollicitudin.  Praesent imperdietmi nec ante. 
Donec ullamcorper, felis non sodales.. Lorem  ipsum  dolor  sit  amet,  consectetuer  adipiscing  
elit.   Etiam  lobortisfacilisis sem.  Nullam nec mi et 
neque pharetra sollicitudin.  Praesent imperdietmi nec ante. 
Donec ullamcorper, felis non sodales...

\section{Preliminaries \& Literature Review}

\subsection{Weakest Precondition}

How is one to give a semantics to computation? One answer is to provide a model of computation that denotes how a mechansim, a computation, or program\footnote{ the three words here being used synonymously} is to be computed; such as a Turing machine, or a FSA. Another way however is to describe what the mechanism, computation, or program can \emph{do} for us; that is, specifiying what input states it can accept, and what states it will produce.


OR: given a desired output (of states) specifying, how (read, in which input states), if at all, it can produce the desired output.

This approach gives us a way of specifying a \ldots without caring about the eventual form of the mechanism if indeed it ever takes form at all!

Computation as traversing the state space. Descartes - Cartesian product - [Dijkstra,p12]


Weakest precondition (according to Dijskstra) is unique when considered as a state space, but multiple predicates could denote the same space. (i.e. $x == y$ and $y == x$)

Strongest postcondition?
[Gries, exercise 4 section 9.1]


`If for a given $P$, $S$, and $R$, $P \Rightarrow wp(S,R)$ holds, this can often be proved without explicit formulation --- or, if you prefer, ``computation'', or ``derivation'' --- of the predicate $wp(S,R)$'




Note that in the text [dijkstra], $wp(S,R)$, is used interchangeably as a predicate and as the state space that said predicate captures. With our constructive formalisation however, this lack of precision is not possible nor desired, so we end up with the, perhaps superfluous, distinction between predicates and the state space that they describe. Meaning that under our formalisation, $wp(S,F)$ is empty when considered as a state space, but inhabited when considered as a predicate (inhabited uniquely by $F$ itself). This exposition also explains why \pctrip{F}{S}{Q} is an inhabited type, as $F$ \emph{is} a valid precondition of any computation for any postcondition (think absurd function, or bluff function). $<<$ Actually explained by the fact that what we have formalised is the weakest \emph{liberal} precodnition!


Weakest Liberal Precondition is what has actually been formalised! Total correctness is denoted by \tctrip{P}{S}{Q}


7 regions of the statesapce. As such, we can --- if we wish --- give a semantics to the notion of a derterministic mechanism as one in which the last four regions of the state space are empty.

\subsection{Hoare Logic}


\subsection{Agda}

\subsection{Constructive Mathematics}

`Agda is a constructive mathematical system by default, which amounts to saying that it can also be considered as a
programming language for manipulating mathematical objects.' - MHE


\subsection{Formal Proof}

\subsection{Applications}

\section{Specification}



\subsection{Obfuscating Interfaces}

Might have also wanted to abstract away expression language
(page 42 surface properties (Ligler))


x and y refer to possibly identical identifiers, while z is guaranteed to be distinct from both x and y. With more time a more expansive interface would be given allowing for as many identifiers as were necessary to reason about the desired program along with a mechanism for obtaining free identifiers from an expression, with the identifiers being represented as natural numbers, a new \emph{free} identifier could always be generated by summing the numerical value of the identifiers present in the supplied expression, or in a given list.

\subsection{Exppresion Language}

Carving up state space.
Every predicate denotes a subset of the statespace
(which in our case is infinite).

(day = 23) Dijkstra's example

T/F, x == 2

Relationship between logical operators and set theoretic operators
i.e. $\wedge \Leftrightarrow \cap$

{\advance\leftskip\mathindent


\begin{code}%
\>[2]\AgdaOperator{\AgdaFunction{𝑒𝑣𝑒𝑛\textlangle\AgdaUnderscore{}\textrangle}}\AgdaSpace{}%
\AgdaSymbol{:}\AgdaSpace{}%
\AgdaDatatype{Exp}\AgdaSpace{}%
\AgdaSymbol{→}\AgdaSpace{}%
\AgdaDatatype{Exp}\<%
\\
%
\>[2]\AgdaOperator{\AgdaFunction{𝑒𝑣𝑒𝑛\textlangle}}\AgdaSpace{}%
\AgdaBound{P}\AgdaSpace{}%
\AgdaOperator{\AgdaFunction{\textrangle}}\AgdaSpace{}%
\AgdaSymbol{=}\AgdaSpace{}%
\AgdaInductiveConstructor{op$_2$}\AgdaSpace{}%
\AgdaSymbol{(}\AgdaInductiveConstructor{op$_2$}\AgdaSpace{}%
\AgdaBound{P}\AgdaSpace{}%
\AgdaInductiveConstructor{\%$_o$}\AgdaSpace{}%
\AgdaSymbol{(}\AgdaInductiveConstructor{const}\AgdaSpace{}%
\AgdaFunction{②}\AgdaSymbol{))}\AgdaSpace{}%
\AgdaInductiveConstructor{==$_o$}\AgdaSpace{}%
\AgdaSymbol{(}\AgdaInductiveConstructor{const}\AgdaSpace{}%
\AgdaFunction{⓪}\AgdaSymbol{)}\<%
\end{code}



}

Ought to have differentiated between non stuck-ness and termination. I.e. D(E) as domain of expression E, to eliminate divide by zero and non-defined variables in an expression, as that is a problem that can be handled distinctly from termination
(i.e. (I think anway) that given a state S, and an expression E, one can deterministically/decidably determine whether or not it is a WFF).

Donec ullamcorper, felis non sodales.. Lorem  ipsum  dolor  sit  amet,  consectetuer  adipiscing  
elit.

{\advance\leftskip\mathindent
  \advance\leftskip\mathindent
  
\begin{code}%
\>[0]\AgdaFunction{\agdamath{WFF}}\AgdaSpace{}%
\AgdaSymbol{:}\AgdaSpace{}%
\AgdaFunction{Assertion}\AgdaSpace{}%
\AgdaSymbol{→}\AgdaSpace{}%
\AgdaField{S}\AgdaSpace{}%
\AgdaSymbol{→}\AgdaSpace{}%
\AgdaPrimitiveType{Set}\<%
\\
%
\>[0]\AgdaFunction{\agdamath{WFF}}\AgdaSpace{}%
\AgdaBound{a}\AgdaSpace{}%
\AgdaBound{s}\AgdaSpace{}%
\AgdaSymbol{=}\AgdaSpace{}%
\AgdaFunction{Is-just}\AgdaSpace{}%
\AgdaSymbol{(}\AgdaFunction{evalExp}\AgdaSpace{}%
\AgdaBound{a}\AgdaSpace{}%
\AgdaBound{s}\AgdaSymbol{)}\<%
\end{code}



}

Donec ullamcorper, felis non sodales.. Lorem  ipsum  dolor  sit  amet,  consectetuer  adipiscing  
elit.

{\advance\leftskip\mathindent
  \advance\leftskip\mathindent

\begin{code}%
\>[2]\AgdaFunction{Assert}\AgdaSpace{}%
\AgdaSymbol{:}\AgdaSpace{}%
\AgdaSymbol{∀}\AgdaSpace{}%
\AgdaBound{s}\AgdaSpace{}%
\AgdaBound{A}\AgdaSpace{}%
\AgdaSymbol{→}\AgdaSpace{}%
\AgdaPrimitiveType{Set}\<%
\\
%
\>[2]\AgdaFunction{Assert}\AgdaSpace{}%
\AgdaBound{s}\AgdaSpace{}%
\AgdaBound{A}\AgdaSpace{}%
\AgdaSymbol{=}\AgdaSpace{}%
\AgdaRecord{Σ}\AgdaSpace{}%
\AgdaSymbol{(}\AgdaFunction{\agdamath{WFF}}\AgdaSpace{}%
\AgdaBound{A}\AgdaSpace{}%
\AgdaBound{s}\AgdaSymbol{)}\AgdaSpace{}%
\AgdaSymbol{(}\AgdaFunction{T}\AgdaSpace{}%
\AgdaOperator{\AgdaFunction{∘}}\AgdaSpace{}%
\AgdaFunction{toTruthValue}\AgdaSymbol{)}\<%
\\
\>[2]\AgdaComment{-- Alternative, condensed syntax:}\<%
\\
\>[2]\AgdaOperator{\AgdaFunction{\AgdaUnderscore{}⊨\AgdaUnderscore{}}}\AgdaSpace{}%
\AgdaSymbol{:}\AgdaSpace{}%
\AgdaSymbol{∀}\AgdaSpace{}%
\AgdaBound{s}\AgdaSpace{}%
\AgdaBound{A}\AgdaSpace{}%
\AgdaSymbol{→}\AgdaSpace{}%
\AgdaPrimitiveType{Set}\<%
\\
%
\>[2]\AgdaBound{s}\AgdaSpace{}%
\AgdaOperator{\AgdaFunction{⊨}}\AgdaSpace{}%
\AgdaBound{A}\AgdaSpace{}%
\AgdaSymbol{=}\AgdaSpace{}%
\AgdaFunction{Assert}\AgdaSpace{}%
\AgdaBound{s}\AgdaSpace{}%
\AgdaBound{A}\<%
\end{code}

}

Donec ullamcorper, felis non sodales.. Lorem  ipsum  dolor  sit  amet,  consectetuer  adipiscing  
elit. 
Donec ullamcorper, felis non sodales.. Lorem  ipsum  dolor  sit  amet,  consectetuer  adipiscing  
elit.

{\centering \begin{code}%
\>[2]\AgdaOperator{\AgdaFunction{\AgdaUnderscore{}⇒\AgdaUnderscore{}}}\AgdaSpace{}%
\AgdaSymbol{:}\AgdaSpace{}%
\AgdaFunction{Assertion}\AgdaSpace{}%
\AgdaSymbol{→}\AgdaSpace{}%
\AgdaFunction{Assertion}\AgdaSpace{}%
\AgdaSymbol{→}\AgdaSpace{}%
\AgdaPrimitiveType{Set}\<%
\\
%
\>[2]\AgdaBound{P}\AgdaSpace{}%
\AgdaOperator{\AgdaFunction{⇒}}\AgdaSpace{}%
\AgdaBound{Q}\AgdaSpace{}%
\AgdaSymbol{=}\AgdaSpace{}%
\AgdaSymbol{(}\AgdaBound{s}\AgdaSpace{}%
\AgdaSymbol{:}\AgdaSpace{}%
\AgdaField{S}\AgdaSymbol{)}\AgdaSpace{}%
\AgdaSymbol{→}\AgdaSpace{}%
\AgdaBound{s}\AgdaSpace{}%
\AgdaOperator{\AgdaFunction{⊨}}\AgdaSpace{}%
\AgdaBound{P}\AgdaSpace{}%
\AgdaSymbol{→}\AgdaSpace{}%
\AgdaBound{s}\AgdaSpace{}%
\AgdaOperator{\AgdaFunction{⊨}}\AgdaSpace{}%
\AgdaBound{Q}\<%
\end{code}}


Donec ullamcorper, felis non sodales.. Lorem  ipsum  dolor  sit  amet,  consectetuer  adipiscing  
elit. 
Donec ullamcorper, felis non sodales.. Lorem ipsum  dolor  sit  amet,  consectetuer  adipiscing  
elit.


\begin{figure}
  \caption{Example of reasoning with the deep embedding of propositional logic}
  \centering
  
  \begin{tabular}{cc}
    \centering
    \begin{minipage}[t]{0.45\textwidth}
      \centering
      \begin{code}%
\>[2]\AgdaComment{a$_1$\;\;\eqdef\, \codevar{x} == 2 ∧ \codevar{y} == 1}\<%
\\
%
\>[2]\AgdaKeyword{private}\AgdaSpace{}%
\AgdaFunction{a₁}\AgdaSpace{}%
\AgdaSymbol{:}\AgdaSpace{}%
\AgdaFunction{Assertion}\<%
\\
%
\>[2]\AgdaFunction{a₁}\AgdaSpace{}%
\AgdaSymbol{=}%
\>[142I]\AgdaSymbol{((}\AgdaInductiveConstructor{𝑣𝑎𝑙}\AgdaSpace{}%
\AgdaFunction{\codevar{x}}\AgdaSymbol{)}\AgdaSpace{}%
\AgdaOperator{\AgdaFunction{==}}\AgdaSpace{}%
\AgdaSymbol{(}\AgdaInductiveConstructor{𝑐𝑜𝑛𝑠𝑡}\AgdaSpace{}%
\AgdaFunction{②}\AgdaSymbol{))}\<%
\\
\>[.][@{}l@{}]\<[142I]%
\>[7]\AgdaOperator{\AgdaFunction{∧}}\<%
\\
%
\>[7]\AgdaSymbol{((}\AgdaInductiveConstructor{𝑣𝑎𝑙}\AgdaSpace{}%
\AgdaFunction{\codevar{y}}\AgdaSymbol{)}\AgdaSpace{}%
\AgdaOperator{\AgdaFunction{==}}\AgdaSpace{}%
\AgdaSymbol{(}\AgdaInductiveConstructor{𝑐𝑜𝑛𝑠𝑡}\AgdaSpace{}%
\AgdaFunction{①}\AgdaSymbol{))}\<%
\end{code}
    \end{minipage}

    &
   
      \begin{minipage}[t]{0.45\textwidth}
        \centering
        \begin{code}
\>[2]\AgdaComment{a$_2$\;\;\eqdef\, \codevar{x} == 2}\<%
\\
%
\>[2]\AgdaKeyword{private}\AgdaSpace{}%
\AgdaFunction{a$_2$}\AgdaSpace{}%
\AgdaSymbol{:}\AgdaSpace{}%
\AgdaFunction{Assertion}\<%
\\
%
\>[2]\AgdaFunction{a$_2$}\AgdaSpace{}%
\AgdaSymbol{=}\AgdaSpace{}%
\AgdaSymbol{(}\AgdaInductiveConstructor{𝑣𝑎𝑙}\AgdaSpace{}%
\AgdaFunction{\codevar{x}}\AgdaSymbol{)}\AgdaSpace{}%
\AgdaOperator{\AgdaFunction{==}}\AgdaSpace{}%
\AgdaSymbol{(}\AgdaInductiveConstructor{𝑐𝑜𝑛𝑠𝑡}\AgdaSpace{}%
\AgdaFunction{②}\AgdaSymbol{)}\<%
\end{code}
      \end{minipage}
      
    \\
    
    \multicolumn{2}{c}{
    \begin{minipage}{0.9\textwidth}
      \centering
      \vspace{-0.5cm}
      \begin{code}%
\>[2]\AgdaFunction{inferenceExample}\AgdaSpace{}%
\AgdaSymbol{:}\AgdaSpace{}%
\AgdaFunction{a$_1$}\AgdaSpace{}%
\AgdaOperator{\AgdaFunction{⇒}}\AgdaSpace{}%
\AgdaFunction{a$_2$}\<%
\\
%
\>[2]\AgdaFunction{inferenceExample}%
\>[20]\AgdaBound{s}\AgdaSpace{}%
\AgdaBound{\,\,⊨𝒙$\scriptstyle{\,\&}$𝒚\,\,}\AgdaSpace{}%
\AgdaSymbol{=}%
\>[166I]\AgdaKeyword{let}\AgdaSpace{}%
\AgdaBound{x}\AgdaSpace{}%
\AgdaSymbol{=}\AgdaSpace{}%
\AgdaField{getVarVal}\AgdaSpace{}%
\AgdaFunction{𝒙}\AgdaSpace{}%
\AgdaBound{s}\AgdaSpace{}%
\AgdaOperator{\AgdaFunction{==$_v$}}\AgdaSpace{}%
\AgdaSymbol{(}\AgdaInductiveConstructor{just}\AgdaSpace{}%
\AgdaFunction{\constv{2}}\AgdaSymbol{)}\AgdaSpace{}%
\AgdaKeyword{in}\<%
\\
\>[.][@{}l@{}]\<[166I]%
\>[29]\AgdaKeyword{let}\AgdaSpace{}%
\AgdaBound{y}\AgdaSpace{}%
\AgdaSymbol{=}\AgdaSpace{}%
\AgdaField{getVarVal}\AgdaSpace{}%
\AgdaFunction{𝒚}\AgdaSpace{}%
\AgdaBound{s}\AgdaSpace{}%
\AgdaOperator{\AgdaFunction{==$_v$}}\AgdaSpace{}%
\AgdaSymbol{(}\AgdaInductiveConstructor{just}\AgdaSpace{}%
\AgdaFunction{\constv{1}}\AgdaSymbol{)}%
\>[65]\AgdaKeyword{in}\<%
\\
%
\>[29]\AgdaFunction{ConjunctionElim$_{left}$}\AgdaSpace{}%
\AgdaBound{\,x\,}\AgdaSpace{}%
\AgdaBound{\,y\,}\AgdaSpace{}%
\AgdaBound{\,\,⊨𝒙$\scriptstyle{\,\&}$𝒚}\<%
\>[0]\<%
\end{code}
    \end{minipage}}
 
  \end{tabular}

\end{figure}

\lipsum[66]

{\centering \begin{code}%
\>[2]\AgdaComment{-- Swap values of 𝒙 and 𝒚}\<%
\\
%
\>[2]\AgdaFunction{swap}\AgdaSpace{}%
\AgdaSymbol{:}\AgdaSpace{}%
\AgdaFunction{Program}\<%
\\
%
\>[2]\AgdaFunction{swap}\AgdaSpace{}%
\AgdaSymbol{=}%
\>[30I]\AgdaFunction{\codevar{z}}\AgdaSpace{}%
\AgdaOperator{\AgdaInductiveConstructor{\impcode{:=}}}\AgdaSpace{}%
\AgdaInductiveConstructor{𝑣𝑎𝑙}\AgdaSpace{}%
\AgdaFunction{𝒙}\AgdaSpace{}%
\AgdaOperator{\AgdaInductiveConstructor{;}}\<%
\\
\>[.][@{}l@{}]\<[30I]%
\>[9]\AgdaFunction{𝒙}\AgdaSpace{}%
\AgdaOperator{\AgdaInductiveConstructor{\impcode{:=}}}\AgdaSpace{}%
\AgdaInductiveConstructor{𝑣𝑎𝑙}\AgdaSpace{}%
\AgdaFunction{𝒚}\AgdaSpace{}%
\AgdaOperator{\AgdaInductiveConstructor{;}}\<%
\\
%
\>[9]\AgdaFunction{𝒚}\AgdaSpace{}%
\AgdaOperator{\AgdaInductiveConstructor{\impcode{:=}}}\AgdaSpace{}%
\AgdaInductiveConstructor{𝑣𝑎𝑙}\AgdaSpace{}%
\AgdaFunction{\codevar{z}}\AgdaSpace{}%
\AgdaOperator{\AgdaInductiveConstructor{;}}\<%
\\
\end{code}}

\lipsum[75]


\begin{figure}
 \caption{Some simple programs defined with Mini-Imp; ripe for reasoning!}

\begin{tabular}{r|l}
  \centering
  \footnotesize
\begin{minipage}[t]{0.4\textwidth}
  \centering
  \begin{code}%
\>[2]\AgdaComment{-- Euclids Algorithm for GCD}\<%
\\
%
\>[2]\AgdaFunction{gcd}\AgdaSpace{}%
\AgdaSymbol{:}\AgdaSpace{}%
\AgdaSymbol{(}\AgdaBound{𝑿}\AgdaSpace{}%
\AgdaBound{𝒀}\AgdaSpace{}%
\AgdaSymbol{:}\AgdaSpace{}%
\AgdaDatatype{Exp}\AgdaSymbol{)}\AgdaSpace{}%
\AgdaSymbol{→}\AgdaSpace{}%
\AgdaFunction{Program}\<%
\\
%
\>[2]\AgdaFunction{gcd}%
\>[70I]\AgdaBound{𝑿}\AgdaSpace{}%
\AgdaBound{𝒀}\AgdaSpace{}%
\AgdaSymbol{=}\<%
\\
\>[.][@{}l@{}]\<[70I]%
\>[6]\AgdaFunction{𝒙}\AgdaSpace{}%
\AgdaOperator{\AgdaInductiveConstructor{\impcode{:=}}}\AgdaSpace{}%
\AgdaBound{𝑿}\AgdaSpace{}%
\AgdaOperator{\AgdaInductiveConstructor{;}}\<%
\\
%
\>[6]\AgdaFunction{𝒚}\AgdaSpace{}%
\AgdaOperator{\AgdaInductiveConstructor{\impcode{:=}}}\AgdaSpace{}%
\AgdaBound{𝒀}\AgdaSpace{}%
\AgdaOperator{\AgdaInductiveConstructor{;}}\<%
\\
\>[2][@{}l@{\AgdaIndent{0}}]%
\>[5]\AgdaSymbol{(}\AgdaOperator{\AgdaInductiveConstructor{𝔴𝔥𝔦𝔩𝔢}}\AgdaSpace{}%
\AgdaSymbol{(}\AgdaFunction{𝑛𝑜𝑡}\AgdaSpace{}%
\AgdaSymbol{(}\AgdaSpace{}%
\AgdaInductiveConstructor{𝑣𝑎𝑙}\AgdaSpace{}%
\AgdaFunction{𝒙}\AgdaSpace{}%
\AgdaOperator{\AgdaFunction{==}}\AgdaSpace{}%
\AgdaInductiveConstructor{𝑣𝑎𝑙}\AgdaSpace{}%
\AgdaFunction{𝒚}\AgdaSpace{}%
\AgdaSymbol{))}\<%
\\
\>[5][@{}l@{\AgdaIndent{0}}]%
\>[6]\AgdaOperator{\AgdaInductiveConstructor{𝒹ℴ}}%
\>[87I]\AgdaSymbol{(}\AgdaOperator{\AgdaInductiveConstructor{𝔦𝔣}}\AgdaSpace{}%
\AgdaSymbol{(}\AgdaSpace{}%
\AgdaInductiveConstructor{𝑣𝑎𝑙}\AgdaSpace{}%
\AgdaFunction{𝒙}\AgdaSpace{}%
\AgdaOperator{\AgdaFunction{>}}\AgdaSpace{}%
\AgdaInductiveConstructor{𝑣𝑎𝑙}\AgdaSpace{}%
\AgdaFunction{𝒚}%
\>[30]\AgdaSymbol{)}\<%
\\
\>[.][@{}l@{}]\<[87I]%
\>[9]\AgdaOperator{\AgdaInductiveConstructor{𝔱𝔥𝔢𝔫}}\AgdaSpace{}%
\AgdaSymbol{(}\<%
\\
\>[9][@{}l@{\AgdaIndent{0}}]%
\>[11]\AgdaFunction{𝒙}\AgdaSpace{}%
\AgdaOperator{\AgdaInductiveConstructor{\impcode{:=}}}\AgdaSpace{}%
\AgdaInductiveConstructor{𝑣𝑎𝑙}\AgdaSpace{}%
\AgdaFunction{𝒙}\AgdaSpace{}%
\AgdaOperator{\AgdaFunction{-}}\AgdaSpace{}%
\AgdaInductiveConstructor{𝑣𝑎𝑙}\AgdaSpace{}%
\AgdaFunction{𝒚}\AgdaSpace{}%
\AgdaOperator{\AgdaInductiveConstructor{;}}\AgdaSymbol{)}\<%
\\
%
\>[9]\AgdaOperator{\AgdaInductiveConstructor{𝔢𝔩𝔰𝔢}}\AgdaSpace{}%
\AgdaSymbol{(}\<%
\\
\>[9][@{}l@{\AgdaIndent{0}}]%
\>[11]\AgdaFunction{𝒚}\AgdaSpace{}%
\AgdaOperator{\AgdaInductiveConstructor{\impcode{:=}}}\AgdaSpace{}%
\AgdaInductiveConstructor{𝑣𝑎𝑙}\AgdaSpace{}%
\AgdaFunction{𝒚}\AgdaSpace{}%
\AgdaOperator{\AgdaFunction{-}}\AgdaSpace{}%
\AgdaInductiveConstructor{𝑣𝑎𝑙}\AgdaSpace{}%
\AgdaFunction{𝒙}\AgdaSpace{}%
\AgdaOperator{\AgdaInductiveConstructor{;}}\AgdaSymbol{)}%
\>[34]\AgdaOperator{\AgdaInductiveConstructor{;}}\AgdaSymbol{)}\AgdaSpace{}%
\AgdaSymbol{)}\AgdaOperator{\AgdaInductiveConstructor{;}}\<%
\\
\end{code}

\end{minipage}

& 
   
\begin{minipage}[t]{0.5\textwidth}
  \centering
  \footnotesize
  \advance\leftskip0.2cm
  \begin{code}
\>[2]\AgdaComment{-- Multiply 𝑿 and 𝒀, and store in 𝒛}\<%
\\
%
\>[2]\AgdaComment{-- without using multiplication op.}\<%
\\
%
\>[2]\AgdaComment{-- ((11.4) in TSOP,Gries)}\<%
\\
%
\>[2]\AgdaFunction{add*}\AgdaSpace{}%
\AgdaSymbol{:}\AgdaSpace{}%
\AgdaSymbol{(}\AgdaBound{𝑿}\AgdaSpace{}%
\AgdaBound{𝒀}\AgdaSpace{}%
\AgdaSymbol{:}\AgdaSpace{}%
\AgdaDatatype{Exp}\AgdaSymbol{)}\AgdaSpace{}%
\AgdaSymbol{→}\AgdaSpace{}%
\AgdaFunction{Program}\<%
\\
%
\>[2]\AgdaFunction{add*}%
\>[118I]\AgdaBound{𝑿}\AgdaSpace{}%
\AgdaBound{𝒀}\AgdaSpace{}%
\AgdaSymbol{=}\<%
\\
\>[.][@{}l@{}]\<[118I]%
\>[7]\AgdaFunction{𝒙}\AgdaSpace{}%
\AgdaOperator{\AgdaInductiveConstructor{:=}}\AgdaSpace{}%
\AgdaBound{𝑿}\AgdaSpace{}%
\AgdaOperator{\AgdaInductiveConstructor{;}}\<%
\\
%
\>[7]\AgdaFunction{𝒚}\AgdaSpace{}%
\AgdaOperator{\AgdaInductiveConstructor{:=}}\AgdaSpace{}%
\AgdaBound{𝒀}\AgdaSpace{}%
\AgdaOperator{\AgdaInductiveConstructor{;}}\<%
\\
%
\>[7]\AgdaFunction{𝒛}\AgdaSpace{}%
\AgdaOperator{\AgdaInductiveConstructor{:=}}\AgdaSpace{}%
\AgdaInductiveConstructor{𝑐𝑜𝑛𝑠𝑡}\AgdaSpace{}%
\AgdaFunction{⓪}\AgdaSpace{}%
\AgdaOperator{\AgdaInductiveConstructor{;}}\<%
\\
\>[2][@{}l@{\AgdaIndent{0}}]%
\>[6]\AgdaSymbol{(}\AgdaOperator{\AgdaInductiveConstructor{𝔴𝔥𝔦𝔩𝔢}}\<%
\\
\>[6][@{}l@{\AgdaIndent{0}}]%
\>[8]\AgdaSymbol{(}%
\>[12]\AgdaSymbol{(}\AgdaSpace{}%
\AgdaInductiveConstructor{𝑣𝑎𝑙}\AgdaSpace{}%
\AgdaFunction{𝒚}\AgdaSpace{}%
\AgdaOperator{\AgdaFunction{>}}\AgdaSpace{}%
\AgdaInductiveConstructor{𝑐𝑜𝑛𝑠𝑡}\AgdaSpace{}%
\AgdaFunction{⓪}\AgdaSpace{}%
\AgdaOperator{\AgdaFunction{∧}}\AgdaSpace{}%
\AgdaOperator{\AgdaFunction{𝑒𝑣𝑒𝑛〈}}\AgdaSpace{}%
\AgdaInductiveConstructor{𝑣𝑎𝑙}\AgdaSpace{}%
\AgdaFunction{𝒚}\AgdaSpace{}%
\AgdaOperator{\AgdaFunction{〉}}\AgdaSpace{}%
\AgdaSymbol{)}\<%
\\
\>[8][@{}l@{\AgdaIndent{0}}]%
\>[10]\AgdaOperator{\AgdaFunction{∨}}\AgdaSpace{}%
\AgdaSymbol{(}\AgdaSpace{}%
\AgdaOperator{\AgdaFunction{𝑜𝑑𝑑〈}}\AgdaSpace{}%
\AgdaInductiveConstructor{𝑣𝑎𝑙}\AgdaSpace{}%
\AgdaFunction{𝒚}\AgdaSpace{}%
\AgdaOperator{\AgdaFunction{〉}}\AgdaSpace{}%
\AgdaSymbol{)}\<%
\\
%
\>[8]\AgdaSymbol{)}\<%
\\
\>[6][@{}l@{\AgdaIndent{0}}]%
\>[7]\AgdaOperator{\AgdaInductiveConstructor{𝒹ℴ}}%
\>[148I]\AgdaSymbol{(}\AgdaOperator{\AgdaInductiveConstructor{𝔦𝔣}}\AgdaSpace{}%
\AgdaSymbol{(}\AgdaSpace{}%
\AgdaOperator{\AgdaFunction{𝑒𝑣𝑒𝑛〈}}\AgdaSpace{}%
\AgdaInductiveConstructor{𝑣𝑎𝑙}\AgdaSpace{}%
\AgdaFunction{𝒚}\AgdaSpace{}%
\AgdaOperator{\AgdaFunction{〉}}\AgdaSpace{}%
\AgdaSymbol{)}\<%
\\
\>[.][@{}l@{}]\<[148I]%
\>[10]\AgdaOperator{\AgdaInductiveConstructor{𝔱𝔥𝔢𝔫}}\AgdaSpace{}%
\AgdaSymbol{(}\<%
\\
\>[10][@{}l@{\AgdaIndent{0}}]%
\>[13]\AgdaFunction{𝒚}\AgdaSpace{}%
\AgdaOperator{\AgdaInductiveConstructor{:=}}\AgdaSpace{}%
\AgdaInductiveConstructor{𝑣𝑎𝑙}\AgdaSpace{}%
\AgdaFunction{𝒚}\AgdaSpace{}%
\AgdaOperator{\AgdaFunction{/}}\AgdaSpace{}%
\AgdaInductiveConstructor{𝑐𝑜𝑛𝑠𝑡}\AgdaSpace{}%
\AgdaFunction{②}\AgdaSpace{}%
\AgdaOperator{\AgdaInductiveConstructor{;}}\<%
\\
%
\>[13]\AgdaFunction{𝒙}\AgdaSpace{}%
\AgdaOperator{\AgdaInductiveConstructor{:=}}\AgdaSpace{}%
\AgdaInductiveConstructor{𝑣𝑎𝑙}\AgdaSpace{}%
\AgdaFunction{𝒙}\AgdaSpace{}%
\AgdaOperator{\AgdaFunction{+}}\AgdaSpace{}%
\AgdaInductiveConstructor{𝑣𝑎𝑙}\AgdaSpace{}%
\AgdaFunction{𝒙}%
\>[35]\AgdaOperator{\AgdaInductiveConstructor{;}}\AgdaSymbol{)}\<%
\\
%
\>[10]\AgdaOperator{\AgdaInductiveConstructor{𝔢𝔩𝔰𝔢}}\AgdaSpace{}%
\AgdaSymbol{(}\<%
\\
\>[10][@{}l@{\AgdaIndent{0}}]%
\>[13]\AgdaFunction{𝒚}\AgdaSpace{}%
\AgdaOperator{\AgdaInductiveConstructor{:=}}\AgdaSpace{}%
\AgdaInductiveConstructor{𝑣𝑎𝑙}\AgdaSpace{}%
\AgdaFunction{𝒚}\AgdaSpace{}%
\AgdaOperator{\AgdaFunction{-}}\AgdaSpace{}%
\AgdaInductiveConstructor{𝑐𝑜𝑛𝑠𝑡}\AgdaSpace{}%
\AgdaFunction{①}\AgdaSpace{}%
\AgdaOperator{\AgdaInductiveConstructor{;}}\<%
\\
%
\>[13]\AgdaFunction{𝒛}\AgdaSpace{}%
\AgdaOperator{\AgdaInductiveConstructor{:=}}\AgdaSpace{}%
\AgdaInductiveConstructor{𝑣𝑎𝑙}\AgdaSpace{}%
\AgdaFunction{𝒛}\AgdaSpace{}%
\AgdaOperator{\AgdaFunction{+}}\AgdaSpace{}%
\AgdaInductiveConstructor{𝑣𝑎𝑙}\AgdaSpace{}%
\AgdaFunction{𝒙}%
\>[35]\AgdaOperator{\AgdaInductiveConstructor{;}}\AgdaSymbol{)}%
\>[41]\AgdaOperator{\AgdaInductiveConstructor{;}}\AgdaSymbol{)}\AgdaSpace{}%
\AgdaSymbol{)}\AgdaOperator{\AgdaInductiveConstructor{;}}\<%
\\  
\end{code}
\end{minipage}

\end{tabular}

\end{figure}

\begin{figure}
  \caption{The deep embedding in Agda of our `Mini-Imp' imperative language.}
  \advance\leftskip2\mathindent
  \begin{code}
\>[2]\AgdaKeyword{data}\AgdaSpace{}%
\AgdaDatatype{S↪}\AgdaSpace{}%
\AgdaSymbol{:}\AgdaSpace{}%
\AgdaPrimitiveType{Set}\AgdaSpace{}%
\AgdaKeyword{where}\<%
\\
\>[2][@{}l@{\AgdaIndent{0}}]%
\>[4]\AgdaInductiveConstructor{𝑠𝑘𝑖𝑝}%
\>[10]\AgdaSymbol{:}\AgdaSpace{}%
\AgdaDatatype{S↪}\<%
\\
%
\>[4]\AgdaOperator{\AgdaInductiveConstructor{𝔴𝔥𝔦𝔩𝔢\AgdaUnderscore{}𝒹ℴ\AgdaUnderscore{}}}\AgdaSpace{}%
\AgdaSymbol{:}\AgdaSpace{}%
\AgdaDatatype{Exp}\AgdaSpace{}%
\AgdaSymbol{→}\AgdaSpace{}%
\AgdaDatatype{Program}\AgdaSpace{}%
\AgdaSymbol{→}\AgdaSpace{}%
\AgdaDatatype{S↪}\<%
\\
%
\>[4]\AgdaOperator{\AgdaInductiveConstructor{𝔦𝔣\AgdaUnderscore{}𝔱𝔥𝔢𝔫\AgdaUnderscore{}𝔢𝔩𝔰𝔢\AgdaUnderscore{}}}\AgdaSpace{}%
\AgdaSymbol{:}\AgdaSpace{}%
\AgdaDatatype{Exp}\AgdaSpace{}%
\AgdaSymbol{→}\AgdaSpace{}%
\AgdaDatatype{Program}\AgdaSpace{}%
\AgdaSymbol{→}\AgdaSpace{}%
\AgdaDatatype{Program}\AgdaSpace{}%
\AgdaSymbol{→}\AgdaSpace{}%
\AgdaDatatype{S↪}\<%
\\
%
\>[4]\AgdaOperator{\AgdaInductiveConstructor{\AgdaUnderscore{}:=\AgdaUnderscore{}}}\AgdaSpace{}%
\AgdaSymbol{:}\AgdaSpace{}%
\AgdaFunction{Id}\AgdaSpace{}%
\AgdaSymbol{→}\AgdaSpace{}%
\AgdaDatatype{Exp}\AgdaSpace{}%
\AgdaSymbol{→}\AgdaSpace{}%
\AgdaDatatype{S↪}\<%
\\
%
\\[\AgdaEmptyExtraSkip]%
%
\>[2]\AgdaKeyword{data}\AgdaSpace{}%
\AgdaDatatype{Program}\AgdaSpace{}%
\AgdaSymbol{:}\AgdaSpace{}%
\AgdaPrimitiveType{Set}\AgdaSpace{}%
\AgdaKeyword{where}\<%
\\
\>[2][@{}l@{\AgdaIndent{0}}]%
\>[4]\AgdaComment{-- Terminator:}\<%
\\
%
\>[4]\AgdaOperator{\AgdaInductiveConstructor{\AgdaUnderscore{};}}%
\>[8]\AgdaSymbol{:}\AgdaSpace{}%
\AgdaDatatype{S↪}\AgdaSpace{}%
\AgdaSymbol{→}\AgdaSpace{}%
\AgdaDatatype{Program}\<%
\\
%
\>[4]\AgdaComment{-- Separator:}\<%
\\
%
\>[4]\AgdaOperator{\AgdaInductiveConstructor{\AgdaUnderscore{};\AgdaUnderscore{}}}\AgdaSpace{}%
\AgdaSymbol{:}\AgdaSpace{}%
\AgdaDatatype{S↪}\AgdaSpace{}%
\AgdaSymbol{→}\AgdaSpace{}%
\AgdaDatatype{Program}\AgdaSpace{}%
\AgdaSymbol{→}\AgdaSpace{}%
\AgdaDatatype{Program}\<%
\end{code}

\end{figure}

\lipsum[63]


{\advance\leftskip2\mathindent
  \begin{code}
\>[2]\AgdaComment{-- Program composition}\<%
\\
%
\>[2]\AgdaOperator{\AgdaFunction{\AgdaUnderscore{}𝔱𝔥𝔢𝔫\AgdaUnderscore{}}}\AgdaSpace{}%
\AgdaSymbol{:}\AgdaSpace{}%
\AgdaDatatype{Program}\AgdaSpace{}%
\AgdaSymbol{→}\AgdaSpace{}%
\AgdaDatatype{Program}\AgdaSpace{}%
\AgdaSymbol{→}\AgdaSpace{}%
\AgdaDatatype{Program}\<%
\\
%
\>[2]\AgdaSymbol{(}\AgdaBound{c}\AgdaSpace{}%
\AgdaOperator{\AgdaInductiveConstructor{;}}\AgdaSymbol{)}\AgdaSpace{}%
\AgdaOperator{\AgdaFunction{𝔱𝔥𝔢𝔫}}\AgdaSpace{}%
\AgdaBound{b}\AgdaSpace{}%
\AgdaSymbol{=}\AgdaSpace{}%
\AgdaBound{c}\AgdaSpace{}%
\AgdaOperator{\AgdaInductiveConstructor{;}}\AgdaSpace{}%
\AgdaBound{b}\<%
\\
%
\>[2]\AgdaSymbol{(}\AgdaBound{c}\AgdaSpace{}%
\AgdaOperator{\AgdaInductiveConstructor{;}}\AgdaSpace{}%
\AgdaBound{b₁}\AgdaSymbol{)}\AgdaSpace{}%
\AgdaOperator{\AgdaFunction{𝔱𝔥𝔢𝔫}}\AgdaSpace{}%
\AgdaBound{b₂}\AgdaSpace{}%
\AgdaSymbol{=}\AgdaSpace{}%
\AgdaBound{c}\AgdaSpace{}%
\AgdaOperator{\AgdaInductiveConstructor{;}}\AgdaSpace{}%
\AgdaSymbol{(}\AgdaBound{b₁}\AgdaSpace{}%
\AgdaOperator{\AgdaFunction{𝔱𝔥𝔢𝔫}}\AgdaSpace{}%
\AgdaBound{b₂}\AgdaSymbol{)}\<%
\end{code}
  \begin{code}
\>[2]\AgdaComment{-- Commutativity of program composition}\<%
\\
%
\>[2]\AgdaFunction{𝔱𝔥𝔢𝔫-comm}\AgdaSpace{}%
\AgdaSymbol{:}\AgdaSpace{}%
\AgdaSymbol{∀}\AgdaSpace{}%
\AgdaBound{c₁}\AgdaSpace{}%
\AgdaBound{c₂}\AgdaSpace{}%
\AgdaBound{c₃}\AgdaSpace{}%
\AgdaSymbol{→}\<%
\\
\>[2][@{}l@{\AgdaIndent{0}}]%
\>[4]\AgdaBound{c₁}\AgdaSpace{}%
\AgdaOperator{\AgdaFunction{𝔱𝔥𝔢𝔫}}\AgdaSpace{}%
\AgdaSymbol{(}\AgdaBound{c₂}\AgdaSpace{}%
\AgdaOperator{\AgdaFunction{𝔱𝔥𝔢𝔫}}\AgdaSpace{}%
\AgdaBound{c₃}\AgdaSymbol{)}\AgdaSpace{}%
\AgdaOperator{\AgdaDatatype{≡}}\AgdaSpace{}%
\AgdaSymbol{(}\AgdaBound{c₁}\AgdaSpace{}%
\AgdaOperator{\AgdaFunction{𝔱𝔥𝔢𝔫}}\AgdaSpace{}%
\AgdaBound{c₂}\AgdaSymbol{)}\AgdaSpace{}%
\AgdaOperator{\AgdaFunction{𝔱𝔥𝔢𝔫}}\AgdaSpace{}%
\AgdaBound{c₃}\<%
\\
%
\>[2]\AgdaFunction{𝔱𝔥𝔢𝔫-comm}\AgdaSpace{}%
\AgdaSymbol{(}\AgdaBound{s↪}\AgdaSpace{}%
\AgdaOperator{\AgdaInductiveConstructor{;}}\AgdaSymbol{)}\AgdaSpace{}%
\AgdaBound{c₂}\AgdaSpace{}%
\AgdaBound{c₃}\AgdaSpace{}%
\AgdaSymbol{=}\AgdaSpace{}%
\AgdaInductiveConstructor{refl}\<%
\\
%
\>[2]\AgdaFunction{𝔱𝔥𝔢𝔫-comm}\AgdaSpace{}%
\AgdaSymbol{(}\AgdaBound{s↪}\AgdaSpace{}%
\AgdaOperator{\AgdaInductiveConstructor{;}}\AgdaSpace{}%
\AgdaBound{c₁}\AgdaSymbol{)}\AgdaSpace{}%
\AgdaBound{c₂}\AgdaSpace{}%
\AgdaBound{c₃}\<%
\\
\>[2][@{}l@{\AgdaIndent{0}}]%
\>[4]\AgdaKeyword{rewrite}\AgdaSpace{}%
\AgdaFunction{𝔱𝔥𝔢𝔫-comm}\AgdaSpace{}%
\AgdaBound{c₁}\AgdaSpace{}%
\AgdaBound{c₂}\AgdaSpace{}%
\AgdaBound{c₃}\AgdaSpace{}%
\AgdaSymbol{=}\AgdaSpace{}%
\AgdaInductiveConstructor{refl}\<%
\end{code}
}


\lipsum[63]

\begin{code}
\>[2]\AgdaFunction{ssEvalwithFuel}\AgdaSpace{}%
\AgdaSymbol{:}%
\>[20]\AgdaDatatype{ℕ}\AgdaSpace{}%
\AgdaSymbol{→}\AgdaSpace{}%
\AgdaFunction{Program}\AgdaSpace{}%
\AgdaSymbol{→}\AgdaSpace{}%
\AgdaField{S}\AgdaSpace{}%
\AgdaSymbol{→}\AgdaSpace{}%
\AgdaDatatype{Maybe}\AgdaSpace{}%
\AgdaField{S}\<%
\end{code}

\lipsum[66]

{\centering
\begin{code}
\>[2]\AgdaComment{-----------------------------------------------------------------}\<%
\\
%
\>[2]\AgdaComment{-\,\!- SINGLE WHILE }\<%
\\
%
\>[2]\AgdaFunction{ssEvalwithFuel}\AgdaSpace{}%
\AgdaSymbol{(}\AgdaInductiveConstructor{suc}\AgdaSpace{}%
\AgdaBound{n}\AgdaSymbol{)}\AgdaSpace{}%
\AgdaSymbol{(}\AgdaSpace{}%
\AgdaOperator{\AgdaInductiveConstructor{𝔴𝔥𝔦𝔩𝔢}}\AgdaSpace{}%
\AgdaBound{exp}\AgdaSpace{}%
\AgdaOperator{\AgdaInductiveConstructor{𝒹ℴ}}\AgdaSpace{}%
\AgdaBound{c}\AgdaSpace{}%
\AgdaOperator{\AgdaInductiveConstructor{;}}\AgdaSymbol{)}\AgdaSpace{}%
\AgdaBound{s}\AgdaSpace{}%
\AgdaKeyword{with}\AgdaSpace{}%
\AgdaFunction{evalExp}\AgdaSpace{}%
\AgdaBound{exp}\AgdaSpace{}%
\AgdaBound{s}\<%
\\
%
\>[2]\AgdaSymbol{...}\AgdaSpace{}%
\AgdaSymbol{|}\AgdaSpace{}%
\AgdaInductiveConstructor{nothing}\AgdaSpace{}%
\AgdaSymbol{=}\AgdaSpace{}%
\AgdaInductiveConstructor{nothing}\AgdaSpace{}%
\AgdaComment{\,-\,\!-\!\! Computation failed e.\!g.\,div by 0}\<%
\\
%
\>[2]\AgdaSymbol{...}\AgdaSpace{}%
\AgdaSymbol{|}\AgdaSpace{}%
\AgdaBound{f\;}\AgdaSymbol{@\;(}\AgdaInductiveConstructor{just}\AgdaSpace{}%
\AgdaSymbol{\AgdaUnderscore{})}\AgdaSpace{}%
\AgdaKeyword{with}\AgdaSpace{}%
\AgdaFunction{toTruthValue}\AgdaSpace{}%
\AgdaSymbol{\{}\AgdaBound{\,f\,}\AgdaSymbol{\}}\AgdaSpace{}%
\AgdaSymbol{(}\AgdaInductiveConstructor{Any.just}\AgdaSpace{}%
\AgdaInductiveConstructor{tt}\AgdaSymbol{)}\<%
\\
%
\>[2]\AgdaSymbol{...}\AgdaSpace{}%
\AgdaSymbol{|}\AgdaSpace{}%
\AgdaInductiveConstructor{true}%
\>[14]\AgdaSymbol{=}\AgdaSpace{}%
\AgdaFunction{ssEvalwithFuel}\AgdaSpace{}%
\AgdaBound{n}\AgdaSpace{}%
\AgdaSymbol{(}\AgdaSpace{}%
\AgdaBound{c}\AgdaSpace{}%
\AgdaOperator{\AgdaFunction{𝔱𝔥𝔢𝔫}}\AgdaSpace{}%
\AgdaOperator{\AgdaInductiveConstructor{𝔴𝔥𝔦𝔩𝔢}}\AgdaSpace{}%
\AgdaBound{exp}\AgdaSpace{}%
\AgdaOperator{\AgdaInductiveConstructor{𝒹ℴ}}\AgdaSpace{}%
\AgdaBound{c}\AgdaSpace{}%
\AgdaOperator{\AgdaInductiveConstructor{;}}\AgdaSymbol{)}\AgdaSpace{}%
\AgdaBound{s}\<%
\\
%
\>[2]\AgdaSymbol{...}\AgdaSpace{}%
\AgdaSymbol{|}\AgdaSpace{}%
\AgdaInductiveConstructor{false}\AgdaSpace{}%
\AgdaSymbol{=}\AgdaSpace{}%
\AgdaInductiveConstructor{just}\AgdaSpace{}%
\AgdaBound{s}\<%
\\
%
\>[2]\AgdaComment{-----------------------------------------------------------------}\<%
\end{code}

\vspace{-1cm}

\begin{code}
\>[2]\AgdaComment{-----------------------------------------------------------------}\<%
\\
%
\>[2]\AgdaComment{-\,\!- WHILE ; THEN C₂}\<%
\\
%
\>[2]\AgdaFunction{ssEvalwithFuel}\AgdaSpace{}%
\AgdaSymbol{(}\AgdaInductiveConstructor{suc}\AgdaSpace{}%
\AgdaBound{n}\AgdaSymbol{)}\AgdaSpace{}%
\AgdaSymbol{((}\AgdaOperator{\AgdaInductiveConstructor{𝔴𝔥𝔦𝔩𝔢}}\AgdaSpace{}%
\AgdaBound{exp}\AgdaSpace{}%
\AgdaOperator{\AgdaInductiveConstructor{𝒹ℴ}}\AgdaSpace{}%
\AgdaBound{c₁}\AgdaSymbol{)}\AgdaSpace{}%
\AgdaOperator{\AgdaInductiveConstructor{;}}\AgdaSpace{}%
\AgdaBound{c₂}\AgdaSymbol{)}\AgdaSpace{}%
\AgdaBound{s}\<%
\\
\>[2][@{}l@{\AgdaIndent{0}}]%
\>[6]\AgdaKeyword{with}\AgdaSpace{}%
\AgdaFunction{evalExp}\AgdaSpace{}%
\AgdaBound{exp}\AgdaSpace{}%
\AgdaBound{s}\<%
\\
%
\>[2]\AgdaSymbol{...}\AgdaSpace{}%
\AgdaSymbol{|}\AgdaSpace{}%
\AgdaInductiveConstructor{nothing}\AgdaSpace{}%
\AgdaSymbol{=}\AgdaSpace{}%
\AgdaInductiveConstructor{nothing}\AgdaSpace{}%
\AgdaComment{\,-\,\!-\!\! Computation failed e.\!g.\,div by 0}\<%
\\
%
\>[2]\AgdaSymbol{...}\AgdaSpace{}%
\AgdaSymbol{|}\AgdaSpace{}%
\AgdaBound{f\;}\AgdaSymbol{@\;(}\AgdaInductiveConstructor{just}\AgdaSpace{}%
\AgdaSymbol{\AgdaUnderscore{})}\AgdaSpace{}%
\AgdaKeyword{with}\AgdaSpace{}%
\AgdaFunction{toTruthValue}\AgdaSpace{}%
\AgdaSymbol{\{}\AgdaBound{\,f\,}\AgdaSymbol{\}}\AgdaSpace{}%
\AgdaSymbol{(}\AgdaInductiveConstructor{Any.just}\AgdaSpace{}%
\AgdaInductiveConstructor{tt}\AgdaSymbol{)}\<%
\\
%
\>[2]\AgdaSymbol{...}\AgdaSpace{}%
\AgdaSymbol{|}\AgdaSpace{}%
\AgdaInductiveConstructor{true}\AgdaSpace{}%
\AgdaSymbol{=}\AgdaSpace{}%
\AgdaFunction{ssEvalwithFuel}\AgdaSpace{}%
\AgdaBound{n}\AgdaSpace{}%
\AgdaSymbol{(}\AgdaBound{c₁}\AgdaSpace{}%
\AgdaOperator{\AgdaFunction{𝔱𝔥𝔢𝔫}}\AgdaSpace{}%
\AgdaSymbol{((}\AgdaOperator{\AgdaInductiveConstructor{𝔴𝔥𝔦𝔩𝔢}}\AgdaSpace{}%
\AgdaBound{exp}\AgdaSpace{}%
\AgdaOperator{\AgdaInductiveConstructor{𝒹ℴ}}\AgdaSpace{}%
\AgdaBound{c₁}\AgdaSymbol{)}\AgdaSpace{}%
\AgdaOperator{\AgdaInductiveConstructor{;}}\AgdaSpace{}%
\AgdaBound{c₂}\AgdaSymbol{))}\AgdaSpace{}%
\AgdaBound{s}\<%
\\
%
\>[2]\AgdaSymbol{...}\AgdaSpace{}%
\AgdaSymbol{|}\AgdaSpace{}%
\AgdaInductiveConstructor{false}\AgdaSpace{}%
\AgdaSymbol{=}\AgdaSpace{}%
\AgdaFunction{ssEvalwithFuel}\AgdaSpace{}%
\AgdaBound{n}\AgdaSpace{}%
\AgdaBound{c₂}\AgdaSpace{}%
\AgdaBound{s}\<%
\\
%
\>[2]\AgdaComment{-----------------------------------------------------------------}\<%
\end{code}
}

\lipsum[75]



\lipsum[66]

{\advance\leftskip1.5\mathindent

\begin{code}
%
\>[2]\AgdaFunction{Terminates}\AgdaSpace{}%
\AgdaSymbol{:}\AgdaSpace{}%
\AgdaFunction{C}\AgdaSpace{}%
\AgdaSymbol{→}\AgdaSpace{}%
\AgdaField{S}\AgdaSpace{}%
\AgdaSymbol{→}\AgdaSpace{}%
\AgdaPrimitiveType{Set}\<%
\\
%
\>[2]\AgdaFunction{Terminates}\AgdaSpace{}%
\AgdaBound{c}\AgdaSpace{}%
\AgdaBound{s}\AgdaSpace{}%
\AgdaSymbol{=}\AgdaSpace{}%
\AgdaFunction{Σ[}\AgdaSpace{}%
\AgdaBound{ℱ}\AgdaSpace{}%
\AgdaFunction{$\in$}\AgdaSpace{}%
\AgdaDatatype{ℕ}\AgdaSpace{}%
\AgdaFunction{]}\AgdaSpace{}%
\AgdaSymbol{(}\AgdaSpace{}%
\AgdaFunction{Is-just}\AgdaSpace{}%
\AgdaSymbol{(}\AgdaFunction{ssEvalwithFuel}\AgdaSpace{}%
\AgdaBound{ℱ}\AgdaSpace{}%
\AgdaBound{c}\AgdaSpace{}%
\AgdaBound{s}\AgdaSpace{}%
\AgdaSymbol{))}\<%
\end{code}

\vspace{-0.8cm}

\begin{code}
%
\>[2]\AgdaOperator{\AgdaFunction{⌊ᵗ\AgdaUnderscore{}⸴\AgdaUnderscore{}ᵗ⌋}}\AgdaSpace{}%
\AgdaSymbol{:}\AgdaSpace{}%
\AgdaFunction{C}\AgdaSpace{}%
\AgdaSymbol{→}\AgdaSpace{}%
\AgdaField{S}\AgdaSpace{}%
\AgdaSymbol{→}\AgdaSpace{}%
\AgdaPrimitiveType{Set}\<%
\\
%
\>[2]\AgdaOperator{\AgdaFunction{⌊ᵗ\AgdaUnderscore{}⸴\AgdaUnderscore{}ᵗ⌋}}\AgdaSpace{}%
\AgdaSymbol{=}\AgdaSpace{}%
\AgdaFunction{Terminates}\<%
\end{code}

\vspace{-0.8cm}

\begin{code}
%
\>[2]\AgdaFunction{TerminatesWith}\AgdaSpace{}%
\AgdaSymbol{:}\AgdaSpace{}%
\AgdaDatatype{ℕ}\AgdaSpace{}%
\AgdaSymbol{→}\AgdaSpace{}%
\AgdaFunction{C}\AgdaSpace{}%
\AgdaSymbol{→}\AgdaSpace{}%
\AgdaField{S}\AgdaSpace{}%
\AgdaSymbol{→}\AgdaSpace{}%
\AgdaPrimitiveType{Set}\<%
\\
%
\>[2]\AgdaFunction{TerminatesWith}\AgdaSpace{}%
\AgdaBound{ℱ}\AgdaSpace{}%
\AgdaBound{c}\AgdaSpace{}%
\AgdaBound{s}\AgdaSpace{}%
\AgdaSymbol{=}\AgdaSpace{}%
\AgdaFunction{Is-just}\AgdaSpace{}%
\AgdaSymbol{(}\AgdaFunction{ssEvalwithFuel}\AgdaSpace{}%
\AgdaBound{ℱ}\AgdaSpace{}%
\AgdaBound{c}\AgdaSpace{}%
\AgdaBound{s}\AgdaSymbol{)}\<%
\end{code}

\vspace{-0.8cm}

\begin{code}
  \>[2]\AgdaOperator{\AgdaFunction{⌊ᵗ\AgdaUnderscore{}⸴\AgdaUnderscore{}⸴\AgdaUnderscore{}ᵗ⌋}}\AgdaSpace{}%
\AgdaSymbol{:}\AgdaSpace{}%
\AgdaDatatype{ℕ}\AgdaSpace{}%
\AgdaSymbol{→}\AgdaSpace{}%
\AgdaFunction{C}\AgdaSpace{}%
\AgdaSymbol{→}\AgdaSpace{}%
\AgdaField{S}\AgdaSpace{}%
\AgdaSymbol{→}\AgdaSpace{}%
\AgdaPrimitiveType{Set}\<%
\\
%
\>[2]\AgdaOperator{\AgdaFunction{⌊ᵗ}}\AgdaSpace{}%
\AgdaBound{ℱ}\AgdaSpace{}%
\AgdaOperator{\AgdaFunction{⸴}}\AgdaSpace{}%
\AgdaBound{c}\AgdaSpace{}%
\AgdaOperator{\AgdaFunction{⸴}}\AgdaSpace{}%
\AgdaBound{s}\AgdaSpace{}%
\AgdaOperator{\AgdaFunction{ᵗ⌋}}\AgdaSpace{}%
\AgdaSymbol{=}\AgdaSpace{}%
\AgdaFunction{TerminatesWith}\AgdaSpace{}%
\AgdaBound{ℱ}\AgdaSpace{}%
\AgdaBound{c}\AgdaSpace{}%
\AgdaBound{s}\<%
\end{code}

}

\lipsum[75]

{\small \begin{code}
\>[2]\AgdaFunction{EvalDet}\AgdaSpace{}%
\AgdaSymbol{:}\AgdaSpace{}%
\AgdaSymbol{∀}\AgdaSpace{}%
\AgdaSymbol{\{}\AgdaBound{s}\AgdaSpace{}%
\AgdaBound{ℱ}%
\>[325I]\AgdaBound{ℱ'\,}\AgdaSymbol{\}}\AgdaSpace{}%
\AgdaBound{C}\AgdaSpace{}\AgdaSymbol{→}\AgdaSpace{}%
\AgdaSymbol{(}\AgdaBound{a}\AgdaSpace{}%
\AgdaSymbol{:}\AgdaSpace{}%
\AgdaOperator{\AgdaFunction{⌊ᵗ}}\AgdaSpace{}%
\AgdaBound{ℱ}\AgdaSpace{}%
\AgdaOperator{\AgdaFunction{⸴}}\AgdaSpace{}%
\AgdaBound{C}\AgdaSpace{}%
\AgdaOperator{\AgdaFunction{⸴}}\AgdaSpace{}%
\AgdaBound{s}\AgdaSpace{}%
\AgdaOperator{\AgdaFunction{ᵗ⌋}}\AgdaSymbol{)}\AgdaSpace{}%
\AgdaSymbol{→}\AgdaSpace{}%
\AgdaSymbol{(}\AgdaBound{b}\AgdaSpace{}%
\AgdaSymbol{:}\AgdaSpace{}%
\AgdaOperator{\AgdaFunction{⌊ᵗ}}\AgdaSpace{}%
\AgdaBound{ℱ'}\AgdaSpace{}%
\AgdaOperator{\AgdaFunction{⸴}}\AgdaSpace{}%
\AgdaBound{C}\AgdaSpace{}%
\AgdaOperator{\AgdaFunction{⸴}}\AgdaSpace{}%
\AgdaBound{s}\AgdaSpace{}%
\AgdaOperator{\AgdaFunction{ᵗ⌋}}\AgdaSymbol{)}\AgdaSpace{}%
\AgdaSymbol{→}\AgdaSpace{}%
\AgdaFunction{″}\AgdaSpace{}%
\AgdaBound{a}\AgdaSpace{}%
\AgdaOperator{\AgdaDatatype{≡}}\AgdaSpace{}%
\AgdaFunction{″}\AgdaSpace{}%
\AgdaBound{b}\<%
\end{code}}


\lipsum[34]

{\centering \begin{code}
  \>[2]\AgdaKeyword{record}\AgdaSpace{}%
\AgdaRecord{Split-⌊ᵗ⌋}\AgdaSpace{}%
\AgdaBound{s}\AgdaSpace{}%
\AgdaBound{ℱ}\AgdaSpace{}%
\AgdaBound{Q₁}\AgdaSpace{}%
\AgdaBound{Q₂}\AgdaSpace{}%
\AgdaSymbol{(}\AgdaBound{ϕ}\AgdaSpace{}%
\AgdaSymbol{:}\AgdaSpace{}%
\AgdaOperator{\AgdaFunction{⌊ᵗ}}\AgdaSpace{}%
\AgdaBound{ℱ}\AgdaSpace{}%
\AgdaOperator{\AgdaFunction{⸴}}\AgdaSpace{}%
\AgdaBound{Q₁}\AgdaSpace{}%
\AgdaOperator{\AgdaFunction{𝔱𝔥𝔢𝔫}}\AgdaSpace{}%
\AgdaBound{Q₂}\AgdaSpace{}%
\AgdaOperator{\AgdaFunction{⸴}}\AgdaSpace{}%
\AgdaBound{s}\AgdaSpace{}%
\AgdaOperator{\AgdaFunction{ᵗ⌋}}\AgdaSymbol{)}\AgdaSpace{}%
\AgdaSymbol{:}\AgdaSpace{}%
\AgdaPrimitiveType{Set}\AgdaSpace{}%
\AgdaKeyword{where}\<%
\\
\>[2][@{}l@{\AgdaIndent{0}}]%
\>[4]\AgdaKeyword{field}\<%
\\
\>[4][@{}l@{\AgdaIndent{0}}]%
\>[6]\AgdaComment{----- Termination Left}\<%
\\
%
\>[6]\AgdaField{Lᵗ}%
\>[12]\AgdaSymbol{:}\AgdaSpace{}%
\AgdaOperator{\AgdaFunction{⌊ᵗ}}\AgdaSpace{}%
\AgdaBound{ℱ}\AgdaSpace{}%
\AgdaOperator{\AgdaFunction{⸴}}%
\>[22]\AgdaBound{Q₁}\AgdaSpace{}%
\AgdaOperator{\AgdaFunction{⸴}}\AgdaSpace{}%
\AgdaBound{s}\AgdaSpace{}%
\AgdaOperator{\AgdaFunction{ᵗ⌋}}\<%
\\
%
\>[6]\AgdaComment{----- There's an ℱ' s.t.}\<%
\\
%
\>[6]\AgdaField{ℱ'}%
\>[11]\AgdaSymbol{:}\AgdaSpace{}%
\AgdaDatatype{ℕ}\<%
\\
%
\>[6]\AgdaComment{----- Termination Right}\<%
\\
%
\>[6]\AgdaField{Rᵗ}%
\>[12]\AgdaSymbol{:}\AgdaSpace{}%
\AgdaOperator{\AgdaFunction{⌊ᵗ}}\AgdaSpace{}%
\AgdaField{ℱ'}\AgdaSpace{}%
\AgdaOperator{\AgdaFunction{⸴}}\AgdaSpace{}%
\AgdaBound{Q₂}%
\>[26]\AgdaOperator{\AgdaFunction{⸴}}\AgdaSpace{}%
\AgdaSymbol{(}\AgdaFunction{″}\AgdaSpace{}%
\AgdaField{Lᵗ}\AgdaSymbol{)}\AgdaSpace{}%
\AgdaOperator{\AgdaFunction{ᵗ⌋}}\<%
\\
%
\>[6]\AgdaComment{----- and 2nd proof fuel is less than starting fuel:}\<%
\\
%
\>[6]\AgdaField{lt}\AgdaSpace{}%
\AgdaSymbol{:}\AgdaSpace{}%
\AgdaField{ℱ'}\AgdaSpace{}%
\AgdaOperator{\AgdaRecord{≤''}}\AgdaSpace{}%
\AgdaBound{ℱ}\<%
\\
%
\>[6]\AgdaComment{----- And the output unchanged:}\<%
\\
%
\>[6]\AgdaField{Δ}%
\>[9]\AgdaSymbol{:}\AgdaSpace{}%
\AgdaFunction{″}\AgdaSpace{}%
\AgdaField{Rᵗ}\AgdaSpace{}%
\AgdaOperator{\AgdaDatatype{≡}}\AgdaSpace{}%
\AgdaFunction{″}\AgdaSpace{}%
\AgdaBound{ϕ}\<%
\end{code}}

\lipsum[67]

{\centering \begin{code}
\>[2]\AgdaOperator{\AgdaFunction{⟪\AgdaUnderscore{}⟫\AgdaUnderscore{}⟪\AgdaUnderscore{}⟫}}\AgdaSpace{}%
\AgdaSymbol{:}%
\>[13]\AgdaFunction{Assertion}\AgdaSpace{}%
\AgdaSymbol{→}\AgdaSpace{}%
\AgdaFunction{C}\AgdaSpace{}%
\AgdaSymbol{→}\AgdaSpace{}%
\AgdaFunction{Assertion}\AgdaSpace{}%
\AgdaSymbol{→}\AgdaSpace{}%
\AgdaPrimitiveType{Set}\<%
\\
%
\>[2]\AgdaOperator{\AgdaFunction{⟪}}\AgdaSpace{}%
\AgdaBound{P}\AgdaSpace{}%
\AgdaOperator{\AgdaFunction{⟫}}\AgdaSpace{}%
\AgdaBound{C}\AgdaSpace{}%
\AgdaOperator{\AgdaFunction{⟪}}\AgdaSpace{}%
\AgdaBound{Q}\AgdaSpace{}%
\AgdaOperator{\AgdaFunction{⟫}}\AgdaSpace{}%
\AgdaSymbol{=}\AgdaSpace{}%
\AgdaSymbol{(}\AgdaSpace{}%
\AgdaBound{s}\AgdaSpace{}%
\AgdaSymbol{:}\AgdaSpace{}%
\AgdaField{S}\AgdaSpace{}%
\AgdaSymbol{)}\AgdaSpace{}%
\AgdaSymbol{→}\AgdaSpace{}%
\AgdaBound{s}\AgdaSpace{}%
\AgdaOperator{\AgdaFunction{⊨}}\AgdaSpace{}%
\AgdaBound{P}\AgdaSpace{}%
\AgdaSymbol{→}\AgdaSpace{}%
\AgdaSymbol{(}\AgdaBound{ϕ}\AgdaSpace{}%
\AgdaSymbol{:}\AgdaSpace{}%
\AgdaOperator{\AgdaFunction{⌊ᵗ}}\AgdaSpace{}%
\AgdaBound{C}\AgdaSpace{}%
\AgdaOperator{\AgdaFunction{⸴}}\AgdaSpace{}%
\AgdaBound{s}\AgdaSpace{}%
\AgdaOperator{\AgdaFunction{ᵗ⌋}}\AgdaSymbol{)}\AgdaSpace{}%
\AgdaSymbol{→}\AgdaSpace{}%
\AgdaSymbol{(}\AgdaFunction{‵}\AgdaSpace{}%
\AgdaBound{ϕ}\AgdaSymbol{)}\AgdaSpace{}%
\AgdaOperator{\AgdaFunction{⊨}}\AgdaSpace{}%
\AgdaBound{Q}\<%
\end{code}}

\lipsum[66]

{\centering \input{agda-snippets/semantics-tc-trip}}

\lipsum[66]

{\centering \begin{code}
  \>[2]\AgdaFunction{LawOfExcludedMiracle-\agdamath{wp(}:=,-\agdamath{)}}\AgdaSpace{}%
\AgdaSymbol{:}\AgdaSpace{}%
\AgdaSymbol{∀}\AgdaSpace{}%
\AgdaSymbol{\{}\AgdaBound{i}\AgdaSpace{}%
\AgdaBound{e}\AgdaSymbol{\}}\AgdaSpace{}%
\AgdaSymbol{→}\AgdaSpace{}%
\AgdaFunction{sub}\AgdaSpace{}%
\AgdaBound{e}\AgdaSpace{}%
\AgdaBound{i}\AgdaSpace{}%
\AgdaFunction{\;\agdamath{F}}\AgdaSpace{}%
\AgdaOperator{\AgdaDatatype{≡}}\AgdaSpace{}%
\AgdaFunction{\agdamath{F}}\<%
\\
%
\>[2]\AgdaFunction{LawOfExcludedMiracle-\agdamath{wp(}:=,-\agdamath{)}}\AgdaSpace{}%
\AgdaSymbol{=}\AgdaSpace{}%
\AgdaInductiveConstructor{refl}\<%
\end{code}}

\lipsum[66]



\subsection{Language}


\subsection{Axioms \& Rules}


\lipsum[75]

\lipsum[66]

\lipsum[66]


\vbox{\centering
  \begin{code}
  \>[2]\hspace{-2.5cm}\AgdaFunction{D0-Axiom-of-Assignment}\AgdaSpace{}%
\AgdaSymbol{:}\AgdaSpace{}%
\AgdaSymbol{∀}\AgdaSpace{}%
\AgdaBound{i}\AgdaSpace{}%
\AgdaBound{e}\AgdaSpace{}%
\AgdaBound{P}\<%
\\
%
\\[\AgdaEmptyExtraSkip]%
%
\\[\AgdaEmptyExtraSkip]%
\>[2]\AgdaComment{-\,\!- ━━━━━━━━━━━ --}\<%
\\
\>[2][@{}l@{\AgdaIndent{0}}]%
\>[7]\AgdaSymbol{\;\;\;\;→}\AgdaSpace{}%
\AgdaOperator{\AgdaFunction{⟪}}\AgdaSpace{}%
\AgdaSymbol{(}\AgdaFunction{sub}\AgdaSpace{}%
\AgdaBound{e}\AgdaSpace{}%
\AgdaBound{i}\AgdaSpace{}%
\AgdaBound{P}\AgdaSymbol{)}\AgdaSpace{}%
\AgdaOperator{\AgdaFunction{⟫}}%
\>[26]\AgdaSymbol{(}\AgdaSpace{}%
\AgdaBound{i}\AgdaSpace{}%
\AgdaOperator{\AgdaInductiveConstructor{:=}}\AgdaSpace{}%
\AgdaBound{e}\AgdaSpace{}%
\AgdaOperator{\AgdaInductiveConstructor{;}}\AgdaSpace{}%
\AgdaSymbol{)}\AgdaSpace{}%
\AgdaOperator{\AgdaFunction{⟪}}\AgdaSpace{}%
\AgdaBound{P}\AgdaSpace{}%
\AgdaOperator{\AgdaFunction{⟫}}\<%
\end{code}
}

\lipsum[66]

\vbox{\centering
  \input{agda-snippets/rules-cons-post}
}

\vbox{\centering
  \begin{code}
\>[2]\hspace{-2.5cm}\AgdaFunction{D1-Rule-of-Consequence-pre}\AgdaSpace{}%
\AgdaSymbol{:}\AgdaSpace{}%
\AgdaSymbol{∀}\AgdaSpace{}%
\AgdaSymbol{\{}\AgdaBound{P}\AgdaSymbol{\}}\AgdaSpace{}%
\AgdaSymbol{\{}\AgdaBound{Q}\AgdaSymbol{\}}\AgdaSpace{}%
\AgdaSymbol{\{}\AgdaBound{R}\AgdaSymbol{\}}\AgdaSpace{}%
\AgdaSymbol{\{}\AgdaBound{S}\AgdaSymbol{\}}\<%
\\
%
\\[\AgdaEmptyExtraSkip]%
\>[2][@{}l@{\AgdaIndent{0}}]%
\>[6]\AgdaSymbol{\hspace{0.75cm}→}\AgdaSpace{}%
\AgdaOperator{\AgdaFunction{⟪}}\AgdaSpace{}%
\AgdaBound{P}\AgdaSpace{}%
\AgdaOperator{\AgdaFunction{⟫}}\AgdaSpace{}%
\AgdaBound{Q}\AgdaSpace{}%
\AgdaOperator{\AgdaFunction{⟪}}\AgdaSpace{}%
\AgdaBound{R}\AgdaSpace{}%
\AgdaOperator{\AgdaFunction{⟫}}\AgdaSpace{}%
\AgdaSymbol{→}\AgdaSpace{}%
\AgdaBound{S}\AgdaSpace{}%
\AgdaOperator{\AgdaFunction{⇒}}\AgdaSpace{}%
\AgdaBound{P}\<%
\\
%
\>[2]\AgdaComment{-\,\!- ━━━━━━━━━━━ --}\<%
\\
\>[2][@{}l@{\AgdaIndent{0}}]%
\>[11]\AgdaSymbol{\hspace{0.75cm}→}\AgdaSpace{}%
\AgdaOperator{\AgdaFunction{⟪}}\AgdaSpace{}%
\AgdaBound{S}\AgdaSpace{}%
\AgdaOperator{\AgdaFunction{⟫}}\AgdaSpace{}%
\AgdaBound{Q}\AgdaSpace{}%
\AgdaOperator{\AgdaFunction{⟪}}\AgdaSpace{}%
\AgdaBound{R}\AgdaSpace{}%
\AgdaOperator{\AgdaFunction{⟫}}\<%
\end{code}
}

\lipsum[66]

\vbox{\centering
  \begin{code}
\>[2]\hspace{-1.4cm}\AgdaFunction{D2-Rule-of-Composition}\AgdaSpace{}%
\AgdaSymbol{:}\AgdaSpace{}%
\AgdaSymbol{∀}\AgdaSpace{}%
\AgdaSymbol{\{}\AgdaBound{P}\AgdaSymbol{\}}\AgdaSpace{}%
\AgdaSymbol{\{}\AgdaBound{R₁}\AgdaSymbol{\}}\AgdaSpace{}%
\AgdaSymbol{\{}\AgdaBound{R}\AgdaSymbol{\}}\AgdaSpace{}%
\AgdaSymbol{\{}\AgdaBound{Q₁}\AgdaSymbol{\}}\AgdaSpace{}%
\AgdaSymbol{\{}\AgdaBound{Q₂}\AgdaSymbol{\}}\<%
\\
%
\\[\AgdaEmptyExtraSkip]%
\>[2][@{}l@{\AgdaIndent{0}}]%
\>[8]\AgdaSymbol{\hspace{0.75cm}→}\AgdaSpace{}%
\AgdaOperator{\AgdaFunction{⟪}}\AgdaSpace{}%
\AgdaBound{P}\AgdaSpace{}%
\AgdaOperator{\AgdaFunction{⟫}}\AgdaSpace{}%
\AgdaBound{Q₁}\AgdaSpace{}%
\AgdaOperator{\AgdaFunction{⟪}}\AgdaSpace{}%
\AgdaBound{R₁}\AgdaSpace{}%
\AgdaOperator{\AgdaFunction{⟫}}\AgdaSpace{}%
\AgdaSymbol{→}\AgdaSpace{}%
\AgdaOperator{\AgdaFunction{⟪}}\AgdaSpace{}%
\AgdaBound{R₁}\AgdaSpace{}%
\AgdaOperator{\AgdaFunction{⟫}}\AgdaSpace{}%
\AgdaBound{Q₂}\AgdaSpace{}%
\AgdaOperator{\AgdaFunction{⟪}}\AgdaSpace{}%
\AgdaBound{R}\AgdaSpace{}%
\AgdaOperator{\AgdaFunction{⟫}}\<%
\\
%
\>[2]\AgdaComment{-\,\!- ━━━━━━━━━━━━━━ -\,\!-}\<%
\\
\>[2][@{}l@{\AgdaIndent{0}}]%
\>[12]\AgdaSymbol{\hspace{0.75cm}→}\AgdaSpace{}%
\AgdaOperator{\AgdaFunction{⟪}}\AgdaSpace{}%
\AgdaBound{P}\AgdaSpace{}%
\AgdaOperator{\AgdaFunction{⟫}}\AgdaSpace{}%
\AgdaBound{Q₁}\AgdaSpace{}%
\AgdaOperator{\AgdaFunction{𝔱𝔥𝔢𝔫}}\AgdaSpace{}%
\AgdaBound{Q₂}\AgdaSpace{}%
\AgdaOperator{\AgdaFunction{⟪}}\AgdaSpace{}%
\AgdaBound{R}\AgdaSpace{}%
\AgdaOperator{\AgdaFunction{⟫}}\<%
\end{code}
}

\lipsum[66]

\vbox{\centering
  \begin{code}
  \>[2]\hspace{-1.4cm}\AgdaFunction{D3-While-Rule}%
\>[254I]\AgdaSymbol{:}\AgdaSpace{}%
\AgdaSymbol{∀}\AgdaSpace{}%
\AgdaSymbol{\{}\AgdaBound{P}\AgdaSymbol{\}}\AgdaSpace{}%
\AgdaSymbol{\{}\AgdaBound{B}\AgdaSymbol{\}}\AgdaSpace{}%
\AgdaSymbol{\{}\AgdaBound{C}\AgdaSymbol{\}}\<%
\\
%
\\[\AgdaEmptyExtraSkip]%
\>[254I][@{}l@{\AgdaIndent{0}}]%
\>[17]\AgdaSymbol{\hspace{0.75cm}→}\AgdaSpace{}%
\AgdaOperator{\AgdaFunction{⟪}}\AgdaSpace{}%
\AgdaBound{P}\AgdaSpace{}%
\AgdaOperator{\AgdaFunction{∧}}\AgdaSpace{}%
\AgdaBound{B}\AgdaSpace{}%
\AgdaOperator{\AgdaFunction{⟫}}\AgdaSpace{}%
\AgdaBound{C}\AgdaSpace{}%
\AgdaOperator{\AgdaFunction{⟪}}\AgdaSpace{}%
\AgdaBound{P}\AgdaSpace{}%
\AgdaOperator{\AgdaFunction{⟫}}\<%
\\
%
\>[2]\AgdaComment{-\,\!- ━━━━━━━━━━━━━━ -\,\!-}\<%
\\
\>[2][@{}l@{\AgdaIndent{0}}]%
\>[8]\AgdaSymbol{\hspace{0.75cm}→}\AgdaSpace{}%
\AgdaOperator{\AgdaFunction{⟪}}\AgdaSpace{}%
\AgdaBound{P}\AgdaSpace{}%
\AgdaOperator{\AgdaFunction{⟫}}\AgdaSpace{}%
\AgdaOperator{\AgdaInductiveConstructor{𝔴𝔥𝔦𝔩𝔢}}\AgdaSpace{}%
\AgdaBound{B}\AgdaSpace{}%
\AgdaOperator{\AgdaInductiveConstructor{𝒹ℴ}}\AgdaSpace{}%
\AgdaBound{C}\AgdaSpace{}%
\AgdaOperator{\AgdaInductiveConstructor{;}}\AgdaSpace{}%
\AgdaOperator{\AgdaFunction{⟪}}\AgdaSpace{}%
\AgdaSymbol{(}\AgdaFunction{\;𝑛𝑜𝑡}\AgdaSpace{}%
\AgdaBound{B}\AgdaSymbol{\;)}\AgdaSpace{}%
\AgdaOperator{\AgdaFunction{∧}}\AgdaSpace{}%
\AgdaBound{P}\AgdaSpace{}%
\AgdaOperator{\AgdaFunction{⟫}}\<%
\end{code}
}

\lipsum[66]

\vbox{\centering
  \begin{code}
  \>[2]\AgdaFunction{D4-Conditional-Rule}\AgdaSpace{}%
\AgdaSymbol{:}\AgdaSpace{}%
\AgdaSymbol{∀}\AgdaSpace{}%
\AgdaSymbol{\{}\AgdaBound{A}\AgdaSymbol{\}}\AgdaSpace{}%
\AgdaSymbol{\{}\AgdaBound{B}\AgdaSymbol{\}}\AgdaSpace{}%
\AgdaSymbol{\{}\AgdaBound{C}\AgdaSymbol{\}}\AgdaSpace{}%
\AgdaSymbol{\{}\AgdaBound{P}\AgdaSymbol{\}}\AgdaSpace{}%
\AgdaSymbol{\{}\AgdaBound{Q}\AgdaSymbol{\}}\<%
\\
%
\\[\AgdaEmptyExtraSkip]%
\>[2][@{}l@{\AgdaIndent{0}}]%
\>[6]\AgdaSymbol{\hspace{0.75cm}→}\AgdaSpace{}%
\AgdaOperator{\AgdaFunction{⟪}}\AgdaSpace{}%
\AgdaBound{C}\AgdaSpace{}%
\AgdaOperator{\AgdaFunction{∧}}\AgdaSpace{}%
\AgdaBound{P}\AgdaSpace{}%
\AgdaOperator{\AgdaFunction{⟫}}\AgdaSpace{}%
\AgdaBound{A}\AgdaSpace{}%
\AgdaOperator{\AgdaFunction{⟪}}\AgdaSpace{}%
\AgdaBound{Q}\AgdaSpace{}%
\AgdaOperator{\AgdaFunction{⟫}}\AgdaSpace{}%
\AgdaSymbol{→}\AgdaSpace{}%
\AgdaOperator{\AgdaFunction{⟪}}\AgdaSpace{}%
\AgdaSymbol{(}\AgdaFunction{\;𝑛𝑜𝑡}\AgdaSpace{}%
\AgdaBound{C}\AgdaSymbol{\;)}\AgdaSpace{}%
\AgdaOperator{\AgdaFunction{∧}}\AgdaSpace{}%
\AgdaBound{P}\AgdaSpace{}%
\AgdaOperator{\AgdaFunction{⟫}}\AgdaSpace{}%
\AgdaBound{B}\AgdaSpace{}%
\AgdaOperator{\AgdaFunction{⟪}}\AgdaSpace{}%
\AgdaBound{Q}\AgdaSpace{}%
\AgdaOperator{\AgdaFunction{⟫}}\<%
\\
%
\>[2]\AgdaComment{-\,\!- ━━━━━━━━━━━━━━━━━━ --}\<%
\\
\>[2][@{}l@{\AgdaIndent{0}}]%
\>[14]\AgdaSymbol{\hspace{0.75cm}→}\AgdaSpace{}%
\AgdaOperator{\AgdaFunction{⟪}}\AgdaSpace{}%
\AgdaBound{P}\AgdaSpace{}%
\AgdaOperator{\AgdaFunction{⟫}}%
\>[23]\AgdaOperator{\AgdaInductiveConstructor{𝔦𝔣}}\AgdaSpace{}%
\AgdaBound{C}\AgdaSpace{}%
\AgdaOperator{\AgdaInductiveConstructor{𝔱𝔥𝔢𝔫}}\AgdaSpace{}%
\AgdaBound{A}\AgdaSpace{}%
\AgdaOperator{\AgdaInductiveConstructor{𝔢𝔩𝔰𝔢}}\AgdaSpace{}%
\AgdaBound{B}\AgdaSpace{}%
\AgdaOperator{\AgdaInductiveConstructor{;}}\AgdaSpace{}%
\AgdaOperator{\AgdaFunction{⟪}}\AgdaSpace{}%
\AgdaBound{Q}\AgdaSpace{}%
\AgdaOperator{\AgdaFunction{⟫}}\<%
\end{code}
}



\begin{figure}
  \caption{D3-While: Full proof of the while rule; the crucial rule for reasoning with Hoare Logic:}
  \centering
  \small
  \begin{code}
  \>[2]\AgdaFunction{D3-While-Rule}\AgdaSpace{}%
\AgdaSymbol{\{}\AgdaBound{P}\AgdaSymbol{\}}\AgdaSpace{}%
\AgdaSymbol{\{}\AgdaBound{B}\AgdaSymbol{\}}\AgdaSpace{}%
\AgdaSymbol{\{}\AgdaBound{C}\AgdaSymbol{\}}\AgdaSpace{}%
\AgdaBound{PBCP}\AgdaSpace{}%
\AgdaBound{s}\AgdaSpace{}%
\AgdaBound{⊨P}\AgdaSpace{}%
\AgdaSymbol{(}\AgdaInductiveConstructor{suc}\AgdaSpace{}%
\AgdaBound{ℱ}\AgdaSpace{}%
\AgdaOperator{\AgdaInductiveConstructor{,}}\AgdaSpace{}%
\AgdaBound{⌊ᵗC\,ᵗ⌋}\AgdaSymbol{)}\AgdaSpace{}%
\AgdaSymbol{=}\AgdaSpace{}%
\AgdaFunction{go}\AgdaSpace{}%
\AgdaSymbol{(}\AgdaInductiveConstructor{suc}\AgdaSpace{}%
\AgdaBound{ℱ}\AgdaSymbol{)}\AgdaSpace{}%
\AgdaBound{⊨P}\AgdaSpace{}%
\AgdaBound{⌊ᵗC\,ᵗ⌋}\<%
\\
\>[2][@{}l@{\AgdaIndent{0}}]%
\>[6]\AgdaKeyword{where}\<%
\\
%
\>[6]\AgdaComment{------------------------------------------------------------------------------------------------------------------------}\<%
\\
%
\>[6]\AgdaComment{-- Using mutually recursive functions go and go-true      }\<%
\\
%
\>[6]\AgdaFunction{go}\AgdaSpace{}%
\AgdaSymbol{:}%
\>[633I]\AgdaSymbol{∀}\AgdaSpace{}%
\AgdaSymbol{\{}\AgdaBound{s}\AgdaSymbol{\}}\AgdaSpace{}%
\AgdaBound{ℱ}\AgdaSpace{}%
\AgdaSymbol{→}\AgdaSpace{}%
\AgdaBound{s}\AgdaSpace{}%
\AgdaOperator{\AgdaFunction{⊨}}\AgdaSpace{}%
\AgdaBound{P}\AgdaSpace{}%
\AgdaSymbol{→}\AgdaSpace{}%
\AgdaSymbol{(}\AgdaBound{⌊ᵗC\,ᵗ⌋}\AgdaSpace{}%
\AgdaSymbol{:}\AgdaSpace{}%
\AgdaOperator{\AgdaFunction{⌊ᵗ}}\AgdaSpace{}%
\AgdaBound{ℱ}\AgdaSpace{}%
\AgdaOperator{\AgdaFunction{⸴}}\AgdaSpace{}%
\AgdaSymbol{(}\AgdaOperator{\AgdaInductiveConstructor{𝔴𝔥𝔦𝔩𝔢}}\AgdaSpace{}%
\AgdaBound{B}\AgdaSpace{}%
\AgdaOperator{\AgdaInductiveConstructor{𝒹ℴ}}\AgdaSpace{}%
\AgdaBound{C}\AgdaSpace{}%
\AgdaOperator{\AgdaInductiveConstructor{;}}\AgdaSymbol{)}\AgdaSpace{}%
\AgdaOperator{\AgdaFunction{⸴}}\AgdaSpace{}%
\AgdaBound{s}\AgdaSpace{}%
\AgdaOperator{\AgdaFunction{ᵗ⌋}}\AgdaSymbol{)}\<%
\\
\>[.][@{}l@{}]\<[633I]%
\>[11]\AgdaSymbol{→}\AgdaSpace{}%
\AgdaSymbol{(}\AgdaFunction{″}\AgdaSpace{}%
\AgdaBound{⌊ᵗC\,ᵗ⌋}\AgdaSymbol{)}\AgdaSpace{}%
\AgdaOperator{\AgdaFunction{⊨}}\AgdaSpace{}%
\AgdaSymbol{(}\AgdaInductiveConstructor{op₂}\AgdaSpace{}%
\AgdaSymbol{(}\AgdaInductiveConstructor{op₁}\AgdaSpace{}%
\AgdaInductiveConstructor{¬ₒ}\AgdaSpace{}%
\AgdaBound{B}\AgdaSymbol{)}\AgdaSpace{}%
\AgdaInductiveConstructor{\&\&ₒ}\AgdaSpace{}%
\AgdaBound{P}\AgdaSpace{}%
\AgdaSymbol{)}\<%
\\
%
\>[6]\AgdaComment{-- ℱ needs to be an argument by itself outside the Sigma type}\<%
\\
%
\>[6]\AgdaComment{-- so we can recurse on it as Agda can't see it always}\<%
\\
%
\>[6]\AgdaComment{-- decrements with each call if it is inside the product.}\<%
\\
%
\>[6]\AgdaComment{---------------------------------------------------------------------------------------------------------------------------}\<%
\\
%
\>[6]\AgdaComment{-- case where B is true}\<%
\\
%
\>[6]\AgdaFunction{go-true}%
\>[664I]\AgdaSymbol{:}\AgdaSpace{}%
\AgdaSymbol{∀}\AgdaSpace{}%
\AgdaSymbol{\{}\AgdaBound{s}\AgdaSymbol{\}}\AgdaSpace{}%
\AgdaSymbol{\{}\AgdaBound{ℱ}\AgdaSymbol{\}}\AgdaSpace{}%
\AgdaSymbol{\{}\AgdaBound{v}\AgdaSymbol{\}}\AgdaSpace{}%
\AgdaSymbol{→}\AgdaSpace{}%
\AgdaBound{s}\AgdaSpace{}%
\AgdaOperator{\AgdaFunction{⊨}}\AgdaSpace{}%
\AgdaBound{P}\AgdaSpace{}%
\AgdaSymbol{→}\AgdaSpace{}%
\AgdaSymbol{(}\AgdaFunction{evalExp}\AgdaSpace{}%
\AgdaBound{B}\AgdaSpace{}%
\AgdaBound{s}\AgdaSpace{}%
\AgdaOperator{\AgdaDatatype{≡}}\AgdaSpace{}%
\AgdaInductiveConstructor{just}\AgdaSpace{}%
\AgdaBound{v}\AgdaSymbol{)}\<%
\\
\>[.][@{}l@{}]\<[664I]%
\>[14]\AgdaSymbol{→}\AgdaSpace{}%
\AgdaSymbol{(}\AgdaFunction{toTruthValue}\AgdaSpace{}%
\AgdaSymbol{\{}\AgdaInductiveConstructor{just}\AgdaSpace{}%
\AgdaBound{v}\AgdaSymbol{\}}\AgdaSpace{}%
\AgdaSymbol{(}\AgdaInductiveConstructor{just}\AgdaSpace{}%
\AgdaInductiveConstructor{tt}\AgdaSymbol{)}\AgdaSpace{}%
\AgdaOperator{\AgdaDatatype{≡}}\AgdaSpace{}%
\AgdaInductiveConstructor{true}\AgdaSymbol{)}\<%
\\
%
\>[14]\AgdaSymbol{→}\AgdaSpace{}%
\AgdaSymbol{(}\AgdaBound{⌊ᵗC\,ᵗ⌋}\AgdaSpace{}%
\AgdaSymbol{:}\AgdaSpace{}%
\AgdaOperator{\AgdaFunction{⌊ᵗ}}\AgdaSpace{}%
\AgdaBound{ℱ}\AgdaSpace{}%
\AgdaOperator{\AgdaFunction{⸴}}\AgdaSpace{}%
\AgdaSymbol{(}\AgdaBound{C}\AgdaSpace{}%
\AgdaOperator{\AgdaFunction{𝔱𝔥𝔢𝔫}}\AgdaSpace{}%
\AgdaOperator{\AgdaInductiveConstructor{𝔴𝔥𝔦𝔩𝔢}}\AgdaSpace{}%
\AgdaBound{B}\AgdaSpace{}%
\AgdaOperator{\AgdaInductiveConstructor{𝒹ℴ}}\AgdaSpace{}%
\AgdaBound{C}\AgdaSpace{}%
\AgdaOperator{\AgdaInductiveConstructor{;}}\AgdaSymbol{)}\AgdaSpace{}%
\AgdaOperator{\AgdaFunction{⸴}}\AgdaSpace{}%
\AgdaBound{s}\AgdaSpace{}%
\AgdaOperator{\AgdaFunction{ᵗ⌋}}\AgdaSymbol{)}\<%
\\
%
\>[14]\AgdaSymbol{→}\AgdaSpace{}%
\AgdaSymbol{(}\AgdaFunction{to-witness}\AgdaSpace{}%
\AgdaBound{⌊ᵗC\,ᵗ⌋}\AgdaSymbol{)}\AgdaSpace{}%
\AgdaOperator{\AgdaFunction{⊨}}\AgdaSpace{}%
\AgdaSymbol{(}\AgdaInductiveConstructor{op₂}\AgdaSpace{}%
\AgdaSymbol{(}\AgdaInductiveConstructor{op₁}\AgdaSpace{}%
\AgdaInductiveConstructor{¬ₒ}\AgdaSpace{}%
\AgdaBound{B}\AgdaSymbol{)}\AgdaSpace{}%
\AgdaInductiveConstructor{\&\&ₒ}\AgdaSpace{}%
\AgdaBound{P}\AgdaSymbol{)}\<%
\\
%
\>[6]\AgdaFunction{go-true}\AgdaSpace{}%
\AgdaSymbol{\{}\AgdaBound{s}\AgdaSymbol{\}}\AgdaSpace{}%
\AgdaSymbol{\{}\AgdaBound{ℱ}\AgdaSymbol{\}}\AgdaSpace{}%
\AgdaBound{⊨P}\AgdaSpace{}%
\AgdaBound{p₁}\AgdaSpace{}%
\AgdaBound{p₂}\AgdaSpace{}%
\AgdaBound{⌊ᵗC\,ᵗ⌋}\<%
\\
\>[6][@{}l@{\AgdaIndent{0}}]%
\>[10]\AgdaKeyword{with}\AgdaSpace{}%
\AgdaFunction{⌊ᵗ⌋-split}\AgdaSpace{}%
\AgdaBound{ℱ}\AgdaSpace{}%
\AgdaBound{s}\AgdaSpace{}%
\AgdaBound{C}\AgdaSpace{}%
\AgdaSymbol{(}\AgdaOperator{\AgdaInductiveConstructor{𝔴𝔥𝔦𝔩𝔢}}\AgdaSpace{}%
\AgdaBound{B}\AgdaSpace{}%
\AgdaOperator{\AgdaInductiveConstructor{𝒹ℴ}}\AgdaSpace{}%
\AgdaBound{C}\AgdaSpace{}%
\AgdaOperator{\AgdaInductiveConstructor{;}}\AgdaSymbol{)}\AgdaSpace{}%
\AgdaBound{⌊ᵗC\,ᵗ⌋}\<%
\\
%
\>[6]\AgdaSymbol{...}\AgdaSpace{}%
\AgdaSymbol{|}\AgdaSpace{}%
\AgdaKeyword{record}\AgdaSpace{}%
\AgdaSymbol{\{}\AgdaSpace{}%
\AgdaField{Lᵗ}\AgdaSpace{}%
\AgdaSymbol{=}\AgdaSpace{}%
\AgdaBound{Lᵗ}\AgdaSpace{}%
\AgdaSymbol{;}\AgdaSpace{}%
\AgdaField{ℱ'}\AgdaSpace{}%
\AgdaSymbol{=}\AgdaSpace{}%
\AgdaBound{ℱ'}\AgdaSpace{}%
\AgdaSymbol{;}\AgdaSpace{}%
\AgdaField{Rᵗ}\AgdaSpace{}%
\AgdaSymbol{=}\AgdaSpace{}%
\AgdaBound{Rᵗ}\AgdaSpace{}%
\AgdaSymbol{;}\AgdaSpace{}%
\AgdaField{lt}\AgdaSpace{}%
\AgdaSymbol{=}\AgdaSpace{}%
\AgdaBound{lt}\AgdaSpace{}%
\AgdaSymbol{;}\AgdaSpace{}%
\AgdaField{Δ}\AgdaSpace{}%
\AgdaSymbol{=}\AgdaSpace{}%
\AgdaBound{Δ}\AgdaSpace{}%
\AgdaSymbol{\}}\AgdaSpace{}%
\AgdaSymbol{=}\AgdaSpace{}%
\AgdaFunction{Λ}\<%
\\
\>[6][@{}l@{\AgdaIndent{0}}]%
\>[9]\AgdaKeyword{where}\<%
\\
%
\>[9]\AgdaFunction{⊨B}\AgdaSpace{}%
\AgdaSymbol{:}\AgdaSpace{}%
\AgdaBound{s}\AgdaSpace{}%
\AgdaOperator{\AgdaFunction{⊨}}\AgdaSpace{}%
\AgdaBound{B}\<%
\\
%
\>[9]\AgdaFunction{⊨B}\AgdaSpace{}%
\AgdaKeyword{rewrite}\AgdaSpace{}%
\AgdaBound{p₁}\AgdaSpace{}%
\AgdaSymbol{=}\AgdaSpace{}%
\AgdaSymbol{(}\AgdaInductiveConstructor{just}\AgdaSpace{}%
\AgdaInductiveConstructor{tt}\AgdaSpace{}%
\AgdaOperator{\AgdaInductiveConstructor{,}}\AgdaSpace{}%
\AgdaFunction{subst}\AgdaSpace{}%
\AgdaFunction{T}\AgdaSpace{}%
\AgdaSymbol{(}\AgdaFunction{sym}\AgdaSpace{}%
\AgdaBound{p₂}\AgdaSymbol{)}\AgdaSpace{}%
\AgdaInductiveConstructor{tt}\AgdaSymbol{)}\<%
\\
%
\>[9]\AgdaFunction{⊨P\&B}\AgdaSpace{}%
\AgdaSymbol{:}\AgdaSpace{}%
\AgdaBound{s}\AgdaSpace{}%
\AgdaOperator{\AgdaFunction{⊨}}\AgdaSpace{}%
\AgdaSymbol{(}\AgdaInductiveConstructor{op₂}\AgdaSpace{}%
\AgdaBound{P}\AgdaSpace{}%
\AgdaInductiveConstructor{\&\&ₒ}\AgdaSpace{}%
\AgdaBound{B}\AgdaSymbol{)}\<%
\\
%
\>[9]\AgdaFunction{⊨P\&B}\AgdaSpace{}%
\AgdaSymbol{=}\AgdaSpace{}%
\AgdaFunction{ConjunctionIntro}\AgdaSpace{}%
\AgdaSymbol{\AgdaUnderscore{}}\AgdaSpace{}%
\AgdaSymbol{\AgdaUnderscore{}}\AgdaSpace{}%
\AgdaBound{⊨P}\AgdaSpace{}%
\AgdaFunction{⊨B}\<%
\\
%
\>[9]\AgdaFunction{⊨P'}\AgdaSpace{}%
\AgdaSymbol{:}\AgdaSpace{}%
\AgdaSymbol{(}\AgdaFunction{″}\AgdaSpace{}%
\AgdaBound{Lᵗ}\AgdaSymbol{)}\AgdaSpace{}%
\AgdaOperator{\AgdaFunction{⊨}}\AgdaSpace{}%
\AgdaBound{P}\<%
\\
%
\>[9]\AgdaFunction{⊨P'}\AgdaSpace{}%
\AgdaSymbol{=}\AgdaSpace{}%
\AgdaBound{PBCP}\AgdaSpace{}%
\AgdaBound{s}\AgdaSpace{}%
\AgdaFunction{⊨P\&B}\AgdaSpace{}%
\AgdaSymbol{(}\AgdaBound{ℱ}\AgdaSpace{}%
\AgdaOperator{\AgdaInductiveConstructor{,}}\AgdaSpace{}%
\AgdaBound{Lᵗ}\AgdaSymbol{)}\<%
\\
 \>[9]\AgdaComment{-- Proof of termination of rhs of split with ℱ'}\<%
\\
%
\>[9]\AgdaFunction{Rᵗ+}\AgdaSpace{}%
\AgdaSymbol{:}\AgdaSpace{}%
\AgdaOperator{\AgdaFunction{⌊ᵗ}}\AgdaSpace{}%
\AgdaBound{ℱ'}\AgdaSpace{}%
\AgdaOperator{\AgdaPrimitive{+}}\AgdaSpace{}%
\AgdaSymbol{(}\AgdaField{k}\AgdaSpace{}%
\AgdaBound{lt}\AgdaSymbol{)}\AgdaSpace{}%
\AgdaOperator{\AgdaFunction{⸴}}\AgdaSpace{}%
\AgdaSymbol{(}\AgdaOperator{\AgdaInductiveConstructor{𝔴𝔥𝔦𝔩𝔢}}\AgdaSpace{}%
\AgdaBound{B}\AgdaSpace{}%
\AgdaOperator{\AgdaInductiveConstructor{𝒹ℴ}}\AgdaSpace{}%
\AgdaBound{C}\AgdaSpace{}%
\AgdaOperator{\AgdaInductiveConstructor{;}}\AgdaSymbol{)}\AgdaSpace{}%
\AgdaOperator{\AgdaFunction{⸴}}\AgdaSpace{}%
\AgdaSymbol{(}\AgdaFunction{″}\AgdaSpace{}%
\AgdaBound{Lᵗ}\AgdaSymbol{)}\AgdaSpace{}%
\AgdaOperator{\AgdaFunction{ᵗ⌋}}\<%
\\
%
\>[9]\AgdaFunction{Rᵗ+}\AgdaSpace{}%
\AgdaSymbol{=}\AgdaSpace{}%
\AgdaFunction{addFuel}\AgdaSpace{}%
\AgdaSymbol{\{}\AgdaOperator{\AgdaInductiveConstructor{𝔴𝔥𝔦𝔩𝔢}}\AgdaSpace{}%
\AgdaBound{B}\AgdaSpace{}%
\AgdaOperator{\AgdaInductiveConstructor{𝒹ℴ}}\AgdaSpace{}%
\AgdaBound{C}\AgdaSpace{}%
\AgdaOperator{\AgdaInductiveConstructor{;}}\AgdaSymbol{\}}\AgdaSpace{}%
\AgdaBound{ℱ'}\AgdaSpace{}%
\AgdaSymbol{(}\AgdaField{k}\AgdaSpace{}%
\AgdaBound{lt}\AgdaSymbol{)}\AgdaSpace{}%
\AgdaBound{Rᵗ}\<%
\\
 \>[9]\AgdaComment{-- ℱ' with (ℱ' ≤ ℱ) implies termination with ℱ fuel}\<%
\\
%
\>[9]\AgdaFunction{Rᵗℱ}\AgdaSpace{}%
\AgdaSymbol{:}\AgdaSpace{}%
\AgdaOperator{\AgdaFunction{⌊ᵗ}}\AgdaSpace{}%
\AgdaBound{ℱ}\AgdaSpace{}%
\AgdaOperator{\AgdaFunction{⸴}}\AgdaSpace{}%
\AgdaSymbol{(}\AgdaOperator{\AgdaInductiveConstructor{𝔴𝔥𝔦𝔩𝔢}}\AgdaSpace{}%
\AgdaBound{B}\AgdaSpace{}%
\AgdaOperator{\AgdaInductiveConstructor{𝒹ℴ}}\AgdaSpace{}%
\AgdaBound{C}\AgdaSpace{}%
\AgdaOperator{\AgdaInductiveConstructor{;}}\AgdaSymbol{)}\AgdaSpace{}%
\AgdaOperator{\AgdaFunction{⸴}}\AgdaSpace{}%
\AgdaSymbol{(}\AgdaFunction{″}\AgdaSpace{}%
\AgdaBound{Lᵗ}\AgdaSymbol{)}\AgdaSpace{}%
\AgdaOperator{\AgdaFunction{ᵗ⌋}}\<%
\\
%
\>[9]\AgdaFunction{Rᵗℱ}\AgdaSpace{}%
\AgdaSymbol{=}%
\>[833I]\AgdaKeyword{let}\AgdaSpace{}%
\AgdaBound{𝐶}\AgdaSpace{}%
\AgdaSymbol{=}\AgdaSpace{}%
\AgdaSymbol{(}\AgdaOperator{\AgdaInductiveConstructor{𝔴𝔥𝔦𝔩𝔢}}\AgdaSpace{}%
\AgdaBound{B}\AgdaSpace{}%
\AgdaOperator{\AgdaInductiveConstructor{𝒹ℴ}}\AgdaSpace{}%
\AgdaBound{C}\AgdaSpace{}%
\AgdaOperator{\AgdaInductiveConstructor{;}}\AgdaSymbol{)}\AgdaSpace{}%
\AgdaKeyword{in}\AgdaSpace{}%
\AgdaFunction{subst}\<%
\\
\>[.][@{}l@{}]\<[833I]%
\>[15]\AgdaSymbol{(λ}\AgdaSpace{}%
\AgdaBound{ℱ}\AgdaSpace{}%
\AgdaSymbol{→}\AgdaSpace{}%
\AgdaOperator{\AgdaFunction{⌊ᵗ}}\AgdaSpace{}%
\AgdaBound{ℱ}\AgdaSpace{}%
\AgdaOperator{\AgdaFunction{⸴}}\AgdaSpace{}%
\AgdaBound{𝐶}\AgdaSpace{}%
\AgdaOperator{\AgdaFunction{⸴}}\AgdaSpace{}%
\AgdaSymbol{(}\AgdaFunction{″}\AgdaSpace{}%
\AgdaBound{Lᵗ}\AgdaSymbol{)}\AgdaSpace{}%
\AgdaOperator{\AgdaFunction{ᵗ⌋}}\AgdaSymbol{)}\AgdaSpace{}%
\AgdaSymbol{(}\AgdaField{proof}\AgdaSpace{}%
\AgdaBound{lt}\AgdaSymbol{)}\AgdaSpace{}%
\AgdaFunction{Rᵗ+}\<%
\end{code}
  {\centering \hfill \Huge{\vdots} \hfill }
\end{figure}

\begin{figure}\ContinuedFloat
  \caption{D3-While: Full proof of the while rule; the crucial rule for reasoning with Hoare Logic:}
  \vspace{-0.8cm}
  \begin{center}\!\!\!\small{cont.}\end{center}
  {\centering \hfill \Huge{\vdots} \hfill }
   \centering
   \small
   \begin{code}
\>[9]\AgdaComment{-- This new proof of termination Rᵗℱ has same output}\<%
\\
%
\>[9]\AgdaFunction{isDet}\AgdaSpace{}%
\AgdaSymbol{:}\AgdaSpace{}%
\AgdaFunction{″}\AgdaSpace{}%
\AgdaFunction{Rᵗℱ}\AgdaSpace{}%
\AgdaOperator{\AgdaDatatype{≡}}\AgdaSpace{}%
\AgdaFunction{″}\AgdaSpace{}%
\AgdaBound{Rᵗ}\<%
\\
%
\>[9]\AgdaFunction{isDet}\AgdaSpace{}%
\AgdaSymbol{=}\AgdaSpace{}%
\AgdaFunction{EvalDet}\AgdaSpace{}%
\AgdaSymbol{\{\AgdaUnderscore{}\}}\AgdaSpace{}%
\AgdaSymbol{\{}\AgdaBound{ℱ}\AgdaSymbol{\}}\AgdaSpace{}%
\AgdaSymbol{\{}\AgdaBound{ℱ'}\AgdaSymbol{\}}\AgdaSpace{}%
\AgdaSymbol{(}\AgdaOperator{\AgdaInductiveConstructor{𝔴𝔥𝔦𝔩𝔢}}\AgdaSpace{}%
\AgdaBound{B}\AgdaSpace{}%
\AgdaOperator{\AgdaInductiveConstructor{𝒹ℴ}}\AgdaSpace{}%
\AgdaBound{C}\AgdaSpace{}%
\AgdaOperator{\AgdaInductiveConstructor{;}}\AgdaSymbol{)}\AgdaSpace{}%
\AgdaFunction{Rᵗℱ}\AgdaSpace{}%
\AgdaBound{Rᵗ}\<%
\\
%
\>[9]\AgdaComment{-- and said output is identical to the original output}\<%
\\
%
\>[9]\AgdaFunction{Δ'}\AgdaSpace{}%
\AgdaSymbol{:}\AgdaSpace{}%
\AgdaFunction{″}\AgdaSpace{}%
\AgdaFunction{Rᵗℱ}\AgdaSpace{}%
\AgdaOperator{\AgdaDatatype{≡}}\AgdaSpace{}%
\AgdaFunction{″}\AgdaSpace{}%
\AgdaBound{⌊ᵗC\,ᵗ⌋}\<%
\\
%
\>[9]\AgdaFunction{Δ'}\AgdaSpace{}%
\AgdaKeyword{rewrite}\AgdaSpace{}%
\AgdaFunction{isDet}\AgdaSpace{}%
\AgdaSymbol{=}\AgdaSpace{}%
\AgdaBound{Δ}\<%
\\
%
\>[9]\AgdaComment{-- which we can now use in a recursive call: (suc ℱ) ⇒ ℱ}\<%
\\
%
\>[9]\AgdaFunction{GO}%
\>[13]\AgdaSymbol{:}\AgdaSpace{}%
\AgdaSymbol{(}\AgdaFunction{″}\AgdaSpace{}%
\AgdaFunction{Rᵗℱ}\AgdaSymbol{)}\AgdaSpace{}%
\AgdaOperator{\AgdaFunction{⊨}}\AgdaSpace{}%
\AgdaSymbol{(}\AgdaInductiveConstructor{op₂}\AgdaSpace{}%
\AgdaSymbol{(}\AgdaInductiveConstructor{op₁}\AgdaSpace{}%
\AgdaInductiveConstructor{¬ₒ}\AgdaSpace{}%
\AgdaBound{B}\AgdaSymbol{)}\AgdaSpace{}%
\AgdaInductiveConstructor{\&\&ₒ}\AgdaSpace{}%
\AgdaBound{P}\AgdaSymbol{)}\<%
\\
%
\>[9]\AgdaFunction{GO}%
\>[13]\AgdaSymbol{=}\AgdaSpace{}%
\AgdaFunction{go}\AgdaSpace{}%
\AgdaSymbol{\{}\AgdaFunction{″}\AgdaSpace{}%
\AgdaBound{Lᵗ}\AgdaSymbol{\}}\AgdaSpace{}%
\AgdaBound{ℱ}\AgdaSpace{}%
\AgdaFunction{⊨P'}\AgdaSpace{}%
\AgdaFunction{Rᵗℱ}\<%
\\
\>[0]\<%
\\
%
\>[9]\AgdaComment{-- and finally get the type we need via substitution with Δ'}\<%
\\
%
\>[9]\AgdaFunction{Λ}\AgdaSpace{}%
\AgdaSymbol{:}\AgdaSpace{}%
\AgdaSymbol{(}\AgdaFunction{″}\AgdaSpace{}%
\AgdaBound{⌊ᵗCᵗ⌋}\AgdaSymbol{)}\AgdaSpace{}%
\AgdaOperator{\AgdaFunction{⊨}}\AgdaSpace{}%
\AgdaSymbol{(}\AgdaInductiveConstructor{op₂}\AgdaSpace{}%
\AgdaSymbol{(}\AgdaInductiveConstructor{op₁}\AgdaSpace{}%
\AgdaInductiveConstructor{¬ₒ}\AgdaSpace{}%
\AgdaBound{B}\AgdaSymbol{)}\AgdaSpace{}%
\AgdaInductiveConstructor{\&\&ₒ}\AgdaSpace{}%
\AgdaBound{P}\AgdaSymbol{)}\<%
\\
%
\>[9]\AgdaFunction{Λ}\AgdaSpace{}%
\AgdaSymbol{=}\AgdaSpace{}%
\AgdaFunction{subst}\AgdaSpace{}%
\AgdaSymbol{(λ}\AgdaSpace{}%
\AgdaBound{s}\AgdaSpace{}%
\AgdaSymbol{→}\AgdaSpace{}%
\AgdaBound{s}\AgdaSpace{}%
\AgdaOperator{\AgdaFunction{⊨}}\AgdaSpace{}%
\AgdaSymbol{(}\AgdaInductiveConstructor{op₂}\AgdaSpace{}%
\AgdaSymbol{(}\AgdaInductiveConstructor{op₁}\AgdaSpace{}%
\AgdaInductiveConstructor{¬ₒ}\AgdaSpace{}%
\AgdaBound{B}\AgdaSymbol{)}\AgdaSpace{}%
\AgdaInductiveConstructor{\&\&ₒ}\AgdaSpace{}%
\AgdaBound{P}\AgdaSymbol{))}\AgdaSpace{}%
\AgdaFunction{Δ'}\AgdaSpace{}%
\AgdaFunction{GO}\<%
\\
\>[6]\AgdaComment{-- case where B is false}\<%
\\
%
\>[6]\AgdaFunction{go-false}\AgdaSpace{}%
\AgdaSymbol{:}%
\>[925I]\AgdaSymbol{∀}\AgdaSpace{}%
\AgdaSymbol{\{}\AgdaBound{s}\AgdaSymbol{\}}\AgdaSpace{}%
\AgdaSymbol{\{}\AgdaBound{v}\AgdaSymbol{\}}\AgdaSpace{}%
\AgdaSymbol{→}\AgdaSpace{}%
\AgdaBound{s}\AgdaSpace{}%
\AgdaOperator{\AgdaFunction{⊨}}\AgdaSpace{}%
\AgdaBound{P}\AgdaSpace{}%
\AgdaSymbol{→}\AgdaSpace{}%
\AgdaSymbol{(}\AgdaFunction{evalExp}\AgdaSpace{}%
\AgdaBound{B}\AgdaSpace{}%
\AgdaBound{s}\AgdaSpace{}%
\AgdaOperator{\AgdaDatatype{≡}}\AgdaSpace{}%
\AgdaInductiveConstructor{just}\AgdaSpace{}%
\AgdaBound{v}\AgdaSymbol{)}\<%
\\
\>[.][@{}l@{}]\<[925I]%
\>[17]\AgdaSymbol{→}\AgdaSpace{}%
\AgdaSymbol{(}\AgdaFunction{toTruthValue}\AgdaSpace{}%
\AgdaSymbol{\{}\AgdaInductiveConstructor{just}\AgdaSpace{}%
\AgdaBound{v}\AgdaSymbol{\}}\AgdaSpace{}%
\AgdaSymbol{(}\AgdaInductiveConstructor{just}\AgdaSpace{}%
\AgdaInductiveConstructor{tt}\AgdaSymbol{)}\AgdaSpace{}%
\AgdaOperator{\AgdaDatatype{≡}}\AgdaSpace{}%
\AgdaInductiveConstructor{false}\AgdaSymbol{)}\<%
\\
%
\>[17]\AgdaSymbol{→}\AgdaSpace{}%
\AgdaBound{s}\AgdaSpace{}%
\AgdaOperator{\AgdaFunction{⊨}}\AgdaSpace{}%
\AgdaSymbol{(}\AgdaInductiveConstructor{op₂}\AgdaSpace{}%
\AgdaSymbol{(}\AgdaInductiveConstructor{op₁}\AgdaSpace{}%
\AgdaInductiveConstructor{¬ₒ}\AgdaSpace{}%
\AgdaBound{B}\AgdaSymbol{)}\AgdaSpace{}%
\AgdaInductiveConstructor{\&\&ₒ}\AgdaSpace{}%
\AgdaBound{P}\AgdaSymbol{)}\<%
\\
%
\>[6]\AgdaFunction{go-false}\AgdaSpace{}%
\AgdaSymbol{\{}\AgdaBound{s}\AgdaSymbol{\}}\AgdaSpace{}%
\AgdaSymbol{\{}\AgdaBound{v}\AgdaSymbol{\}}\AgdaSpace{}%
\AgdaBound{⊨P}\AgdaSpace{}%
\AgdaBound{p₁}\AgdaSpace{}%
\AgdaBound{p₂}\AgdaSpace{}%
\AgdaSymbol{=}\AgdaSpace{}%
\AgdaFunction{ConjunctionIntro}\AgdaSpace{}%
\AgdaSymbol{\AgdaUnderscore{}}\AgdaSpace{}%
\AgdaSymbol{\AgdaUnderscore{}}\AgdaSpace{}%
\AgdaFunction{⊨¬B}\AgdaSpace{}%
\AgdaBound{⊨P}\<%
\\
\>[6][@{}l@{\AgdaIndent{0}}]%
\>[8]\AgdaKeyword{where}\<%
\\
%
\>[8]\AgdaFunction{⊭B}\AgdaSpace{}%
\AgdaSymbol{:}\AgdaSpace{}%
\AgdaFunction{⊬}\AgdaSpace{}%
\AgdaSymbol{(}\AgdaInductiveConstructor{just}\AgdaSpace{}%
\AgdaBound{v}\AgdaSymbol{)}\<%
\\
%
\>[8]\AgdaFunction{⊭B}\AgdaSpace{}%
\AgdaKeyword{rewrite}\AgdaSpace{}%
\AgdaBound{p₁}\AgdaSpace{}%
\AgdaSymbol{=}\AgdaSpace{}%
\AgdaSymbol{(}\AgdaInductiveConstructor{just}\AgdaSpace{}%
\AgdaInductiveConstructor{tt}\AgdaSymbol{)}\AgdaSpace{}%
\AgdaOperator{\AgdaInductiveConstructor{,}}\AgdaSpace{}%
\AgdaFunction{subst}\AgdaSpace{}%
\AgdaSymbol{(}\AgdaFunction{T}\AgdaSpace{}%
\AgdaOperator{\AgdaFunction{∘}}\AgdaSpace{}%
\AgdaFunction{not}\AgdaSymbol{)}\AgdaSpace{}%
\AgdaSymbol{(}\AgdaFunction{sym}\AgdaSpace{}%
\AgdaBound{p₂}\AgdaSymbol{)}\AgdaSpace{}%
\AgdaInductiveConstructor{tt}\<%
\\
%
\>[8]\AgdaFunction{⊨¬B}\AgdaSpace{}%
\AgdaSymbol{:}\AgdaSpace{}%
\AgdaBound{s}\AgdaSpace{}%
\AgdaOperator{\AgdaFunction{⊨}}\AgdaSpace{}%
\AgdaSymbol{(}\AgdaInductiveConstructor{op₁}\AgdaSpace{}%
\AgdaInductiveConstructor{¬ₒ}\AgdaSpace{}%
\AgdaBound{B}\AgdaSymbol{)}\<%
\\
%
\>[8]\AgdaFunction{⊨¬B}\AgdaSpace{}%
\AgdaKeyword{rewrite}\AgdaSpace{}%
\AgdaBound{p₁}\AgdaSpace{}%
\AgdaSymbol{=}\AgdaSpace{}%
\AgdaSymbol{(}\AgdaFunction{NegationIntro}\AgdaSpace{}%
\AgdaSymbol{(}\AgdaInductiveConstructor{just}\AgdaSpace{}%
\AgdaBound{v}\AgdaSymbol{)}\AgdaSpace{}%
\AgdaSymbol{(}\AgdaFunction{⊭B}\AgdaSymbol{))}\<%
\\
%
\>[6]\AgdaComment{---------------------------------------------------------------}\<%
\\
%
\>[6]\AgdaFunction{go}%
\>[995I]\AgdaSymbol{\{}\AgdaBound{s}\AgdaSymbol{\}}\AgdaSpace{}%
\AgdaSymbol{(}\AgdaInductiveConstructor{suc}\AgdaSpace{}%
\AgdaBound{ℱ}\AgdaSymbol{)}\AgdaSpace{}%
\AgdaBound{⊨P}\AgdaSpace{}%
\AgdaBound{⌊ᵗCᵗ⌋}\AgdaSpace{}%
\AgdaKeyword{with}\<%
\\
\>[995I][@{}l@{\AgdaIndent{0}}]%
\>[10]\AgdaFunction{evalExp}\AgdaSpace{}%
\AgdaBound{B}\AgdaSpace{}%
\AgdaBound{s}%
\>[23]\AgdaSymbol{|}\AgdaSpace{}%
\AgdaFunction{inspect}\AgdaSpace{}%
\AgdaSymbol{(}\AgdaFunction{evalExp}\AgdaSpace{}%
\AgdaBound{B}\AgdaSymbol{)}\AgdaSpace{}%
\AgdaBound{s}\<%
\\
%
\>[6]\AgdaSymbol{...}%
\>[1007I]\AgdaSymbol{|}\AgdaSpace{}%
\AgdaBound{f}\AgdaSymbol{@(}\AgdaInductiveConstructor{just}\AgdaSpace{}%
\AgdaBound{v}\AgdaSymbol{)}\AgdaSpace{}%
\AgdaSymbol{|}\AgdaSpace{}%
\AgdaOperator{\AgdaInductiveConstructor{[}}\AgdaSpace{}%
\AgdaBound{p₁}\AgdaSpace{}%
\AgdaOperator{\AgdaInductiveConstructor{]}}\AgdaSpace{}%
\AgdaKeyword{with}\<%
\\
\>[.][@{}l@{}]\<[1007I]%
\>[10]\AgdaFunction{toTruthValue}\AgdaSpace{}%
\AgdaSymbol{\{}\AgdaBound{f}\AgdaSymbol{\}}\AgdaSpace{}%
\AgdaSymbol{(}\AgdaInductiveConstructor{any}\AgdaSpace{}%
\AgdaInductiveConstructor{tt}\AgdaSymbol{)}\AgdaSpace{}%
\AgdaSymbol{|}\AgdaSpace{}%
\AgdaFunction{inspect}\AgdaSpace{}%
\AgdaSymbol{(}\AgdaFunction{toTruthValue}\AgdaSpace{}%
\AgdaSymbol{\{}\AgdaBound{f}\AgdaSymbol{\})}\AgdaSpace{}%
\AgdaSymbol{(}\AgdaInductiveConstructor{any}\AgdaSpace{}%
\AgdaInductiveConstructor{tt}\AgdaSymbol{)}\<%
\\
%
\>[6]\AgdaSymbol{...}\AgdaSpace{}%
\AgdaSymbol{|}\AgdaSpace{}%
\AgdaInductiveConstructor{true}%
\>[18]\AgdaSymbol{|}\AgdaSpace{}%
\AgdaOperator{\AgdaInductiveConstructor{[}}\AgdaSpace{}%
\AgdaBound{p₂}\AgdaSpace{}%
\AgdaOperator{\AgdaInductiveConstructor{]}}\AgdaSpace{}%
\AgdaSymbol{=}\AgdaSpace{}%
\AgdaFunction{go-true}\AgdaSpace{}%
\AgdaSymbol{\{}\AgdaBound{s}\AgdaSymbol{\}}\AgdaSpace{}%
\AgdaSymbol{\{}\AgdaBound{ℱ}\AgdaSymbol{\}}\AgdaSpace{}%
\AgdaBound{⊨P}\AgdaSpace{}%
\AgdaBound{p₁}\AgdaSpace{}%
\AgdaBound{p₂}\AgdaSpace{}%
\AgdaBound{⌊ᵗCᵗ⌋}\<%
\\
%
\>[6]\AgdaSymbol{...}\AgdaSpace{}%
\AgdaSymbol{|}\AgdaSpace{}%
\AgdaInductiveConstructor{false}\AgdaSpace{}%
\AgdaSymbol{|}\AgdaSpace{}%
\AgdaOperator{\AgdaInductiveConstructor{[}}\AgdaSpace{}%
\AgdaBound{p₂}\AgdaSpace{}%
\AgdaOperator{\AgdaInductiveConstructor{]}}\AgdaSpace{}%
\AgdaKeyword{rewrite}\AgdaSpace{}%
\AgdaFunction{Is-just-just}\AgdaSpace{}%
\AgdaBound{⌊ᵗCᵗ⌋}\AgdaSpace{}%
\AgdaSymbol{=}\AgdaSpace{}%
\AgdaFunction{go-false}\AgdaSpace{}%
\AgdaBound{⊨P}\AgdaSpace{}%
\AgdaBound{p₁}\AgdaSpace{}%
\AgdaBound{p₂}\<%
\\
%
\>[6]\AgdaComment{---------------------------------------------------------------}\<%
\end{code}
\end{figure}


\section{Implementation}

\subsection{Constructive Termination}


\lipsum[66-69]


\subsection{Small Step Evaluation with Fuel}


\lipsum[66-69]


\subsection{Termination Splitting}


\lipsum[66-75]


\subsection{Axiom \& Rules in Agda}


\begin{figure}
  \caption{ssEvalwithFuel: The small-step evaluation function}
  \centering
  \begin{code}
  \>[2]\AgdaComment{-----------------------------------------------------------------}\<%
\\
%
\>[2]\AgdaFunction{ssEvalwithFuel}\AgdaSpace{}%
\AgdaSymbol{:}%
\>[20]\AgdaDatatype{ℕ}\AgdaSpace{}%
\AgdaSymbol{→}\AgdaSpace{}%
\AgdaFunction{C}\AgdaSpace{}%
\AgdaSymbol{→}\AgdaSpace{}%
\AgdaField{S}\AgdaSpace{}%
\AgdaSymbol{→}\AgdaSpace{}%
\AgdaDatatype{Maybe}\AgdaSpace{}%
\AgdaField{S}\<%
\\
%
\>[2]\AgdaComment{-----------------------------------------------------------------  }\<%
\\
%
\>[2]\AgdaComment{-\,\!- Skip always terminates successfully even with zero fuel}\<%
\\
%
\>[2]\AgdaFunction{ssEvalwithFuel}\AgdaSpace{}%
\AgdaInductiveConstructor{zero}\AgdaSpace{}%
\AgdaSymbol{(}\AgdaInductiveConstructor{𝑠𝑘𝑖𝑝}\AgdaSpace{}%
\AgdaOperator{\AgdaInductiveConstructor{;}}\AgdaSymbol{)}\AgdaSpace{}%
\AgdaBound{s}\AgdaSpace{}%
\AgdaSymbol{=}\AgdaSpace{}%
\AgdaInductiveConstructor{just}\AgdaSpace{}%
\AgdaBound{s}\<%
\\
%
\>[2]\AgdaFunction{ssEvalwithFuel}\AgdaSpace{}%
\AgdaSymbol{(}\AgdaInductiveConstructor{suc}\AgdaSpace{}%
\AgdaBound{n}\AgdaSymbol{)}\AgdaSpace{}%
\AgdaSymbol{(}\AgdaSpace{}%
\AgdaInductiveConstructor{𝑠𝑘𝑖𝑝}\AgdaSpace{}%
\AgdaOperator{\AgdaInductiveConstructor{;}}\AgdaSymbol{)}\AgdaSpace{}%
\AgdaBound{s}\AgdaSpace{}%
\AgdaSymbol{=}\AgdaSpace{}%
\AgdaInductiveConstructor{just}\AgdaSpace{}%
\AgdaBound{s}\<%
\\
%
\>[2]\AgdaComment{-----------------------------------------------------------------}\<%
\\
%
\>[2]\AgdaComment{-\,\!- Out of fuel}\<%
\\
%
\>[2]\AgdaComment{-\,\!- Need to explicitly give all cases here so Agda can see}\<%
\\
%
\>[2]\AgdaComment{-\,\!- `eval zero C = nothing' is definitionally true when C≠skip}\<%
\\
%
\>[2]\AgdaFunction{ssEvalwithFuel}\AgdaSpace{}%
\AgdaInductiveConstructor{zero}\AgdaSpace{}%
\AgdaSymbol{(}\AgdaSpace{}%
\AgdaOperator{\AgdaInductiveConstructor{𝔴𝔥𝔦𝔩𝔢}}\AgdaSpace{}%
\AgdaSymbol{\AgdaUnderscore{}}\AgdaSpace{}%
\AgdaOperator{\AgdaInductiveConstructor{𝒹ℴ}}\AgdaSpace{}%
\AgdaSymbol{\AgdaUnderscore{}}\AgdaSpace{}%
\AgdaOperator{\AgdaInductiveConstructor{;}}\AgdaSymbol{)}\AgdaSpace{}%
\AgdaSymbol{\AgdaUnderscore{}}\AgdaSpace{}%
\AgdaSymbol{=}\AgdaSpace{}%
\AgdaInductiveConstructor{nothing}\<%
\\
%
\>[2]\AgdaFunction{ssEvalwithFuel}\AgdaSpace{}%
\AgdaInductiveConstructor{zero}\AgdaSpace{}%
\AgdaSymbol{(}\AgdaSpace{}%
\AgdaOperator{\AgdaInductiveConstructor{𝔦𝔣}}\AgdaSpace{}%
\AgdaSymbol{\AgdaUnderscore{}}\AgdaSpace{}%
\AgdaOperator{\AgdaInductiveConstructor{𝔱𝔥𝔢𝔫}}\AgdaSpace{}%
\AgdaSymbol{\AgdaUnderscore{}}\AgdaSpace{}%
\AgdaOperator{\AgdaInductiveConstructor{𝔢𝔩𝔰𝔢}}\AgdaSpace{}%
\AgdaSymbol{\AgdaUnderscore{}}\AgdaSpace{}%
\AgdaOperator{\AgdaInductiveConstructor{;}}\AgdaSymbol{)}\AgdaSpace{}%
\AgdaSymbol{\AgdaUnderscore{}}\AgdaSpace{}%
\AgdaSymbol{=}\AgdaSpace{}%
\AgdaInductiveConstructor{nothing}\<%
\\
%
\>[2]\AgdaFunction{ssEvalwithFuel}\AgdaSpace{}%
\AgdaInductiveConstructor{zero}\AgdaSpace{}%
\AgdaSymbol{(}\AgdaSpace{}%
\AgdaSymbol{\AgdaUnderscore{}}\AgdaSpace{}%
\AgdaOperator{\AgdaInductiveConstructor{:=}}\AgdaSpace{}%
\AgdaSymbol{\AgdaUnderscore{}}\AgdaSpace{}%
\AgdaOperator{\AgdaInductiveConstructor{;}}\AgdaSpace{}%
\AgdaSymbol{)}\AgdaSpace{}%
\AgdaSymbol{\AgdaUnderscore{}}\AgdaSpace{}%
\AgdaSymbol{=}\AgdaSpace{}%
\AgdaInductiveConstructor{nothing}\<%
\\
%
\>[2]\AgdaFunction{ssEvalwithFuel}\AgdaSpace{}%
\AgdaInductiveConstructor{zero}\AgdaSpace{}%
\AgdaSymbol{((}\AgdaOperator{\AgdaInductiveConstructor{𝔴𝔥𝔦𝔩𝔢}}\AgdaSpace{}%
\AgdaSymbol{\AgdaUnderscore{}}\AgdaSpace{}%
\AgdaOperator{\AgdaInductiveConstructor{𝒹ℴ}}\AgdaSpace{}%
\AgdaSymbol{\AgdaUnderscore{})}\AgdaSpace{}%
\AgdaOperator{\AgdaInductiveConstructor{;}}\AgdaSpace{}%
\AgdaSymbol{\AgdaUnderscore{})}\AgdaSpace{}%
\AgdaSymbol{\AgdaUnderscore{}}\AgdaSpace{}%
\AgdaSymbol{=}\AgdaSpace{}%
\AgdaInductiveConstructor{nothing}\<%
\\
%
\>[2]\AgdaFunction{ssEvalwithFuel}\AgdaSpace{}%
\AgdaInductiveConstructor{zero}\AgdaSpace{}%
\AgdaSymbol{((}\AgdaOperator{\AgdaInductiveConstructor{𝔦𝔣}}\AgdaSpace{}%
\AgdaSymbol{\AgdaUnderscore{}}\AgdaSpace{}%
\AgdaOperator{\AgdaInductiveConstructor{𝔱𝔥𝔢𝔫}}\AgdaSpace{}%
\AgdaSymbol{\AgdaUnderscore{}}\AgdaSpace{}%
\AgdaOperator{\AgdaInductiveConstructor{𝔢𝔩𝔰𝔢}}\AgdaSpace{}%
\AgdaSymbol{\AgdaUnderscore{})}\AgdaSpace{}%
\AgdaOperator{\AgdaInductiveConstructor{;}}\AgdaSpace{}%
\AgdaSymbol{\AgdaUnderscore{})}\AgdaSpace{}%
\AgdaSymbol{\AgdaUnderscore{}}\AgdaSpace{}%
\AgdaSymbol{=}\AgdaSpace{}%
\AgdaInductiveConstructor{nothing}\<%
\\
%
\>[2]\AgdaFunction{ssEvalwithFuel}\AgdaSpace{}%
\AgdaInductiveConstructor{zero}\AgdaSpace{}%
\AgdaSymbol{((\AgdaUnderscore{}}\AgdaSpace{}%
\AgdaOperator{\AgdaInductiveConstructor{:=}}\AgdaSpace{}%
\AgdaSymbol{\AgdaUnderscore{})}\AgdaSpace{}%
\AgdaOperator{\AgdaInductiveConstructor{;}}\AgdaSpace{}%
\AgdaSymbol{\AgdaUnderscore{})}\AgdaSpace{}%
\AgdaSymbol{\AgdaUnderscore{}}\AgdaSpace{}%
\AgdaSymbol{=}\AgdaSpace{}%
\AgdaInductiveConstructor{nothing}\<%
\\
%
\>[2]\AgdaFunction{ssEvalwithFuel}\AgdaSpace{}%
\AgdaInductiveConstructor{zero}\AgdaSpace{}%
\AgdaSymbol{(}\AgdaSpace{}%
\AgdaInductiveConstructor{𝑠𝑘𝑖𝑝}\AgdaSpace{}%
\AgdaOperator{\AgdaInductiveConstructor{;}}\AgdaSpace{}%
\AgdaBound{b}\AgdaSpace{}%
\AgdaSymbol{)}\AgdaSpace{}%
\AgdaBound{s}\AgdaSpace{}%
\AgdaSymbol{=}\AgdaSpace{}%
\AgdaFunction{ssEvalwithFuel}\AgdaSpace{}%
\AgdaInductiveConstructor{zero}\AgdaSpace{}%
\AgdaBound{b}\AgdaSpace{}%
\AgdaBound{s}\<%
\\
%
\>[2]\AgdaComment{-----------------------------------------------------------------}\<%
\\
%
\>[2]\AgdaComment{-\,\!- SINGLE WHILE }\<%
\\
%
\>[2]\AgdaFunction{ssEvalwithFuel}\AgdaSpace{}%
\AgdaSymbol{(}\AgdaInductiveConstructor{suc}\AgdaSpace{}%
\AgdaBound{n}\AgdaSymbol{)}\AgdaSpace{}%
\AgdaSymbol{(}\AgdaSpace{}%
\AgdaOperator{\AgdaInductiveConstructor{𝔴𝔥𝔦𝔩𝔢}}\AgdaSpace{}%
\AgdaBound{exp}\AgdaSpace{}%
\AgdaOperator{\AgdaInductiveConstructor{𝒹ℴ}}\AgdaSpace{}%
\AgdaBound{c}\AgdaSpace{}%
\AgdaOperator{\AgdaInductiveConstructor{;}}\AgdaSymbol{)}\AgdaSpace{}%
\AgdaBound{s}\AgdaSpace{}%
\AgdaKeyword{with}\AgdaSpace{}%
\AgdaFunction{evalExp}\AgdaSpace{}%
\AgdaBound{exp}\AgdaSpace{}%
\AgdaBound{s}\<%
\\
%
\>[2]\AgdaSymbol{...}\AgdaSpace{}%
\AgdaSymbol{|}\AgdaSpace{}%
\AgdaInductiveConstructor{nothing}\AgdaSpace{}%
\AgdaSymbol{=}\AgdaSpace{}%
\AgdaInductiveConstructor{nothing}\AgdaSpace{}%
\AgdaComment{-\,\!- Computation failed i.e. div by 0}\<%
\\
%
\>[2]\AgdaSymbol{...}\AgdaSpace{}%
\AgdaSymbol{|}\AgdaSpace{}%
\AgdaBound{f\;}\AgdaSymbol{@\;(}\AgdaInductiveConstructor{just}\AgdaSpace{}%
\AgdaSymbol{\AgdaUnderscore{})}\AgdaSpace{}%
\AgdaKeyword{with}\AgdaSpace{}%
\AgdaFunction{toTruthValue}\AgdaSpace{}%
\AgdaSymbol{\{}\AgdaBound{\,f\,}\AgdaSymbol{\}}\AgdaSpace{}%
\AgdaSymbol{(}\AgdaInductiveConstructor{Any.just}\AgdaSpace{}%
\AgdaInductiveConstructor{tt}\AgdaSymbol{)}\<%
\\
%
\>[2]\AgdaSymbol{...}\AgdaSpace{}%
\AgdaSymbol{|}\AgdaSpace{}%
\AgdaInductiveConstructor{true}%
\>[14]\AgdaSymbol{=}\AgdaSpace{}%
\AgdaFunction{ssEvalwithFuel}\AgdaSpace{}%
\AgdaBound{n}\AgdaSpace{}%
\AgdaSymbol{(}\AgdaSpace{}%
\AgdaBound{c}\AgdaSpace{}%
\AgdaOperator{\AgdaFunction{𝔱𝔥𝔢𝔫}}\AgdaSpace{}%
\AgdaOperator{\AgdaInductiveConstructor{𝔴𝔥𝔦𝔩𝔢}}\AgdaSpace{}%
\AgdaBound{exp}\AgdaSpace{}%
\AgdaOperator{\AgdaInductiveConstructor{𝒹ℴ}}\AgdaSpace{}%
\AgdaBound{c}\AgdaSpace{}%
\AgdaOperator{\AgdaInductiveConstructor{;}}\AgdaSymbol{)}\AgdaSpace{}%
\AgdaBound{s}\<%
\\
%
\>[2]\AgdaSymbol{...}\AgdaSpace{}%
\AgdaSymbol{|}\AgdaSpace{}%
\AgdaInductiveConstructor{false}\AgdaSpace{}%
\AgdaSymbol{=}\AgdaSpace{}%
\AgdaInductiveConstructor{just}\AgdaSpace{}%
\AgdaBound{s}\<%
\\
%
\>[2]\AgdaComment{-----------------------------------------------------------------}\<%
\\
%
\>[2]\AgdaComment{-\,\!- SINGLE IF THEN ELSE}\<%
\\
%
\>[2]\AgdaFunction{ssEvalwithFuel}\AgdaSpace{}%
\AgdaSymbol{(}\AgdaInductiveConstructor{suc}\AgdaSpace{}%
\AgdaBound{n}\AgdaSymbol{)}\AgdaSpace{}%
\AgdaSymbol{(}\AgdaSpace{}%
\AgdaOperator{\AgdaInductiveConstructor{𝔦𝔣}}\AgdaSpace{}%
\AgdaBound{exp}\AgdaSpace{}%
\AgdaOperator{\AgdaInductiveConstructor{𝔱𝔥𝔢𝔫}}\AgdaSpace{}%
\AgdaBound{c₁}\AgdaSpace{}%
\AgdaOperator{\AgdaInductiveConstructor{𝔢𝔩𝔰𝔢}}\AgdaSpace{}%
\AgdaBound{c₂}%
\>[51]\AgdaOperator{\AgdaInductiveConstructor{;}}\AgdaSymbol{)}\AgdaSpace{}%
\AgdaBound{s}\<%
\\
\>[2][@{}l@{\AgdaIndent{0}}]%
\>[6]\AgdaKeyword{with}\AgdaSpace{}%
\AgdaFunction{evalExp}\AgdaSpace{}%
\AgdaBound{exp}\AgdaSpace{}%
\AgdaBound{s}\<%
\\
%
\>[2]\AgdaSymbol{...}\AgdaSpace{}%
\AgdaSymbol{|}\AgdaSpace{}%
\AgdaInductiveConstructor{nothing}\AgdaSpace{}%
\AgdaSymbol{=}\AgdaSpace{}%
\AgdaInductiveConstructor{nothing}\AgdaSpace{}%
\AgdaComment{-\,\!- Computation failed i.e. div by 0}\<%
\\
%
\>[2]\AgdaSymbol{...}\AgdaSpace{}%
\AgdaSymbol{|}\AgdaSpace{}%
\AgdaBound{f\;}\AgdaSymbol{@\;(}\AgdaInductiveConstructor{just}\AgdaSpace{}%
\AgdaSymbol{\AgdaUnderscore{})}\AgdaSpace{}%
\AgdaKeyword{with}\AgdaSpace{}%
\AgdaFunction{toTruthValue}\AgdaSpace{}%
\AgdaSymbol{\{}\AgdaBound{\,f\,}\AgdaSymbol{\}}\AgdaSpace{}%
\AgdaSymbol{(}\AgdaInductiveConstructor{Any.just}\AgdaSpace{}%
\AgdaInductiveConstructor{tt}\AgdaSymbol{)}\<%
\\
%
\>[2]\AgdaSymbol{...}\AgdaSpace{}%
\AgdaSymbol{|}\AgdaSpace{}%
\AgdaInductiveConstructor{true}\AgdaSpace{}%
\AgdaSymbol{=}\AgdaSpace{}%
\AgdaFunction{ssEvalwithFuel}\AgdaSpace{}%
\AgdaBound{n}\AgdaSpace{}%
\AgdaBound{c₁}\AgdaSpace{}%
\AgdaBound{s}\<%
\\
%
\>[2]\AgdaSymbol{...}\AgdaSpace{}%
\AgdaSymbol{|}\AgdaSpace{}%
\AgdaInductiveConstructor{false}\AgdaSpace{}%
\AgdaSymbol{=}\AgdaSpace{}%
\AgdaFunction{ssEvalwithFuel}\AgdaSpace{}%
\AgdaBound{n}\AgdaSpace{}%
\AgdaBound{c₂}\AgdaSpace{}%
\AgdaBound{s}\<%
\\
%
\>[2]\AgdaComment{-----------------------------------------------------------------}\<%
\end{code}

  {\centering \hfill \Huge{\vdots} \hfill }
\end{figure}

\begin{figure}\ContinuedFloat
  \caption{ssEvalwithFuel cont.}
  \centering
  { \hfill \Huge{\vdots} \hfill }
  \begin{code}
\>[2]\AgdaComment{----------------------------------------------------------------------------------------------------------------------}\<%
\\
%
\>[2]\AgdaComment{-\,\!- SINGLE ASSI}\<%
\\
%
\>[2]\AgdaFunction{ssEvalwithFuel}\AgdaSpace{}%
\AgdaSymbol{(}\AgdaInductiveConstructor{suc}\AgdaSpace{}%
\AgdaBound{n}\AgdaSymbol{)}\AgdaSpace{}%
\AgdaSymbol{(}\AgdaSpace{}%
\AgdaBound{id}\AgdaSpace{}%
\AgdaOperator{\AgdaInductiveConstructor{:=}}\AgdaSpace{}%
\AgdaBound{exp}\AgdaSpace{}%
\AgdaOperator{\AgdaInductiveConstructor{;}}\AgdaSymbol{)}\AgdaSpace{}%
\AgdaBound{s}\AgdaSpace{}%
\AgdaSymbol{=}\<%
\\
\>[2][@{}l@{\AgdaIndent{0}}]%
\>[4]\AgdaFunction{map}\AgdaSpace{}%
\AgdaSymbol{(λ}\AgdaSpace{}%
\AgdaBound{v}\AgdaSpace{}%
\AgdaSymbol{→}\AgdaSpace{}%
\AgdaField{updateState}\AgdaSpace{}%
\AgdaBound{id}\AgdaSpace{}%
\AgdaBound{v}\AgdaSpace{}%
\AgdaBound{s}\AgdaSymbol{)}\AgdaSpace{}%
\AgdaSymbol{(}\AgdaFunction{evalExp}\AgdaSpace{}%
\AgdaBound{exp}\AgdaSpace{}%
\AgdaBound{s}\AgdaSymbol{)}\<%
\\
%
\>[2]\AgdaComment{----------------------------------------------------------------------------------------------------------------------}\<%
\\
%
\>[2]\AgdaComment{-\,\!- SKIP ; THEN C }\<%
\\
%
\>[2]\AgdaFunction{ssEvalwithFuel}\AgdaSpace{}%
\AgdaSymbol{(}\AgdaInductiveConstructor{suc}\AgdaSpace{}%
\AgdaBound{n}\AgdaSymbol{)}\AgdaSpace{}%
\AgdaSymbol{(}\AgdaInductiveConstructor{𝑠𝑘𝑖𝑝}\AgdaSpace{}%
\AgdaOperator{\AgdaInductiveConstructor{;}}\AgdaSpace{}%
\AgdaBound{c}\AgdaSymbol{)}\AgdaSpace{}%
\AgdaBound{s}\AgdaSpace{}%
\AgdaSymbol{=}\AgdaSpace{}%
\AgdaFunction{ssEvalwithFuel}\AgdaSpace{}%
\AgdaSymbol{(}\AgdaInductiveConstructor{suc}\AgdaSpace{}%
\AgdaBound{n}\AgdaSymbol{)}\AgdaSpace{}%
\AgdaBound{c}\AgdaSpace{}%
\AgdaBound{s}\<%
\\
%
\>[2]\AgdaComment{----------------------------------------------------------------------------------------------------------------------}\<%
\\
%
\>[2]\AgdaComment{-\,\!- WHILE ; THEN C₂}\<%
\\
%
\>[2]\AgdaFunction{ssEvalwithFuel}\AgdaSpace{}%
\AgdaSymbol{(}\AgdaInductiveConstructor{suc}\AgdaSpace{}%
\AgdaBound{n}\AgdaSymbol{)}\AgdaSpace{}%
\AgdaSymbol{((}\AgdaOperator{\AgdaInductiveConstructor{𝔴𝔥𝔦𝔩𝔢}}\AgdaSpace{}%
\AgdaBound{exp}\AgdaSpace{}%
\AgdaOperator{\AgdaInductiveConstructor{𝒹ℴ}}\AgdaSpace{}%
\AgdaBound{c₁}\AgdaSymbol{)}\AgdaSpace{}%
\AgdaOperator{\AgdaInductiveConstructor{;}}\AgdaSpace{}%
\AgdaBound{c₂}\AgdaSymbol{)}\AgdaSpace{}%
\AgdaBound{s}\<%
\\
\>[2][@{}l@{\AgdaIndent{0}}]%
\>[6]\AgdaKeyword{with}\AgdaSpace{}%
\AgdaFunction{evalExp}\AgdaSpace{}%
\AgdaBound{exp}\AgdaSpace{}%
\AgdaBound{s}\<%
\\
%
\>[2]\AgdaSymbol{...}\AgdaSpace{}%
\AgdaSymbol{|}\AgdaSpace{}%
\AgdaInductiveConstructor{nothing}\AgdaSpace{}%
\AgdaSymbol{=}\AgdaSpace{}%
\AgdaInductiveConstructor{nothing}\AgdaSpace{}%
\AgdaComment{-\,\!- Computation failed i.e. div by 0}\<%
\\
%
\>[2]\AgdaSymbol{...}\AgdaSpace{}%
\AgdaSymbol{|}\AgdaSpace{}%
\AgdaBound{f\;}\AgdaSymbol{@\;(}\AgdaInductiveConstructor{just}\AgdaSpace{}%
\AgdaSymbol{\AgdaUnderscore{})}\AgdaSpace{}%
\AgdaKeyword{with}\AgdaSpace{}%
\AgdaFunction{toTruthValue}\AgdaSpace{}%
\AgdaSymbol{\{}\AgdaBound{\,f\,}\AgdaSymbol{\}}\AgdaSpace{}%
\AgdaSymbol{(}\AgdaInductiveConstructor{Any.just}\AgdaSpace{}%
\AgdaInductiveConstructor{tt}\AgdaSymbol{)}\<%
\\
%
\>[2]\AgdaSymbol{...}\AgdaSpace{}%
\AgdaSymbol{|}\AgdaSpace{}%
\AgdaInductiveConstructor{true}\AgdaSpace{}%
\AgdaSymbol{=}\AgdaSpace{}%
\AgdaFunction{ssEvalwithFuel}\AgdaSpace{}%
\AgdaBound{n}\AgdaSpace{}%
\AgdaSymbol{(}\AgdaBound{c₁}\AgdaSpace{}%
\AgdaOperator{\AgdaFunction{𝔱𝔥𝔢𝔫}}\AgdaSpace{}%
\AgdaSymbol{((}\AgdaOperator{\AgdaInductiveConstructor{𝔴𝔥𝔦𝔩𝔢}}\AgdaSpace{}%
\AgdaBound{exp}\AgdaSpace{}%
\AgdaOperator{\AgdaInductiveConstructor{𝒹ℴ}}\AgdaSpace{}%
\AgdaBound{c₁}\AgdaSymbol{)}\AgdaSpace{}%
\AgdaOperator{\AgdaInductiveConstructor{;}}\AgdaSpace{}%
\AgdaBound{c₂}\AgdaSymbol{))}\AgdaSpace{}%
\AgdaBound{s}\<%
\\
%
\>[2]\AgdaSymbol{...}\AgdaSpace{}%
\AgdaSymbol{|}\AgdaSpace{}%
\AgdaInductiveConstructor{false}\AgdaSpace{}%
\AgdaSymbol{=}\AgdaSpace{}%
\AgdaFunction{ssEvalwithFuel}\AgdaSpace{}%
\AgdaBound{n}\AgdaSpace{}%
\AgdaBound{c₂}\AgdaSpace{}%
\AgdaBound{s}\<%
\\
%
\>[2]\AgdaComment{----------------------------------------------------------------------------------------------------------------------}\<%
\\
%
\>[2]\AgdaComment{--\,\!- IF THEN ELSE ; THEN C₂}\<%
\\
%
\>[2]\AgdaFunction{ssEvalwithFuel}\AgdaSpace{}%
\AgdaSymbol{(}\AgdaInductiveConstructor{suc}\AgdaSpace{}%
\AgdaBound{n}\AgdaSymbol{)}\AgdaSpace{}%
\AgdaSymbol{((}\AgdaOperator{\AgdaInductiveConstructor{𝔦𝔣}}\AgdaSpace{}%
\AgdaBound{exp}\AgdaSpace{}%
\AgdaOperator{\AgdaInductiveConstructor{𝔱𝔥𝔢𝔫}}\AgdaSpace{}%
\AgdaBound{c₁}\AgdaSpace{}%
\AgdaOperator{\AgdaInductiveConstructor{𝔢𝔩𝔰𝔢}}\AgdaSpace{}%
\AgdaBound{c₂}\AgdaSymbol{)}\AgdaSpace{}%
\AgdaOperator{\AgdaInductiveConstructor{;}}\AgdaSpace{}%
\AgdaBound{c₃}\AgdaSymbol{)}\AgdaSpace{}%
\AgdaBound{s}\<%
\\
\>[2][@{}l@{\AgdaIndent{0}}]%
\>[6]\AgdaKeyword{with}\AgdaSpace{}%
\AgdaFunction{evalExp}\AgdaSpace{}%
\AgdaBound{exp}\AgdaSpace{}%
\AgdaBound{s}\<%
\\
%
\>[2]\AgdaSymbol{...}\AgdaSpace{}%
\AgdaSymbol{|}\AgdaSpace{}%
\AgdaInductiveConstructor{nothing}\AgdaSpace{}%
\AgdaSymbol{=}\AgdaSpace{}%
\AgdaInductiveConstructor{nothing}\AgdaSpace{}%
\AgdaComment{-\,\!- Computation failed i.e. div by 0}\<%
\\
%
\>[2]\AgdaSymbol{...}\AgdaSpace{}%
\AgdaSymbol{|}\AgdaSpace{}%
\AgdaBound{f\;}\AgdaSymbol{@\;(}\AgdaInductiveConstructor{just}\AgdaSpace{}%
\AgdaSymbol{\AgdaUnderscore{})}\AgdaSpace{}%
\AgdaKeyword{with}\AgdaSpace{}%
\AgdaFunction{toTruthValue}\AgdaSpace{}%
\AgdaSymbol{\{}\AgdaBound{\,f\,}\AgdaSymbol{\}}\AgdaSpace{}%
\AgdaSymbol{(}\AgdaInductiveConstructor{Any.just}\AgdaSpace{}%
\AgdaInductiveConstructor{tt}\AgdaSymbol{)}\<%
\\
%
\>[2]\AgdaSymbol{...}\AgdaSpace{}%
\AgdaSymbol{|}\AgdaSpace{}%
\AgdaInductiveConstructor{true}\AgdaSpace{}%
\AgdaSymbol{=}\AgdaSpace{}%
\AgdaFunction{ssEvalwithFuel}\AgdaSpace{}%
\AgdaBound{n}\AgdaSpace{}%
\AgdaSymbol{(}\AgdaBound{c₁}\AgdaSpace{}%
\AgdaOperator{\AgdaFunction{𝔱𝔥𝔢𝔫}}\AgdaSpace{}%
\AgdaBound{c₃}\AgdaSymbol{)}\AgdaSpace{}%
\AgdaBound{s}\<%
\\
%
\>[2]\AgdaSymbol{...}\AgdaSpace{}%
\AgdaSymbol{|}\AgdaSpace{}%
\AgdaInductiveConstructor{false}\AgdaSpace{}%
\AgdaSymbol{=}\AgdaSpace{}%
\AgdaFunction{ssEvalwithFuel}\AgdaSpace{}%
\AgdaBound{n}\AgdaSpace{}%
\AgdaSymbol{(}\AgdaBound{c₂}\AgdaSpace{}%
\AgdaOperator{\AgdaFunction{𝔱𝔥𝔢𝔫}}\AgdaSpace{}%
\AgdaBound{c₃}\AgdaSymbol{)}\AgdaSpace{}%
\AgdaBound{s}\<%
\\
%
\>[2]\AgdaComment{----------------------------------------------------------------------------------------------------------------------}\<%
\\
%
\>[2]\AgdaComment{--\,\!- ASSI ; THEN C}\<%
\\
%
\>[2]\AgdaFunction{ssEvalwithFuel}\AgdaSpace{}%
\AgdaSymbol{(}\AgdaInductiveConstructor{suc}\AgdaSpace{}%
\AgdaBound{n}\AgdaSymbol{)}\AgdaSpace{}%
\AgdaSymbol{((}\AgdaBound{id}\AgdaSpace{}%
\AgdaOperator{\AgdaInductiveConstructor{:=}}\AgdaSpace{}%
\AgdaBound{exp}\AgdaSymbol{)}\AgdaSpace{}%
\AgdaOperator{\AgdaInductiveConstructor{;}}\AgdaSpace{}%
\AgdaBound{c}\AgdaSymbol{)}\AgdaSpace{}%
\AgdaBound{s}\AgdaSpace{}%
\AgdaKeyword{with}\AgdaSpace{}%
\AgdaFunction{evalExp}\AgdaSpace{}%
\AgdaBound{exp}\AgdaSpace{}%
\AgdaBound{s}\<%
\\
%
\>[2]\AgdaSymbol{...}\AgdaSpace{}%
\AgdaSymbol{|}\AgdaSpace{}%
\AgdaInductiveConstructor{nothing}\AgdaSpace{}%
\AgdaSymbol{=}\AgdaSpace{}%
\AgdaInductiveConstructor{nothing}\AgdaSpace{}%
\AgdaComment{-\,\!- Computation failed i.e. div by 0}\<%
\\
%
\>[2]\AgdaSymbol{...}\AgdaSpace{}%
\AgdaSymbol{|}\AgdaSpace{}%
\AgdaSymbol{(}\AgdaInductiveConstructor{just}\AgdaSpace{}%
\AgdaBound{v}\AgdaSymbol{)}\AgdaSpace{}%
\AgdaSymbol{=}\AgdaSpace{}%
\AgdaFunction{ssEvalwithFuel}\AgdaSpace{}%
\AgdaBound{n}\AgdaSpace{}%
\AgdaBound{c}\AgdaSpace{}%
\AgdaSymbol{(}\AgdaField{updateState}\AgdaSpace{}%
\AgdaBound{id}\AgdaSpace{}%
\AgdaBound{v}\AgdaSpace{}%
\AgdaBound{s}\AgdaSymbol{)}\<%
\\
%
\>[2]\AgdaComment{----------------------------------------------------------------------------------------------------------------------}\<%
\end{code}
\end{figure}

\begin{figure}
  \caption{The full proof that evaluation is deterministic via proof that for any two proofs of termination, the resultant states serving as evidence for each of those proofs - in accordance with them being \emph{constructive} proofs - will be identical.  \\ n.b. that the $\dagger$ function is the function that extracts the witness from the proof of termination - i.e. the resultant state after the computation has terminated successfully.}
  \centering
  \scriptsize
  \begin{code}
\>[2]\AgdaComment{-------------------------------------------------------------------------------------------------------------------------------------------------------------------}\<%
\\
%
\>[2]\AgdaFunction{EvalDet}\AgdaSpace{}%
\AgdaSymbol{:}\AgdaSpace{}%
\AgdaSymbol{∀}\AgdaSpace{}%
\AgdaSymbol{\{}\AgdaBound{s}\AgdaSpace{}%
\AgdaBound{ℱ}%
\>[325I]\AgdaBound{ℱ'}\AgdaSymbol{\}}\AgdaSpace{}%
\AgdaBound{C}\<%
\\
\>[325I][@{}l@{\AgdaIndent{0}}]%
\>[21]\AgdaSymbol{→}\AgdaSpace{}%
\AgdaSymbol{(}\AgdaBound{a}\AgdaSpace{}%
\AgdaSymbol{:}\AgdaSpace{}%
\AgdaOperator{\AgdaFunction{⌊ᵗ}}\AgdaSpace{}%
\AgdaBound{ℱ}\AgdaSpace{}%
\AgdaOperator{\AgdaFunction{⸴}}\AgdaSpace{}%
\AgdaBound{C}\AgdaSpace{}%
\AgdaOperator{\AgdaFunction{⸴}}\AgdaSpace{}%
\AgdaBound{s}\AgdaSpace{}%
\AgdaOperator{\AgdaFunction{ᵗ⌋}}\AgdaSymbol{)}\AgdaSpace{}%
\AgdaSymbol{→}\AgdaSpace{}%
\AgdaSymbol{(}\AgdaBound{b}\AgdaSpace{}%
\AgdaSymbol{:}\AgdaSpace{}%
\AgdaOperator{\AgdaFunction{⌊ᵗ}}\AgdaSpace{}%
\AgdaBound{ℱ'}\AgdaSpace{}%
\AgdaOperator{\AgdaFunction{⸴}}\AgdaSpace{}%
\AgdaBound{C}\AgdaSpace{}%
\AgdaOperator{\AgdaFunction{⸴}}\AgdaSpace{}%
\AgdaBound{s}\AgdaSpace{}%
\AgdaOperator{\AgdaFunction{ᵗ⌋}}\AgdaSymbol{)}\AgdaSpace{}%
\AgdaSymbol{→}\AgdaSpace{}%
\AgdaFunction{″}\AgdaSpace{}%
\AgdaBound{a}\AgdaSpace{}%
\AgdaOperator{\AgdaDatatype{≡}}\AgdaSpace{}%
\AgdaFunction{″}\AgdaSpace{}%
\AgdaBound{b}\<%
\\
%
\>[2]\AgdaComment{-------------------------------------------------------------------------------------------------------------------------------------------------------------------}\<%
\\
%
\>[2]\AgdaKeyword{pattern}\AgdaSpace{}%
\AgdaInductiveConstructor{⇧}\AgdaSpace{}%
\AgdaBound{x}\AgdaSpace{}%
\AgdaSymbol{=}\AgdaSpace{}%
\AgdaInductiveConstructor{suc}\AgdaSpace{}%
\AgdaBound{x}\<%
\\
%
\>[2]\AgdaFunction{EvaluationIsDeterministic}\AgdaSpace{}%
\AgdaSymbol{=}\AgdaSpace{}%
\AgdaFunction{EvalDet}\<%
\\
%
\>[2]\AgdaComment{-------------------------------------------------------------------------------------------------------------------------------------------------------------------}\<%
\\
%
\>[2]\AgdaFunction{EvalDet}\AgdaSpace{}%
\AgdaSymbol{\{}\AgdaBound{s}\AgdaSymbol{\}}\AgdaSpace{}%
\AgdaSymbol{\{}\AgdaNumber{0}\AgdaSymbol{\}}\AgdaSpace{}%
\AgdaSymbol{\{}\AgdaNumber{0}\AgdaSymbol{\}}\AgdaSpace{}%
\AgdaSymbol{(\AgdaUnderscore{}}\AgdaSpace{}%
\AgdaOperator{\AgdaInductiveConstructor{;}}\AgdaSymbol{)}\AgdaSpace{}%
\AgdaBound{ij₁}\AgdaSpace{}%
\AgdaBound{ij₂}\AgdaSpace{}%
\AgdaKeyword{rewrite}\AgdaSpace{}%
\AgdaFunction{∃!IJ}\AgdaSpace{}%
\AgdaBound{ij₁}\AgdaSpace{}%
\AgdaBound{ij₂}\AgdaSpace{}%
\AgdaSymbol{=}\AgdaSpace{}%
\AgdaInductiveConstructor{refl}\<%
\\
%
\>[2]\AgdaFunction{EvalDet}\AgdaSpace{}%
\AgdaSymbol{\{}\AgdaBound{s}\AgdaSymbol{\}}\AgdaSpace{}%
\AgdaSymbol{\{}\AgdaNumber{0}\AgdaSymbol{\}}\AgdaSpace{}%
\AgdaSymbol{\{}\AgdaInductiveConstructor{⇧}\AgdaSpace{}%
\AgdaSymbol{\AgdaUnderscore{}\}}\AgdaSpace{}%
\AgdaSymbol{(}\AgdaInductiveConstructor{𝑠𝑘𝑖𝑝}\AgdaSpace{}%
\AgdaOperator{\AgdaInductiveConstructor{;}}\AgdaSymbol{)}\AgdaSpace{}%
\AgdaBound{ij₁}\AgdaSpace{}%
\AgdaBound{ij₂}\AgdaSpace{}%
\AgdaKeyword{rewrite}\AgdaSpace{}%
\AgdaFunction{∃!IJ}\AgdaSpace{}%
\AgdaBound{ij₁}\AgdaSpace{}%
\AgdaBound{ij₂}\AgdaSpace{}%
\AgdaSymbol{=}\AgdaSpace{}%
\AgdaInductiveConstructor{refl}\<%
\\
%
\>[2]\AgdaFunction{EvalDet}\AgdaSpace{}%
\AgdaSymbol{\{}\AgdaBound{s}\AgdaSymbol{\}}\AgdaSpace{}%
\AgdaSymbol{\{}\AgdaInductiveConstructor{⇧}\AgdaSpace{}%
\AgdaSymbol{\AgdaUnderscore{}\}}\AgdaSpace{}%
\AgdaSymbol{\{}\AgdaNumber{0}\AgdaSymbol{\}}\AgdaSpace{}%
\AgdaSymbol{(}\AgdaInductiveConstructor{𝑠𝑘𝑖𝑝}\AgdaSpace{}%
\AgdaOperator{\AgdaInductiveConstructor{;}}\AgdaSymbol{)}\AgdaSpace{}%
\AgdaBound{ij₁}\AgdaSpace{}%
\AgdaBound{ij₂}\AgdaSpace{}%
\AgdaKeyword{rewrite}\AgdaSpace{}%
\AgdaFunction{∃!IJ}\AgdaSpace{}%
\AgdaBound{ij₁}\AgdaSpace{}%
\AgdaBound{ij₂}\AgdaSpace{}%
\AgdaSymbol{=}\AgdaSpace{}%
\AgdaInductiveConstructor{refl}\<%
\\
%
\>[2]\AgdaFunction{EvalDet}\AgdaSpace{}%
\AgdaSymbol{\{}\AgdaBound{s}\AgdaSymbol{\}}\AgdaSpace{}%
\AgdaSymbol{\{}\AgdaInductiveConstructor{⇧}\AgdaSpace{}%
\AgdaSymbol{\AgdaUnderscore{}\}}\AgdaSpace{}%
\AgdaSymbol{\{}\AgdaInductiveConstructor{⇧}\AgdaSpace{}%
\AgdaSymbol{\AgdaUnderscore{}\}}\AgdaSpace{}%
\AgdaSymbol{(}\AgdaInductiveConstructor{𝑠𝑘𝑖𝑝}\AgdaSpace{}%
\AgdaOperator{\AgdaInductiveConstructor{;}}\AgdaSymbol{)}\AgdaSpace{}%
\AgdaBound{ij₁}\AgdaSpace{}%
\AgdaBound{ij₂}\AgdaSpace{}%
\AgdaKeyword{rewrite}\AgdaSpace{}%
\AgdaFunction{∃!IJ}\AgdaSpace{}%
\AgdaBound{ij₁}\AgdaSpace{}%
\AgdaBound{ij₂}\AgdaSpace{}%
\AgdaSymbol{=}\AgdaSpace{}%
\AgdaInductiveConstructor{refl}\<%
\\
%
\>[2]\AgdaFunction{EvalDet}\AgdaSpace{}%
\AgdaSymbol{\{}\AgdaBound{s}\AgdaSymbol{\}}\AgdaSpace{}%
\AgdaSymbol{\{}\AgdaInductiveConstructor{⇧}\AgdaSpace{}%
\AgdaBound{ℱ}\AgdaSymbol{\}}\AgdaSpace{}%
\AgdaSymbol{\{}\AgdaInductiveConstructor{⇧}\AgdaSpace{}%
\AgdaBound{ℱ'}\AgdaSymbol{\}}\AgdaSpace{}%
\AgdaSymbol{((}\AgdaOperator{\AgdaInductiveConstructor{𝔴𝔥𝔦𝔩𝔢}}\AgdaSpace{}%
\AgdaBound{exp}\AgdaSpace{}%
\AgdaOperator{\AgdaInductiveConstructor{𝒹ℴ}}\AgdaSpace{}%
\AgdaBound{c}\AgdaSymbol{)}\AgdaSpace{}%
\AgdaOperator{\AgdaInductiveConstructor{;}}\AgdaSymbol{)}\AgdaSpace{}%
\AgdaBound{ij₁}\AgdaSpace{}%
\AgdaBound{ij₂}\<%
\\
\>[2][@{}l@{\AgdaIndent{0}}]%
\>[4]\AgdaKeyword{with}\AgdaSpace{}%
\AgdaFunction{evalExp}\AgdaSpace{}%
\AgdaBound{exp}\AgdaSpace{}%
\AgdaBound{s}\<%
\\
%
\>[2]\AgdaSymbol{...}\AgdaSpace{}%
\AgdaSymbol{|}\AgdaSpace{}%
\AgdaBound{cond}\AgdaSymbol{@(}\AgdaInductiveConstructor{just}\AgdaSpace{}%
\AgdaSymbol{\AgdaUnderscore{})}\AgdaSpace{}%
\AgdaKeyword{with}\AgdaSpace{}%
\AgdaFunction{toTruthValue}\AgdaSpace{}%
\AgdaSymbol{\{}\AgdaBound{cond}\AgdaSymbol{\}}\AgdaSpace{}%
\AgdaSymbol{(}\AgdaInductiveConstructor{Any.just}\AgdaSpace{}%
\AgdaInductiveConstructor{tt}\AgdaSymbol{)}\<%
\\
%
\>[2]\AgdaSymbol{...}\AgdaSpace{}%
\AgdaSymbol{|}\AgdaSpace{}%
\AgdaInductiveConstructor{false}\AgdaSpace{}%
\AgdaKeyword{rewrite}\AgdaSpace{}%
\AgdaFunction{∃!IJ}\AgdaSpace{}%
\AgdaBound{ij₁}\AgdaSpace{}%
\AgdaBound{ij₂}\AgdaSpace{}%
\AgdaSymbol{=}\AgdaSpace{}%
\AgdaInductiveConstructor{refl}\<%
\\
%
\>[2]\AgdaSymbol{...}\AgdaSpace{}%
\AgdaSymbol{|}\AgdaSpace{}%
\AgdaInductiveConstructor{true}\AgdaSpace{}%
\AgdaSymbol{=}\AgdaSpace{}%
\AgdaFunction{EvalDet}\AgdaSpace{}%
\AgdaSymbol{\{}\AgdaBound{s}\AgdaSymbol{\}}\AgdaSpace{}%
\AgdaSymbol{\{}\AgdaBound{ℱ}\AgdaSymbol{\}}\AgdaSpace{}%
\AgdaSymbol{\{}\AgdaBound{ℱ'}\AgdaSymbol{\}}\AgdaSpace{}%
\AgdaSymbol{\AgdaUnderscore{}}\AgdaSpace{}%
\AgdaBound{ij₁}\AgdaSpace{}%
\AgdaBound{ij₂}\<%
\\
%
\>[2]\AgdaFunction{EvalDet}\AgdaSpace{}%
\AgdaSymbol{\{}\AgdaBound{s}\AgdaSymbol{\}}\AgdaSpace{}%
\AgdaSymbol{\{}\AgdaInductiveConstructor{⇧}\AgdaSpace{}%
\AgdaBound{ℱ}\AgdaSymbol{\}}\AgdaSpace{}%
\AgdaSymbol{\{}\AgdaInductiveConstructor{⇧}\AgdaSpace{}%
\AgdaBound{ℱ'}\AgdaSymbol{\}}\AgdaSpace{}%
\AgdaSymbol{((}\AgdaOperator{\AgdaInductiveConstructor{𝔦𝔣}}\AgdaSpace{}%
\AgdaBound{exp}\AgdaSpace{}%
\AgdaOperator{\AgdaInductiveConstructor{𝔱𝔥𝔢𝔫}}\AgdaSpace{}%
\AgdaBound{c₁}\AgdaSpace{}%
\AgdaOperator{\AgdaInductiveConstructor{𝔢𝔩𝔰𝔢}}\AgdaSpace{}%
\AgdaBound{c₂}\AgdaSymbol{)}\AgdaSpace{}%
\AgdaOperator{\AgdaInductiveConstructor{;}}\AgdaSymbol{)}\AgdaSpace{}%
\AgdaBound{ij₁}\AgdaSpace{}%
\AgdaBound{ij₂}\<%
\\
\>[2][@{}l@{\AgdaIndent{0}}]%
\>[4]\AgdaKeyword{with}\AgdaSpace{}%
\AgdaFunction{evalExp}\AgdaSpace{}%
\AgdaBound{exp}\AgdaSpace{}%
\AgdaBound{s}\<%
\\
%
\>[2]\AgdaSymbol{...}\AgdaSpace{}%
\AgdaSymbol{|}\AgdaSpace{}%
\AgdaBound{cond}\AgdaSymbol{@(}\AgdaInductiveConstructor{just}\AgdaSpace{}%
\AgdaSymbol{\AgdaUnderscore{})}\AgdaSpace{}%
\AgdaKeyword{with}\AgdaSpace{}%
\AgdaFunction{toTruthValue}\AgdaSpace{}%
\AgdaSymbol{\{}\AgdaBound{cond}\AgdaSymbol{\}}\AgdaSpace{}%
\AgdaSymbol{(}\AgdaInductiveConstructor{Any.just}\AgdaSpace{}%
\AgdaInductiveConstructor{tt}\AgdaSymbol{)}\<%
\\
%
\>[2]\AgdaSymbol{...}\AgdaSpace{}%
\AgdaSymbol{|}\AgdaSpace{}%
\AgdaInductiveConstructor{false}\AgdaSpace{}%
\AgdaSymbol{=}\AgdaSpace{}%
\AgdaFunction{EvalDet}\AgdaSpace{}%
\AgdaSymbol{\{}\AgdaBound{s}\AgdaSymbol{\}}\AgdaSpace{}%
\AgdaSymbol{\{}\AgdaBound{ℱ}\AgdaSymbol{\}}\AgdaSpace{}%
\AgdaSymbol{\{}\AgdaBound{ℱ'}\AgdaSymbol{\}}\AgdaSpace{}%
\AgdaSymbol{\AgdaUnderscore{}}\AgdaSpace{}%
\AgdaBound{ij₁}\AgdaSpace{}%
\AgdaBound{ij₂}\<%
\\
%
\>[2]\AgdaSymbol{...}\AgdaSpace{}%
\AgdaSymbol{|}\AgdaSpace{}%
\AgdaInductiveConstructor{true}\AgdaSpace{}%
\AgdaSymbol{=}\AgdaSpace{}%
\AgdaFunction{EvalDet}\AgdaSpace{}%
\AgdaSymbol{\{}\AgdaBound{s}\AgdaSymbol{\}}\AgdaSpace{}%
\AgdaSymbol{\{}\AgdaBound{ℱ}\AgdaSymbol{\}}\AgdaSpace{}%
\AgdaSymbol{\{}\AgdaBound{ℱ'}\AgdaSymbol{\}}\AgdaSpace{}%
\AgdaSymbol{\AgdaUnderscore{}}\AgdaSpace{}%
\AgdaBound{ij₁}\AgdaSpace{}%
\AgdaBound{ij₂}\<%
\\
%
\>[2]\AgdaFunction{EvalDet}\AgdaSpace{}%
\AgdaSymbol{\{}\AgdaBound{s}\AgdaSymbol{\}}\AgdaSpace{}%
\AgdaSymbol{\{}\AgdaInductiveConstructor{⇧}\AgdaSpace{}%
\AgdaBound{ℱ}\AgdaSymbol{\}}\AgdaSpace{}%
\AgdaSymbol{\{}\AgdaInductiveConstructor{⇧}\AgdaSpace{}%
\AgdaBound{ℱ'}\AgdaSymbol{\}}\AgdaSpace{}%
\AgdaSymbol{((}\AgdaBound{id}\AgdaSpace{}%
\AgdaOperator{\AgdaInductiveConstructor{:=}}\AgdaSpace{}%
\AgdaBound{exp}\AgdaSymbol{)}\AgdaSpace{}%
\AgdaOperator{\AgdaInductiveConstructor{;}}\AgdaSymbol{)}\AgdaSpace{}%
\AgdaBound{ij₁}\AgdaSpace{}%
\AgdaBound{ij₂}\<%
\\
\>[2][@{}l@{\AgdaIndent{0}}]%
\>[4]\AgdaKeyword{with}\AgdaSpace{}%
\AgdaFunction{evalExp}\AgdaSpace{}%
\AgdaBound{exp}\AgdaSpace{}%
\AgdaBound{s}\<%
\\
%
\>[2]\AgdaSymbol{...}\AgdaSpace{}%
\AgdaSymbol{|}\AgdaSpace{}%
\AgdaSymbol{(}\AgdaInductiveConstructor{just}\AgdaSpace{}%
\AgdaSymbol{\AgdaUnderscore{})}\AgdaSpace{}%
\AgdaKeyword{rewrite}\AgdaSpace{}%
\AgdaFunction{∃!IJ}\AgdaSpace{}%
\AgdaBound{ij₁}\AgdaSpace{}%
\AgdaBound{ij₂}\AgdaSpace{}%
\AgdaSymbol{=}\AgdaSpace{}%
\AgdaInductiveConstructor{refl}\<%
\\
%
\>[2]\AgdaFunction{EvalDet}\AgdaSpace{}%
\AgdaSymbol{\{}\AgdaBound{s}\AgdaSymbol{\}}\AgdaSpace{}%
\AgdaSymbol{\{}\AgdaInductiveConstructor{⇧}\AgdaSpace{}%
\AgdaBound{ℱ}\AgdaSymbol{\}}\AgdaSpace{}%
\AgdaSymbol{\{}\AgdaInductiveConstructor{⇧}\AgdaSpace{}%
\AgdaBound{ℱ'}\AgdaSymbol{\}}\AgdaSpace{}%
\AgdaSymbol{((}\AgdaOperator{\AgdaInductiveConstructor{𝔴𝔥𝔦𝔩𝔢}}\AgdaSpace{}%
\AgdaBound{exp}\AgdaSpace{}%
\AgdaOperator{\AgdaInductiveConstructor{𝒹ℴ}}\AgdaSpace{}%
\AgdaBound{c₁}\AgdaSymbol{)}\AgdaSpace{}%
\AgdaOperator{\AgdaInductiveConstructor{;}}\AgdaSpace{}%
\AgdaBound{c₂}\AgdaSymbol{)}\AgdaSpace{}%
\AgdaBound{ij₁}\AgdaSpace{}%
\AgdaBound{ij₂}\<%
\\
\>[2][@{}l@{\AgdaIndent{0}}]%
\>[4]\AgdaKeyword{with}\AgdaSpace{}%
\AgdaFunction{evalExp}\AgdaSpace{}%
\AgdaBound{exp}\AgdaSpace{}%
\AgdaBound{s}\<%
\\
%
\>[2]\AgdaSymbol{...}\AgdaSpace{}%
\AgdaSymbol{|}\AgdaSpace{}%
\AgdaBound{cond}\AgdaSymbol{@(}\AgdaInductiveConstructor{just}\AgdaSpace{}%
\AgdaSymbol{\AgdaUnderscore{})}\AgdaSpace{}%
\AgdaKeyword{with}\AgdaSpace{}%
\AgdaFunction{toTruthValue}\AgdaSpace{}%
\AgdaSymbol{\{}\AgdaBound{cond}\AgdaSymbol{\}}\AgdaSpace{}%
\AgdaSymbol{(}\AgdaInductiveConstructor{Any.just}\AgdaSpace{}%
\AgdaInductiveConstructor{tt}\AgdaSymbol{)}\<%
\\
%
\>[2]\AgdaSymbol{...}\AgdaSpace{}%
\AgdaSymbol{|}\AgdaSpace{}%
\AgdaInductiveConstructor{false}\AgdaSpace{}%
\AgdaSymbol{=}\AgdaSpace{}%
\AgdaFunction{EvalDet}\AgdaSpace{}%
\AgdaSymbol{\{}\AgdaBound{s}\AgdaSymbol{\}}\AgdaSpace{}%
\AgdaSymbol{\{}\AgdaBound{ℱ}\AgdaSymbol{\}}\AgdaSpace{}%
\AgdaSymbol{\{}\AgdaBound{ℱ'}\AgdaSymbol{\}}\AgdaSpace{}%
\AgdaSymbol{\AgdaUnderscore{}}\AgdaSpace{}%
\AgdaBound{ij₁}\AgdaSpace{}%
\AgdaBound{ij₂}\<%
\\
%
\>[2]\AgdaSymbol{...}\AgdaSpace{}%
\AgdaSymbol{|}\AgdaSpace{}%
\AgdaInductiveConstructor{true}\AgdaSpace{}%
\AgdaSymbol{=}\AgdaSpace{}%
\AgdaFunction{EvalDet}\AgdaSpace{}%
\AgdaSymbol{\{}\AgdaBound{s}\AgdaSymbol{\}}\AgdaSpace{}%
\AgdaSymbol{\{}\AgdaBound{ℱ}\AgdaSymbol{\}}\AgdaSpace{}%
\AgdaSymbol{\{}\AgdaBound{ℱ'}\AgdaSymbol{\}}\AgdaSpace{}%
\AgdaSymbol{\AgdaUnderscore{}}\AgdaSpace{}%
\AgdaBound{ij₁}\AgdaSpace{}%
\AgdaBound{ij₂}\<%
\\
%
\>[2]\AgdaFunction{EvalDet}\AgdaSpace{}%
\AgdaSymbol{\{}\AgdaBound{s}\AgdaSymbol{\}}\AgdaSpace{}%
\AgdaSymbol{\{}\AgdaInductiveConstructor{⇧}\AgdaSpace{}%
\AgdaBound{ℱ}\AgdaSymbol{\}}\AgdaSpace{}%
\AgdaSymbol{\{}\AgdaInductiveConstructor{⇧}\AgdaSpace{}%
\AgdaBound{ℱ'}\AgdaSymbol{\}}\AgdaSpace{}%
\AgdaSymbol{((}\AgdaOperator{\AgdaInductiveConstructor{𝔦𝔣}}\AgdaSpace{}%
\AgdaBound{exp}\AgdaSpace{}%
\AgdaOperator{\AgdaInductiveConstructor{𝔱𝔥𝔢𝔫}}\AgdaSpace{}%
\AgdaBound{c₁}\AgdaSpace{}%
\AgdaOperator{\AgdaInductiveConstructor{𝔢𝔩𝔰𝔢}}\AgdaSpace{}%
\AgdaBound{c₂}\AgdaSymbol{)}\AgdaSpace{}%
\AgdaOperator{\AgdaInductiveConstructor{;}}\AgdaSpace{}%
\AgdaBound{c₃}\AgdaSymbol{)}\AgdaSpace{}%
\AgdaBound{ij₁}\AgdaSpace{}%
\AgdaBound{ij₂}\<%
\\
\>[2][@{}l@{\AgdaIndent{0}}]%
\>[4]\AgdaKeyword{with}\AgdaSpace{}%
\AgdaFunction{evalExp}\AgdaSpace{}%
\AgdaBound{exp}\AgdaSpace{}%
\AgdaBound{s}\<%
\\
%
\>[2]\AgdaSymbol{...}\AgdaSpace{}%
\AgdaSymbol{|}\AgdaSpace{}%
\AgdaBound{cond}\AgdaSymbol{@(}\AgdaInductiveConstructor{just}\AgdaSpace{}%
\AgdaSymbol{\AgdaUnderscore{})}\AgdaSpace{}%
\AgdaKeyword{with}\AgdaSpace{}%
\AgdaFunction{toTruthValue}\AgdaSpace{}%
\AgdaSymbol{\{}\AgdaBound{cond}\AgdaSymbol{\}}\AgdaSpace{}%
\AgdaSymbol{(}\AgdaInductiveConstructor{Any.just}\AgdaSpace{}%
\AgdaInductiveConstructor{tt}\AgdaSymbol{)}\<%
\\
%
\>[2]\AgdaSymbol{...}\AgdaSpace{}%
\AgdaSymbol{|}\AgdaSpace{}%
\AgdaInductiveConstructor{false}\AgdaSpace{}%
\AgdaSymbol{=}\AgdaSpace{}%
\AgdaFunction{EvalDet}\AgdaSpace{}%
\AgdaSymbol{\{}\AgdaBound{s}\AgdaSymbol{\}}\AgdaSpace{}%
\AgdaSymbol{\{}\AgdaBound{ℱ}\AgdaSymbol{\}}\AgdaSpace{}%
\AgdaSymbol{\{}\AgdaBound{ℱ'}\AgdaSymbol{\}}\AgdaSpace{}%
\AgdaSymbol{\AgdaUnderscore{}}\AgdaSpace{}%
\AgdaBound{ij₁}\AgdaSpace{}%
\AgdaBound{ij₂}\<%
\\
%
\>[2]\AgdaSymbol{...}\AgdaSpace{}%
\AgdaSymbol{|}\AgdaSpace{}%
\AgdaInductiveConstructor{true}\AgdaSpace{}%
\AgdaSymbol{=}\AgdaSpace{}%
\AgdaFunction{EvalDet}\AgdaSpace{}%
\AgdaSymbol{\{}\AgdaBound{s}\AgdaSymbol{\}}\AgdaSpace{}%
\AgdaSymbol{\{}\AgdaBound{ℱ}\AgdaSymbol{\}}\AgdaSpace{}%
\AgdaSymbol{\{}\AgdaBound{ℱ'}\AgdaSymbol{\}}\AgdaSpace{}%
\AgdaSymbol{\AgdaUnderscore{}}\AgdaSpace{}%
\AgdaBound{ij₁}\AgdaSpace{}%
\AgdaBound{ij₂}\<%
\\
%
\>[2]\AgdaFunction{EvalDet}\AgdaSpace{}%
\AgdaSymbol{\{}\AgdaBound{s}\AgdaSymbol{\}}\AgdaSpace{}%
\AgdaSymbol{\{}\AgdaInductiveConstructor{⇧}\AgdaSpace{}%
\AgdaBound{ℱ}\AgdaSymbol{\}}\AgdaSpace{}%
\AgdaSymbol{\{}\AgdaInductiveConstructor{⇧}\AgdaSpace{}%
\AgdaBound{ℱ'}\AgdaSymbol{\}}\AgdaSpace{}%
\AgdaSymbol{((}\AgdaBound{id}\AgdaSpace{}%
\AgdaOperator{\AgdaInductiveConstructor{:=}}\AgdaSpace{}%
\AgdaBound{exp}\AgdaSymbol{)}\AgdaSpace{}%
\AgdaOperator{\AgdaInductiveConstructor{;}}\AgdaSpace{}%
\AgdaBound{c}\AgdaSymbol{)}\AgdaSpace{}%
\AgdaBound{ij₁}\AgdaSpace{}%
\AgdaBound{ij₂}\<%
\\
\>[2][@{}l@{\AgdaIndent{0}}]%
\>[4]\AgdaKeyword{with}\AgdaSpace{}%
\AgdaFunction{evalExp}\AgdaSpace{}%
\AgdaBound{exp}\AgdaSpace{}%
\AgdaBound{s}\<%
\\
%
\>[2]\AgdaSymbol{...}\AgdaSpace{}%
\AgdaSymbol{|}\AgdaSpace{}%
\AgdaSymbol{(}\AgdaInductiveConstructor{just}\AgdaSpace{}%
\AgdaBound{v}\AgdaSymbol{)}\AgdaSpace{}%
\AgdaSymbol{=}\AgdaSpace{}%
\AgdaFunction{EvalDet}\AgdaSpace{}%
\AgdaSymbol{\{}\AgdaField{updateState}\AgdaSpace{}%
\AgdaBound{id}\AgdaSpace{}%
\AgdaBound{v}\AgdaSpace{}%
\AgdaBound{s}\AgdaSymbol{\}}\AgdaSpace{}%
\AgdaSymbol{\{}\AgdaBound{ℱ}\AgdaSymbol{\}}\AgdaSpace{}%
\AgdaSymbol{\{}\AgdaBound{ℱ'}\AgdaSymbol{\}}\AgdaSpace{}%
\AgdaSymbol{\AgdaUnderscore{}}\AgdaSpace{}%
\AgdaBound{ij₁}\AgdaSpace{}%
\AgdaBound{ij₂}\<%
\\
%
\>[2]\AgdaFunction{EvalDet}\AgdaSpace{}%
\AgdaSymbol{\{}\AgdaBound{s}\AgdaSymbol{\}}\AgdaSpace{}%
\AgdaSymbol{\{}\AgdaInductiveConstructor{⇧}\AgdaSpace{}%
\AgdaBound{ℱ}\AgdaSymbol{\}}\AgdaSpace{}%
\AgdaSymbol{\{}\AgdaInductiveConstructor{⇧}\AgdaSpace{}%
\AgdaBound{ℱ'}\AgdaSymbol{\}}\AgdaSpace{}%
\AgdaSymbol{(}\AgdaInductiveConstructor{𝑠𝑘𝑖𝑝}\AgdaSpace{}%
\AgdaOperator{\AgdaInductiveConstructor{;}}\AgdaSpace{}%
\AgdaBound{c}\AgdaSymbol{)}\AgdaSpace{}%
\AgdaSymbol{=}\AgdaSpace{}%
\AgdaFunction{EvalDet}\AgdaSpace{}%
\AgdaSymbol{\{}\AgdaBound{s}\AgdaSymbol{\}}\AgdaSpace{}%
\AgdaSymbol{\{}\AgdaInductiveConstructor{⇧}\AgdaSpace{}%
\AgdaBound{ℱ}\AgdaSymbol{\}}\AgdaSpace{}%
\AgdaSymbol{\{}\AgdaInductiveConstructor{⇧}\AgdaSpace{}%
\AgdaBound{ℱ'}\AgdaSymbol{\}}\AgdaSpace{}%
\AgdaBound{c}\<%
\\
%
\>[2]\AgdaFunction{EvalDet}\AgdaSpace{}%
\AgdaSymbol{\{}\AgdaBound{s}\AgdaSymbol{\}}\AgdaSpace{}%
\AgdaSymbol{\{}\AgdaNumber{0}\AgdaSymbol{\}}\AgdaSpace{}%
\AgdaSymbol{\{}\AgdaNumber{0}\AgdaSymbol{\}}\AgdaSpace{}%
\AgdaSymbol{(}\AgdaInductiveConstructor{𝑠𝑘𝑖𝑝}\AgdaSpace{}%
\AgdaOperator{\AgdaInductiveConstructor{;}}\AgdaSpace{}%
\AgdaBound{c}\AgdaSymbol{)}\AgdaSpace{}%
\AgdaBound{ij₁}\AgdaSpace{}%
\AgdaBound{ij₂}\AgdaSpace{}%
\AgdaKeyword{rewrite}\AgdaSpace{}%
\AgdaFunction{∃!IJ}\AgdaSpace{}%
\AgdaBound{ij₁}\AgdaSpace{}%
\AgdaBound{ij₂}\AgdaSpace{}%
\AgdaSymbol{=}\AgdaSpace{}%
\AgdaInductiveConstructor{refl}\<%
\\
%
\>[2]\AgdaComment{-\,\!- In the clause below, with ℱ = 0 and ℱ' = ⇧ \AgdaUnderscore{}, the only possibility if we}\<%
\\
%
\>[2]\AgdaComment{-\,\!- are still to have two proofs of termination in ij₁ and ij₂ is that the}\<%
\\
%
\>[2]\AgdaComment{-\,\!- rest of the mechanisms in c are all 𝑎𝑙𝑠𝑜 `𝑠𝑘𝑖𝑝'. So we take each of the two}\<%
\\
%
\>[2]\AgdaComment{-\,\!- cases of either c = (𝑠𝑘𝑖𝑝 ;) or c = (𝑠𝑘𝑖𝑝 ; ... ; (𝑠𝑘𝑖𝑝 ;)) in turn.}\<%
\\
%
\>[2]\AgdaComment{-\,\!- Annoyingly we have to do this for both permutations of ℱ/ℱ' = 0/⇧ \AgdaUnderscore{}}\<%
\\
%
\>[2]\AgdaFunction{EvalDet}\AgdaSpace{}%
\AgdaSymbol{\{}\AgdaBound{s}\AgdaSymbol{\}}\AgdaSpace{}%
\AgdaSymbol{\{}\AgdaNumber{0}\AgdaSymbol{\}}\AgdaSpace{}%
\AgdaSymbol{\{}\AgdaInductiveConstructor{⇧}\AgdaSpace{}%
\AgdaBound{ℱ'}\AgdaSymbol{\}}\AgdaSpace{}%
\AgdaSymbol{(}\AgdaInductiveConstructor{𝑠𝑘𝑖𝑝}\AgdaSpace{}%
\AgdaOperator{\AgdaInductiveConstructor{;}}\AgdaSpace{}%
\AgdaSymbol{(}\AgdaInductiveConstructor{𝑠𝑘𝑖𝑝}\AgdaSpace{}%
\AgdaOperator{\AgdaInductiveConstructor{;}}\AgdaSymbol{))}\AgdaSpace{}%
\AgdaBound{ij₁}\AgdaSpace{}%
\AgdaBound{ij₂}\AgdaSpace{}%
\AgdaKeyword{rewrite}\AgdaSpace{}%
\AgdaFunction{∃!IJ}\AgdaSpace{}%
\AgdaBound{ij₁}\AgdaSpace{}%
\AgdaBound{ij₂}\AgdaSpace{}%
\AgdaSymbol{=}\AgdaSpace{}%
\AgdaInductiveConstructor{refl}\<%
\\
%
\>[2]\AgdaFunction{EvalDet}\AgdaSpace{}%
\AgdaSymbol{\{}\AgdaBound{s}\AgdaSymbol{\}}\AgdaSpace{}%
\AgdaSymbol{\{}\AgdaNumber{0}\AgdaSymbol{\}}\AgdaSpace{}%
\AgdaSymbol{\{}\AgdaInductiveConstructor{⇧}\AgdaSpace{}%
\AgdaBound{ℱ'}\AgdaSymbol{\}}\AgdaSpace{}%
\AgdaSymbol{(}\AgdaInductiveConstructor{𝑠𝑘𝑖𝑝}\AgdaSpace{}%
\AgdaOperator{\AgdaInductiveConstructor{;}}\AgdaSpace{}%
\AgdaSymbol{(}\AgdaInductiveConstructor{𝑠𝑘𝑖𝑝}\AgdaSpace{}%
\AgdaOperator{\AgdaInductiveConstructor{;}}\AgdaSpace{}%
\AgdaBound{c}\AgdaSymbol{))}\AgdaSpace{}%
\AgdaSymbol{=}\AgdaSpace{}%
\AgdaFunction{EvalDet}\AgdaSpace{}%
\AgdaSymbol{\{}\AgdaBound{s}\AgdaSymbol{\}}\AgdaSpace{}%
\AgdaSymbol{\{}\AgdaNumber{0}\AgdaSymbol{\}}\AgdaSpace{}%
\AgdaSymbol{\{}\AgdaInductiveConstructor{⇧}\AgdaSpace{}%
\AgdaBound{ℱ'}\AgdaSymbol{\}}\AgdaSpace{}%
\AgdaBound{c}\<%
\\
%
\>[2]\AgdaFunction{EvalDet}\AgdaSpace{}%
\AgdaSymbol{\{}\AgdaBound{s}\AgdaSymbol{\}}\AgdaSpace{}%
\AgdaSymbol{\{}\AgdaInductiveConstructor{⇧}\AgdaSpace{}%
\AgdaBound{ℱ}\AgdaSymbol{\}}\AgdaSpace{}%
\AgdaSymbol{\{}\AgdaNumber{0}\AgdaSymbol{\}}\AgdaSpace{}%
\AgdaSymbol{(}\AgdaInductiveConstructor{𝑠𝑘𝑖𝑝}\AgdaSpace{}%
\AgdaOperator{\AgdaInductiveConstructor{;}}\AgdaSpace{}%
\AgdaSymbol{(}\AgdaInductiveConstructor{𝑠𝑘𝑖𝑝}\AgdaSpace{}%
\AgdaOperator{\AgdaInductiveConstructor{;}}\AgdaSymbol{))}\AgdaSpace{}%
\AgdaBound{ij₁}\AgdaSpace{}%
\AgdaBound{ij₂}\AgdaSpace{}%
\AgdaKeyword{rewrite}\AgdaSpace{}%
\AgdaFunction{∃!IJ}\AgdaSpace{}%
\AgdaBound{ij₁}\AgdaSpace{}%
\AgdaBound{ij₂}\AgdaSpace{}%
\AgdaSymbol{=}\AgdaSpace{}%
\AgdaInductiveConstructor{refl}\<%
\\
%
\>[2]\AgdaFunction{EvalDet}\AgdaSpace{}%
\AgdaSymbol{\{}\AgdaBound{s}\AgdaSymbol{\}}\AgdaSpace{}%
\AgdaSymbol{\{}\AgdaInductiveConstructor{⇧}\AgdaSpace{}%
\AgdaBound{ℱ}\AgdaSymbol{\}}\AgdaSpace{}%
\AgdaSymbol{\{}\AgdaNumber{0}\AgdaSymbol{\}}\AgdaSpace{}%
\AgdaSymbol{(}\AgdaInductiveConstructor{𝑠𝑘𝑖𝑝}\AgdaSpace{}%
\AgdaOperator{\AgdaInductiveConstructor{;}}\AgdaSpace{}%
\AgdaSymbol{(}\AgdaInductiveConstructor{𝑠𝑘𝑖𝑝}\AgdaSpace{}%
\AgdaOperator{\AgdaInductiveConstructor{;}}\AgdaSpace{}%
\AgdaBound{c}\AgdaSymbol{))}\AgdaSpace{}%
\AgdaSymbol{=}\AgdaSpace{}%
\AgdaFunction{EvalDet}\AgdaSpace{}%
\AgdaSymbol{\{}\AgdaBound{s}\AgdaSymbol{\}}\AgdaSpace{}%
\AgdaSymbol{\{}\AgdaInductiveConstructor{⇧}\AgdaSpace{}%
\AgdaBound{ℱ}\AgdaSymbol{\}}\AgdaSpace{}%
\AgdaSymbol{\{}\AgdaNumber{0}\AgdaSymbol{\}}\AgdaSpace{}%
\AgdaBound{c}\<%
\\
%
\>[2]\AgdaComment{-------------------------------------------------------------------------------------------------------------------------------------------------------------------}\<%
\end{code}
\end{figure}

\begin{figure}
  \caption{[t]-split: The function for splitting two proofs of termination. }
  \footnotesize
  {\centering \begin{code}
  \>[2]\AgdaComment{------------------------------------------------------------------------------------------------------------------ }\<%
\\
%
\>[2]\AgdaFunction{⌊ᵗ⌋-split'}%
\>[975I]\AgdaSymbol{:}\AgdaSpace{}%
\AgdaSymbol{∀}\AgdaSpace{}%
\AgdaBound{ℱ}\AgdaSpace{}%
\AgdaBound{s}\AgdaSpace{}%
\AgdaBound{Q₁}\AgdaSpace{}%
\AgdaBound{Q₂}\AgdaSpace{}%
\AgdaSymbol{→}\AgdaSpace{}%
\AgdaSymbol{(}\AgdaBound{t₁₂}\AgdaSpace{}%
\AgdaSymbol{:}\AgdaSpace{}%
\AgdaOperator{\AgdaFunction{⌊ᵗ}}\AgdaSpace{}%
\AgdaBound{ℱ}\AgdaSpace{}%
\AgdaOperator{\AgdaFunction{⸴}}\AgdaSpace{}%
\AgdaBound{Q₁}\AgdaSpace{}%
\AgdaOperator{\AgdaFunction{𝔱𝔥𝔢𝔫}}\AgdaSpace{}%
\AgdaBound{Q₂}\AgdaSpace{}%
\AgdaOperator{\AgdaFunction{⸴}}\AgdaSpace{}%
\AgdaBound{s}\AgdaSpace{}%
\AgdaOperator{\AgdaFunction{ᵗ⌋}}\AgdaSymbol{)}\<%
\\
\>[975I][@{}l@{\AgdaIndent{0}}]%
\>[14]\AgdaSymbol{→}\AgdaSpace{}%
\AgdaRecord{Σ}\AgdaSpace{}%
\AgdaOperator{\AgdaFunction{⌊ᵗ}}\AgdaSpace{}%
\AgdaBound{ℱ}\AgdaSpace{}%
\AgdaOperator{\AgdaFunction{⸴}}\AgdaSpace{}%
\AgdaBound{Q₁}\AgdaSpace{}%
\AgdaOperator{\AgdaFunction{⸴}}\AgdaSpace{}%
\AgdaBound{s}\AgdaSpace{}%
\AgdaOperator{\AgdaFunction{ᵗ⌋}}\AgdaSpace{}%
\AgdaSymbol{(λ}\AgdaSpace{}%
\AgdaBound{t₁}\<%
\\
%
\>[14]\AgdaSymbol{→}\AgdaSpace{}%
\AgdaRecord{Σ}\AgdaSpace{}%
\AgdaDatatype{ℕ}\AgdaSpace{}%
\AgdaSymbol{(λ}\AgdaSpace{}%
\AgdaBound{ℱ'}\<%
\\
%
\>[14]\AgdaSymbol{→}\AgdaSpace{}%
\AgdaBound{ℱ'}\AgdaSpace{}%
\AgdaOperator{\AgdaRecord{≤''}}\AgdaSpace{}%
\AgdaBound{ℱ}\AgdaSpace{}%
\AgdaOperator{\AgdaFunction{×}}\AgdaSpace{}%
\AgdaRecord{Σ}\AgdaSpace{}%
\AgdaOperator{\AgdaFunction{⌊ᵗ}}\AgdaSpace{}%
\AgdaBound{ℱ'}\AgdaSpace{}%
\AgdaOperator{\AgdaFunction{⸴}}\AgdaSpace{}%
\AgdaBound{Q₂}\AgdaSpace{}%
\AgdaOperator{\AgdaFunction{⸴}}\AgdaSpace{}%
\AgdaFunction{″}\AgdaSpace{}%
\AgdaBound{t₁}\AgdaSpace{}%
\AgdaOperator{\AgdaFunction{ᵗ⌋}}\AgdaSpace{}%
\AgdaSymbol{(λ}\AgdaSpace{}%
\AgdaBound{t₂}\<%
\\
%
\>[14]\AgdaSymbol{→}\AgdaSpace{}%
\AgdaFunction{″}\AgdaSpace{}%
\AgdaBound{t₂}\AgdaSpace{}%
\AgdaOperator{\AgdaDatatype{≡}}\AgdaSpace{}%
\AgdaFunction{″}\AgdaSpace{}%
\AgdaBound{t₁₂}\AgdaSpace{}%
\AgdaSymbol{)))}\<%
\\
%
\>[2]\AgdaComment{------------------------------------------------------------------------------------------------------------------ }\<%
\\
%
\\[\AgdaEmptyExtraSkip]%
%
\>[2]\AgdaComment{--------------------------------------------------------------------}\<%
\\
%
\>[2]\AgdaComment{-- Base case: Q₁ = 𝑠𝑘𝑖𝑝 ;}\<%
\\
%
\>[2]\AgdaFunction{⌊ᵗ⌋-split'}\AgdaSpace{}%
\AgdaBound{ℱ}\AgdaSymbol{@}\AgdaNumber{0}\AgdaSpace{}%
\AgdaBound{s}\AgdaSpace{}%
\AgdaSymbol{(}\AgdaInductiveConstructor{𝑠𝑘𝑖𝑝}\AgdaSpace{}%
\AgdaOperator{\AgdaInductiveConstructor{;}}\AgdaSymbol{)}\AgdaSpace{}%
\AgdaBound{Q₂}\AgdaSpace{}%
\AgdaBound{t₁₂}\AgdaSpace{}%
\AgdaSymbol{=}\<%
\\
\>[2][@{}l@{\AgdaIndent{0}}]%
\>[10]\AgdaSymbol{(}\AgdaInductiveConstructor{Any.just}\AgdaSpace{}%
\AgdaInductiveConstructor{tt}\AgdaSymbol{)}\AgdaSpace{}%
\AgdaOperator{\AgdaInductiveConstructor{,}}\AgdaSpace{}%
\AgdaBound{ℱ}\AgdaSpace{}%
\AgdaOperator{\AgdaInductiveConstructor{,}}\AgdaSpace{}%
\AgdaInductiveConstructor{≤with}\AgdaSpace{}%
\AgdaInductiveConstructor{refl}%
\>[42]\AgdaOperator{\AgdaInductiveConstructor{,}}\AgdaSpace{}%
\AgdaBound{t₁₂}\AgdaSpace{}%
\AgdaOperator{\AgdaInductiveConstructor{,}}\AgdaSpace{}%
\AgdaInductiveConstructor{refl}\<%
\\
%
\>[2]\AgdaFunction{⌊ᵗ⌋-split'}\AgdaSpace{}%
\AgdaBound{ℱ}\AgdaSymbol{@(}\AgdaInductiveConstructor{suc}\AgdaSpace{}%
\AgdaSymbol{\AgdaUnderscore{})}\AgdaSpace{}%
\AgdaBound{s}\AgdaSpace{}%
\AgdaSymbol{(}\AgdaInductiveConstructor{𝑠𝑘𝑖𝑝}\AgdaSpace{}%
\AgdaOperator{\AgdaInductiveConstructor{;}}\AgdaSymbol{)}\AgdaSpace{}%
\AgdaBound{Q₂}\AgdaSpace{}%
\AgdaBound{t₁₂}\AgdaSpace{}%
\AgdaSymbol{=}\<%
\\
\>[2][@{}l@{\AgdaIndent{0}}]%
\>[10]\AgdaSymbol{(}\AgdaInductiveConstructor{Any.just}\AgdaSpace{}%
\AgdaInductiveConstructor{tt}\AgdaSymbol{)}\AgdaSpace{}%
\AgdaOperator{\AgdaInductiveConstructor{,}}\AgdaSpace{}%
\AgdaBound{ℱ}\AgdaSpace{}%
\AgdaOperator{\AgdaInductiveConstructor{,}}\AgdaSpace{}%
\AgdaInductiveConstructor{≤with}\AgdaSpace{}%
\AgdaSymbol{(}\AgdaFunction{+-comm}\AgdaSpace{}%
\AgdaBound{ℱ}\AgdaSpace{}%
\AgdaNumber{0}\AgdaSymbol{)}\AgdaSpace{}%
\AgdaOperator{\AgdaInductiveConstructor{,}}\AgdaSpace{}%
\AgdaBound{t₁₂}\AgdaSpace{}%
\AgdaOperator{\AgdaInductiveConstructor{,}}\AgdaSpace{}%
\AgdaInductiveConstructor{refl}\<%
\\
%
\>[2]\AgdaComment{-- Q₁ = 𝑠𝑘𝑖𝑝 : Q₁ '}\<%
\\
%
\>[2]\AgdaFunction{⌊ᵗ⌋-split'}\AgdaSpace{}%
\AgdaBound{ℱ}\AgdaSymbol{@}\AgdaNumber{0}%
\>[23]\AgdaBound{s}\AgdaSpace{}%
\AgdaSymbol{(}\AgdaInductiveConstructor{𝑠𝑘𝑖𝑝}\AgdaSpace{}%
\AgdaOperator{\AgdaInductiveConstructor{;}}\AgdaSpace{}%
\AgdaBound{Q₁'}\AgdaSymbol{)}\AgdaSpace{}%
\AgdaSymbol{=}\AgdaSpace{}%
\AgdaFunction{⌊ᵗ⌋-split'}\AgdaSpace{}%
\AgdaBound{ℱ}\AgdaSpace{}%
\AgdaBound{s}\AgdaSpace{}%
\AgdaBound{Q₁'}\<%
\\
%
\>[2]\AgdaFunction{⌊ᵗ⌋-split'}\AgdaSpace{}%
\AgdaBound{ℱ}\AgdaSymbol{@(}\AgdaInductiveConstructor{suc}\AgdaSpace{}%
\AgdaSymbol{\AgdaUnderscore{})}\AgdaSpace{}%
\AgdaBound{s}\AgdaSpace{}%
\AgdaSymbol{(}\AgdaInductiveConstructor{𝑠𝑘𝑖𝑝}\AgdaSpace{}%
\AgdaOperator{\AgdaInductiveConstructor{;}}\AgdaSpace{}%
\AgdaBound{Q₁'}\AgdaSymbol{)}\AgdaSpace{}%
\AgdaSymbol{=}\AgdaSpace{}%
\AgdaFunction{⌊ᵗ⌋-split'}\AgdaSpace{}%
\AgdaBound{ℱ}\AgdaSpace{}%
\AgdaBound{s}\AgdaSpace{}%
\AgdaBound{Q₁'}\<%
\\
%
\\[\AgdaEmptyExtraSkip]%
%
\\[\AgdaEmptyExtraSkip]%
%
\>[2]\AgdaComment{--------------------------------------------------------------------}\<%
\\
%
\>[2]\AgdaComment{-- Most interesting inductive case: 𝔴𝔥𝔦𝔩𝔢 followed by Q₁' 𝔱𝔥𝔢𝔫 Q₂.}\<%
\\
%
\>[2]\AgdaComment{-- All other cases follow a similar recursive mechanism}\<%
\\
%
\>[2]\AgdaFunction{⌊ᵗ⌋-split'}\AgdaSpace{}%
\AgdaSymbol{(}\AgdaInductiveConstructor{suc}\AgdaSpace{}%
\AgdaBound{ℱ}\AgdaSymbol{)}\AgdaSpace{}%
\AgdaBound{s}\AgdaSpace{}%
\AgdaBound{Q₁}\AgdaSymbol{@((}\AgdaOperator{\AgdaInductiveConstructor{𝔴𝔥𝔦𝔩𝔢}}\AgdaSpace{}%
\AgdaBound{exp}\AgdaSpace{}%
\AgdaOperator{\AgdaInductiveConstructor{𝒹ℴ}}\AgdaSpace{}%
\AgdaBound{c}\AgdaSymbol{)}\AgdaSpace{}%
\AgdaOperator{\AgdaInductiveConstructor{;}}\AgdaSpace{}%
\AgdaBound{Q₁'}\AgdaSymbol{)}\AgdaSpace{}%
\AgdaBound{Q₂}\AgdaSpace{}%
\AgdaBound{t₁₂}\AgdaSpace{}%
\AgdaSymbol{=}\AgdaSpace{}%
\AgdaFunction{go}\<%
\\
\>[2][@{}l@{\AgdaIndent{0}}]%
\>[4]\AgdaKeyword{where}\<%
\\
%
\>[4]\AgdaFunction{go}\AgdaSpace{}%
\AgdaSymbol{:}%
\>[1098I]\AgdaRecord{Σ}\AgdaSpace{}%
\AgdaOperator{\AgdaFunction{⌊ᵗ}}\AgdaSpace{}%
\AgdaInductiveConstructor{suc}\AgdaSpace{}%
\AgdaBound{ℱ}\AgdaSpace{}%
\AgdaOperator{\AgdaFunction{⸴}}\AgdaSpace{}%
\AgdaBound{Q₁}\AgdaSpace{}%
\AgdaOperator{\AgdaFunction{⸴}}\AgdaSpace{}%
\AgdaBound{s}\AgdaSpace{}%
\AgdaOperator{\AgdaFunction{ᵗ⌋}}\AgdaSpace{}%
\AgdaSymbol{(λ}\AgdaSpace{}%
\AgdaBound{t₁}\AgdaSpace{}%
\AgdaSymbol{→}\AgdaSpace{}%
\AgdaRecord{Σ}\AgdaSpace{}%
\AgdaDatatype{ℕ}\AgdaSpace{}%
\AgdaSymbol{(λ}\AgdaSpace{}%
\AgdaBound{ℱ'}\AgdaSpace{}%
\AgdaSymbol{→}\AgdaSpace{}%
\AgdaBound{ℱ'}\AgdaSpace{}%
\AgdaOperator{\AgdaRecord{≤''}}\AgdaSpace{}%
\AgdaInductiveConstructor{suc}\AgdaSpace{}%
\AgdaBound{ℱ}\AgdaSpace{}%
\AgdaOperator{\AgdaFunction{×}}\<%
\\
\>[.][@{}l@{}]\<[1098I]%
\>[9]\AgdaRecord{Σ}\AgdaSpace{}%
\AgdaOperator{\AgdaFunction{⌊ᵗ}}\AgdaSpace{}%
\AgdaBound{ℱ'}\AgdaSpace{}%
\AgdaOperator{\AgdaFunction{⸴}}\AgdaSpace{}%
\AgdaBound{Q₂}\AgdaSpace{}%
\AgdaOperator{\AgdaFunction{⸴}}\AgdaSpace{}%
\AgdaFunction{″}\AgdaSpace{}%
\AgdaBound{t₁}\AgdaSpace{}%
\AgdaOperator{\AgdaFunction{ᵗ⌋}}\AgdaSpace{}%
\AgdaSymbol{(λ}\AgdaSpace{}%
\AgdaBound{t₂}\AgdaSpace{}%
\AgdaSymbol{→}\AgdaSpace{}%
\AgdaFunction{″}\AgdaSpace{}%
\AgdaBound{t₂}\AgdaSpace{}%
\AgdaOperator{\AgdaDatatype{≡}}\AgdaSpace{}%
\AgdaFunction{″}\AgdaSpace{}%
\AgdaBound{t₁₂}\AgdaSpace{}%
\AgdaSymbol{)))}\<%
\\
%
\>[4]\AgdaFunction{go}\AgdaSpace{}%
\AgdaKeyword{with}\AgdaSpace{}%
\AgdaFunction{evalExp}\AgdaSpace{}%
\AgdaBound{exp}\AgdaSpace{}%
\AgdaBound{s}\<%
\\
%
\>[4]\AgdaFunction{go}\AgdaSpace{}%
\AgdaSymbol{|}\AgdaSpace{}%
\AgdaBound{f}\AgdaSymbol{@(}\AgdaInductiveConstructor{just}\AgdaSpace{}%
\AgdaSymbol{\AgdaUnderscore{})}\AgdaSpace{}%
\AgdaKeyword{with}\AgdaSpace{}%
\AgdaFunction{toTruthValue}\AgdaSpace{}%
\AgdaSymbol{\{}\AgdaBound{f}\AgdaSymbol{\}}\AgdaSpace{}%
\AgdaSymbol{(}\AgdaInductiveConstructor{Any.just}\AgdaSpace{}%
\AgdaInductiveConstructor{tt}\AgdaSymbol{)}\<%
\\
%
\>[4]\AgdaComment{-- if false ---------------------------------------}\<%
\\
%
\>[4]\AgdaFunction{go}\AgdaSpace{}%
\AgdaSymbol{|}\AgdaSpace{}%
\AgdaBound{f}\AgdaSymbol{@(}\AgdaInductiveConstructor{just}\AgdaSpace{}%
\AgdaSymbol{\AgdaUnderscore{})}\AgdaSpace{}%
\AgdaSymbol{|}\AgdaSpace{}%
\AgdaInductiveConstructor{false}\AgdaSpace{}%
\AgdaKeyword{with}\AgdaSpace{}%
\AgdaFunction{⌊ᵗ⌋-split'}\AgdaSpace{}%
\AgdaBound{ℱ}\AgdaSpace{}%
\AgdaBound{s}\AgdaSpace{}%
\AgdaBound{Q₁'}\AgdaSpace{}%
\AgdaBound{Q₂}\AgdaSpace{}%
\AgdaBound{t₁₂}\<%
\\
%
\>[4]\AgdaFunction{go}\AgdaSpace{}%
\AgdaSymbol{|}\AgdaSpace{}%
\AgdaInductiveConstructor{just}\AgdaSpace{}%
\AgdaSymbol{\AgdaUnderscore{}}%
\>[20]\AgdaSymbol{|}\AgdaSpace{}%
\AgdaInductiveConstructor{false}%
\>[1165I]\AgdaSymbol{|}%
\>[31]\AgdaBound{t₁}\AgdaSpace{}%
\AgdaOperator{\AgdaInductiveConstructor{,}}\AgdaSpace{}%
\AgdaBound{ℱ'}\AgdaSpace{}%
\AgdaOperator{\AgdaInductiveConstructor{,}}\AgdaSpace{}%
\AgdaBound{lt}%
\>[50]\AgdaOperator{\AgdaInductiveConstructor{,}}\AgdaSpace{}%
\AgdaBound{t₂}\AgdaSpace{}%
\AgdaOperator{\AgdaInductiveConstructor{,}}\AgdaSpace{}%
\AgdaBound{Δ}\<%
\\
\>[.][@{}l@{}]\<[1165I]%
\>[28]\AgdaSymbol{=}%
\>[31]\AgdaBound{t₁}\AgdaSpace{}%
\AgdaOperator{\AgdaInductiveConstructor{,}}\AgdaSpace{}%
\AgdaBound{ℱ'}\AgdaSpace{}%
\AgdaOperator{\AgdaInductiveConstructor{,}}\AgdaSpace{}%
\AgdaFunction{suc≤''}\AgdaSpace{}%
\AgdaBound{lt}\AgdaSpace{}%
\AgdaOperator{\AgdaInductiveConstructor{,}}\AgdaSpace{}%
\AgdaBound{t₂}\AgdaSpace{}%
\AgdaOperator{\AgdaInductiveConstructor{,}}\AgdaSpace{}%
\AgdaBound{Δ}\<%
\\
%
\>[4]\AgdaComment{-- if true ---------------------------------------- }\<%
\\
%
\>[4]\AgdaFunction{go}%
\>[1182I]\AgdaSymbol{|}%
\>[1183I]\AgdaBound{f}\AgdaSymbol{@(}\AgdaInductiveConstructor{just}\AgdaSpace{}%
\AgdaSymbol{\AgdaUnderscore{})}\AgdaSpace{}%
\AgdaSymbol{|}\AgdaSpace{}%
\AgdaInductiveConstructor{true}\AgdaSpace{}%
\AgdaKeyword{rewrite}\<%
\\
\>[.][@{}l@{}]\<[1183I]%
\>[9]\AgdaFunction{𝔱𝔥𝔢𝔫-comm}\AgdaSpace{}%
\AgdaSymbol{((}\AgdaOperator{\AgdaInductiveConstructor{𝔴𝔥𝔦𝔩𝔢}}\AgdaSpace{}%
\AgdaBound{exp}\AgdaSpace{}%
\AgdaOperator{\AgdaInductiveConstructor{𝒹ℴ}}\AgdaSpace{}%
\AgdaBound{c}\AgdaSymbol{)}\AgdaSpace{}%
\AgdaOperator{\AgdaInductiveConstructor{;}}\AgdaSymbol{)}\AgdaSpace{}%
\AgdaBound{Q₁'}\AgdaSpace{}%
\AgdaBound{Q₂}\<%
\\
\>[.][@{}l@{}]\<[1182I]%
\>[7]\AgdaSymbol{|}%
\>[1195I]\AgdaFunction{𝔱𝔥𝔢𝔫-comm}\AgdaSpace{}%
\AgdaBound{c}\AgdaSpace{}%
\AgdaSymbol{((}\AgdaOperator{\AgdaInductiveConstructor{𝔴𝔥𝔦𝔩𝔢}}\AgdaSpace{}%
\AgdaBound{exp}\AgdaSpace{}%
\AgdaOperator{\AgdaInductiveConstructor{𝒹ℴ}}\AgdaSpace{}%
\AgdaBound{c}\AgdaSymbol{)}\AgdaSpace{}%
\AgdaOperator{\AgdaInductiveConstructor{;}}%
\>[42]\AgdaBound{Q₁'}\AgdaSpace{}%
\AgdaSymbol{)}\AgdaSpace{}%
\AgdaBound{Q₂}\AgdaSpace{}%
\AgdaKeyword{with}\<%
\\
\>[.][@{}l@{}]\<[1195I]%
\>[9]\AgdaFunction{⌊ᵗ⌋-split'}\AgdaSpace{}%
\AgdaBound{ℱ}\AgdaSpace{}%
\AgdaBound{s}\AgdaSpace{}%
\AgdaSymbol{(}\AgdaBound{c}\AgdaSpace{}%
\AgdaOperator{\AgdaFunction{𝔱𝔥𝔢𝔫}}\AgdaSpace{}%
\AgdaOperator{\AgdaInductiveConstructor{𝔴𝔥𝔦𝔩𝔢}}\AgdaSpace{}%
\AgdaBound{exp}\AgdaSpace{}%
\AgdaOperator{\AgdaInductiveConstructor{𝒹ℴ}}\AgdaSpace{}%
\AgdaBound{c}\AgdaSpace{}%
\AgdaOperator{\AgdaInductiveConstructor{;}}\AgdaSpace{}%
\AgdaBound{Q₁'}\AgdaSymbol{)}\AgdaSpace{}%
\AgdaBound{Q₂}\AgdaSpace{}%
\AgdaBound{t₁₂}\<%
\\
%
\>[4]\AgdaFunction{go}\AgdaSpace{}%
\AgdaSymbol{|}\AgdaSpace{}%
\AgdaBound{f}\AgdaSymbol{@(}\AgdaInductiveConstructor{just}\AgdaSpace{}%
\AgdaSymbol{\AgdaUnderscore{})}\AgdaSpace{}%
\AgdaSymbol{|}\AgdaSpace{}%
\AgdaInductiveConstructor{true}%
\>[1222I]\AgdaSymbol{|}%
\>[30]\AgdaBound{t₁}\AgdaSpace{}%
\AgdaOperator{\AgdaInductiveConstructor{,}}\AgdaSpace{}%
\AgdaBound{ℱ'}\AgdaSpace{}%
\AgdaOperator{\AgdaInductiveConstructor{,}}\AgdaSpace{}%
\AgdaBound{lt}%
\>[49]\AgdaOperator{\AgdaInductiveConstructor{,}}\AgdaSpace{}%
\AgdaBound{t₂}\AgdaSpace{}%
\AgdaOperator{\AgdaInductiveConstructor{,}}\AgdaSpace{}%
\AgdaBound{Δ}\<%
\\
\>[.][@{}l@{}]\<[1222I]%
\>[27]\AgdaSymbol{=}%
\>[30]\AgdaBound{t₁}\AgdaSpace{}%
\AgdaOperator{\AgdaInductiveConstructor{,}}\AgdaSpace{}%
\AgdaBound{ℱ'}\AgdaSpace{}%
\AgdaOperator{\AgdaInductiveConstructor{,}}\AgdaSpace{}%
\AgdaFunction{suc≤''}\AgdaSpace{}%
\AgdaBound{lt}\AgdaSpace{}%
\AgdaOperator{\AgdaInductiveConstructor{,}}\AgdaSpace{}%
\AgdaBound{t₂}\AgdaSpace{}%
\AgdaOperator{\AgdaInductiveConstructor{,}}\AgdaSpace{}%
\AgdaBound{Δ}\<%
\end{code}}
  {\centering \hfill \Huge{\vdots} \hfill }
\end{figure}

\begin{figure}\ContinuedFloat
  \caption{[t]-split cont. \\
   n.b. some cases have been omitted but none that vary from the general pattern here. }
  \footnotesize
  {\centering \hfill \Huge{\vdots} \hfill }
  {\centering \begin{code}
  \>[2]\AgdaComment{--------------------------------------------------------------------}\<%
\\
%
\>[2]\AgdaComment{-- Q₁ = if then else                            }\<%
\\
%
\>[2]\AgdaFunction{⌊ᵗ⌋-split'}\AgdaSpace{}%
\AgdaSymbol{(}\AgdaInductiveConstructor{suc}\AgdaSpace{}%
\AgdaBound{ℱ}\AgdaSymbol{)}\AgdaSpace{}%
\AgdaBound{s}\AgdaSpace{}%
\AgdaBound{Q₁}\AgdaSymbol{@((}\AgdaOperator{\AgdaInductiveConstructor{𝔦𝔣}}\AgdaSpace{}%
\AgdaBound{exp}\AgdaSpace{}%
\AgdaOperator{\AgdaInductiveConstructor{𝔱𝔥𝔢𝔫}}\AgdaSpace{}%
\AgdaBound{c₁}\AgdaSpace{}%
\AgdaOperator{\AgdaInductiveConstructor{𝔢𝔩𝔰𝔢}}\AgdaSpace{}%
\AgdaBound{c₂}\AgdaSymbol{)}\AgdaSpace{}%
\AgdaOperator{\AgdaInductiveConstructor{;}}\AgdaSymbol{)}\AgdaSpace{}%
\AgdaBound{Q₂}\AgdaSpace{}%
\AgdaBound{t₁₂}\<%
\\
\>[2][@{}l@{\AgdaIndent{0}}]%
\>[4]\AgdaSymbol{=}\AgdaSpace{}%
\AgdaFunction{go}\<%
\\
%
\>[4]\AgdaKeyword{where}\<%
\\
%
\>[4]\AgdaFunction{go}\AgdaSpace{}%
\AgdaSymbol{:}%
\>[1523I]\AgdaRecord{Σ}\AgdaSpace{}%
\AgdaOperator{\AgdaFunction{⌊ᵗ}}\AgdaSpace{}%
\AgdaInductiveConstructor{suc}\AgdaSpace{}%
\AgdaBound{ℱ}\AgdaSpace{}%
\AgdaOperator{\AgdaFunction{⸴}}\AgdaSpace{}%
\AgdaBound{Q₁}\AgdaSpace{}%
\AgdaOperator{\AgdaFunction{⸴}}\AgdaSpace{}%
\AgdaBound{s}\AgdaSpace{}%
\AgdaOperator{\AgdaFunction{ᵗ⌋}}\AgdaSpace{}%
\AgdaSymbol{(λ}\AgdaSpace{}%
\AgdaBound{t₁}\AgdaSpace{}%
\AgdaSymbol{→}\AgdaSpace{}%
\AgdaRecord{Σ}\AgdaSpace{}%
\AgdaDatatype{ℕ}\AgdaSpace{}%
\AgdaSymbol{(λ}\AgdaSpace{}%
\AgdaBound{ℱ'}\AgdaSpace{}%
\AgdaSymbol{→}\AgdaSpace{}%
\AgdaBound{ℱ'}\AgdaSpace{}%
\AgdaOperator{\AgdaRecord{≤''}}\AgdaSpace{}%
\AgdaInductiveConstructor{suc}\AgdaSpace{}%
\AgdaBound{ℱ}\AgdaSpace{}%
\AgdaOperator{\AgdaFunction{×}}\<%
\\
\>[.][@{}l@{}]\<[1523I]%
\>[9]\AgdaRecord{Σ}\AgdaSpace{}%
\AgdaOperator{\AgdaFunction{⌊ᵗ}}\AgdaSpace{}%
\AgdaBound{ℱ'}\AgdaSpace{}%
\AgdaOperator{\AgdaFunction{⸴}}\AgdaSpace{}%
\AgdaBound{Q₂}\AgdaSpace{}%
\AgdaOperator{\AgdaFunction{⸴}}\AgdaSpace{}%
\AgdaFunction{″}\AgdaSpace{}%
\AgdaBound{t₁}\AgdaSpace{}%
\AgdaOperator{\AgdaFunction{ᵗ⌋}}\AgdaSpace{}%
\AgdaSymbol{(λ}\AgdaSpace{}%
\AgdaBound{t₂}\AgdaSpace{}%
\AgdaSymbol{→}\AgdaSpace{}%
\AgdaFunction{″}\AgdaSpace{}%
\AgdaBound{t₂}\AgdaSpace{}%
\AgdaOperator{\AgdaDatatype{≡}}\AgdaSpace{}%
\AgdaFunction{″}\AgdaSpace{}%
\AgdaBound{t₁₂}\AgdaSpace{}%
\AgdaSymbol{)))}\<%
\\
%
\>[4]\AgdaFunction{go}\AgdaSpace{}%
\AgdaKeyword{with}\AgdaSpace{}%
\AgdaFunction{evalExp}\AgdaSpace{}%
\AgdaBound{exp}\AgdaSpace{}%
\AgdaBound{s}\<%
\\
%
\>[4]\AgdaFunction{go}\AgdaSpace{}%
\AgdaSymbol{|}\AgdaSpace{}%
\AgdaBound{f}\AgdaSymbol{@(}\AgdaInductiveConstructor{just}\AgdaSpace{}%
\AgdaSymbol{\AgdaUnderscore{})}\AgdaSpace{}%
\AgdaKeyword{with}\AgdaSpace{}%
\AgdaFunction{toTruthValue}\AgdaSpace{}%
\AgdaSymbol{\{}\AgdaBound{f}\AgdaSymbol{\}}\AgdaSpace{}%
\AgdaSymbol{(}\AgdaInductiveConstructor{Any.just}\AgdaSpace{}%
\AgdaInductiveConstructor{tt}\AgdaSymbol{)}\<%
\\
%
\>[4]\AgdaComment{-- if false ---------------------------------------}\<%
\\
%
\>[4]\AgdaFunction{go}\AgdaSpace{}%
\AgdaSymbol{|}\AgdaSpace{}%
\AgdaBound{f}\AgdaSymbol{@(}\AgdaInductiveConstructor{just}\AgdaSpace{}%
\AgdaSymbol{\AgdaUnderscore{})}\AgdaSpace{}%
\AgdaSymbol{|}\AgdaSpace{}%
\AgdaInductiveConstructor{false}\AgdaSpace{}%
\AgdaKeyword{with}\AgdaSpace{}%
\AgdaFunction{⌊ᵗ⌋-split'}\AgdaSpace{}%
\AgdaBound{ℱ}\AgdaSpace{}%
\AgdaBound{s}\AgdaSpace{}%
\AgdaBound{c₂}\AgdaSpace{}%
\AgdaBound{Q₂}\AgdaSpace{}%
\AgdaBound{t₁₂}\<%
\\
%
\>[4]\AgdaFunction{go}\AgdaSpace{}%
\AgdaSymbol{|}\AgdaSpace{}%
\AgdaBound{f}\AgdaSymbol{@(}\AgdaInductiveConstructor{just}\AgdaSpace{}%
\AgdaSymbol{\AgdaUnderscore{})}\AgdaSpace{}%
\AgdaSymbol{|}\AgdaSpace{}%
\AgdaInductiveConstructor{false}%
\>[1591I]\AgdaSymbol{|}%
\>[31]\AgdaBound{t₁}\AgdaSpace{}%
\AgdaOperator{\AgdaInductiveConstructor{,}}\AgdaSpace{}%
\AgdaBound{ℱ'}\AgdaSpace{}%
\AgdaOperator{\AgdaInductiveConstructor{,}}\AgdaSpace{}%
\AgdaBound{lt}%
\>[50]\AgdaOperator{\AgdaInductiveConstructor{,}}\AgdaSpace{}%
\AgdaBound{t₂}\AgdaSpace{}%
\AgdaOperator{\AgdaInductiveConstructor{,}}\AgdaSpace{}%
\AgdaBound{Δ}\<%
\\
\>[.][@{}l@{}]\<[1591I]%
\>[28]\AgdaSymbol{=}%
\>[31]\AgdaBound{t₁}\AgdaSpace{}%
\AgdaOperator{\AgdaInductiveConstructor{,}}\AgdaSpace{}%
\AgdaBound{ℱ'}\AgdaSpace{}%
\AgdaOperator{\AgdaInductiveConstructor{,}}\AgdaSpace{}%
\AgdaFunction{suc≤''}\AgdaSpace{}%
\AgdaBound{lt}\AgdaSpace{}%
\AgdaOperator{\AgdaInductiveConstructor{,}}\AgdaSpace{}%
\AgdaBound{t₂}\AgdaSpace{}%
\AgdaOperator{\AgdaInductiveConstructor{,}}\AgdaSpace{}%
\AgdaBound{Δ}\<%
\\
%
\>[4]\AgdaComment{-- if true ---------------------------------------- }\<%
\\
%
\>[4]\AgdaFunction{go}\AgdaSpace{}%
\AgdaSymbol{|}\AgdaSpace{}%
\AgdaBound{f}\AgdaSymbol{@(}\AgdaInductiveConstructor{just}\AgdaSpace{}%
\AgdaSymbol{\AgdaUnderscore{})}\AgdaSpace{}%
\AgdaSymbol{|}\AgdaSpace{}%
\AgdaInductiveConstructor{true}%
\>[28]\AgdaKeyword{with}\AgdaSpace{}%
\AgdaFunction{⌊ᵗ⌋-split'}\AgdaSpace{}%
\AgdaBound{ℱ}\AgdaSpace{}%
\AgdaBound{s}\AgdaSpace{}%
\AgdaBound{c₁}\AgdaSpace{}%
\AgdaBound{Q₂}\AgdaSpace{}%
\AgdaBound{t₁₂}\<%
\\
%
\>[4]\AgdaFunction{go}\AgdaSpace{}%
\AgdaSymbol{|}\AgdaSpace{}%
\AgdaBound{f}\AgdaSymbol{@(}\AgdaInductiveConstructor{just}\AgdaSpace{}%
\AgdaSymbol{\AgdaUnderscore{})}\AgdaSpace{}%
\AgdaSymbol{|}\AgdaSpace{}%
\AgdaInductiveConstructor{true}%
\>[28]\AgdaSymbol{|}%
\>[31]\AgdaBound{t₁}\AgdaSpace{}%
\AgdaOperator{\AgdaInductiveConstructor{,}}\AgdaSpace{}%
\AgdaBound{ℱ'}\AgdaSpace{}%
\AgdaOperator{\AgdaInductiveConstructor{,}}\AgdaSpace{}%
\AgdaBound{lt}%
\>[50]\AgdaOperator{\AgdaInductiveConstructor{,}}\AgdaSpace{}%
\AgdaBound{t₂}\AgdaSpace{}%
\AgdaOperator{\AgdaInductiveConstructor{,}}\AgdaSpace{}%
\AgdaBound{Δ}\<%
\\
%
\>[28]\AgdaSymbol{=}%
\>[31]\AgdaBound{t₁}\AgdaSpace{}%
\AgdaOperator{\AgdaInductiveConstructor{,}}\AgdaSpace{}%
\AgdaBound{ℱ'}\AgdaSpace{}%
\AgdaOperator{\AgdaInductiveConstructor{,}}\AgdaSpace{}%
\AgdaFunction{suc≤''}\AgdaSpace{}%
\AgdaBound{lt}\AgdaSpace{}%
\AgdaOperator{\AgdaInductiveConstructor{,}}\AgdaSpace{}%
\AgdaBound{t₂}\AgdaSpace{}%
\AgdaOperator{\AgdaInductiveConstructor{,}}\AgdaSpace{}%
\AgdaBound{Δ}\<%
\\
%
\>[2]\AgdaComment{--------------------------------------------------------------------}\<%
\\
%
\>[2]\AgdaComment{-- Q₁ = x := exp ; Q₁'}\<%
\\
%
\>[2]\AgdaFunction{⌊ᵗ⌋-split'}\AgdaSpace{}%
\AgdaSymbol{(}\AgdaInductiveConstructor{suc}\AgdaSpace{}%
\AgdaBound{ℱ}\AgdaSymbol{)}\AgdaSpace{}%
\AgdaBound{s}\AgdaSpace{}%
\AgdaBound{Q₁}\AgdaSymbol{@(}\AgdaSpace{}%
\AgdaBound{id}\AgdaSpace{}%
\AgdaOperator{\AgdaInductiveConstructor{:=}}\AgdaSpace{}%
\AgdaBound{exp}%
\>[39]\AgdaOperator{\AgdaInductiveConstructor{;}}\AgdaSpace{}%
\AgdaBound{Q₁'}\AgdaSymbol{)}\AgdaSpace{}%
\AgdaBound{Q₂}\AgdaSpace{}%
\AgdaBound{t₁₂}\AgdaSpace{}%
\AgdaSymbol{=}\AgdaSpace{}%
\AgdaFunction{go}\<%
\\
\>[2][@{}l@{\AgdaIndent{0}}]%
\>[4]\AgdaKeyword{where}\<%
\\
%
\>[4]\AgdaFunction{go}\AgdaSpace{}%
\AgdaSymbol{:}%
\>[1653I]\AgdaRecord{Σ}\AgdaSpace{}%
\AgdaOperator{\AgdaFunction{⌊ᵗ}}\AgdaSpace{}%
\AgdaInductiveConstructor{suc}\AgdaSpace{}%
\AgdaBound{ℱ}\AgdaSpace{}%
\AgdaOperator{\AgdaFunction{⸴}}\AgdaSpace{}%
\AgdaBound{Q₁}\AgdaSpace{}%
\AgdaOperator{\AgdaFunction{⸴}}\AgdaSpace{}%
\AgdaBound{s}\AgdaSpace{}%
\AgdaOperator{\AgdaFunction{ᵗ⌋}}\AgdaSpace{}%
\AgdaSymbol{(λ}\AgdaSpace{}%
\AgdaBound{t₁}\AgdaSpace{}%
\AgdaSymbol{→}\AgdaSpace{}%
\AgdaRecord{Σ}\AgdaSpace{}%
\AgdaDatatype{ℕ}\AgdaSpace{}%
\AgdaSymbol{(λ}\AgdaSpace{}%
\AgdaBound{ℱ'}\AgdaSpace{}%
\AgdaSymbol{→}\AgdaSpace{}%
\AgdaBound{ℱ'}\AgdaSpace{}%
\AgdaOperator{\AgdaRecord{≤''}}\AgdaSpace{}%
\AgdaInductiveConstructor{suc}\AgdaSpace{}%
\AgdaBound{ℱ}\AgdaSpace{}%
\AgdaOperator{\AgdaFunction{×}}\<%
\\
\>[.][@{}l@{}]\<[1653I]%
\>[9]\AgdaRecord{Σ}\AgdaSpace{}%
\AgdaOperator{\AgdaFunction{⌊ᵗ}}\AgdaSpace{}%
\AgdaBound{ℱ'}\AgdaSpace{}%
\AgdaOperator{\AgdaFunction{⸴}}\AgdaSpace{}%
\AgdaBound{Q₂}\AgdaSpace{}%
\AgdaOperator{\AgdaFunction{⸴}}\AgdaSpace{}%
\AgdaFunction{″}\AgdaSpace{}%
\AgdaBound{t₁}\AgdaSpace{}%
\AgdaOperator{\AgdaFunction{ᵗ⌋}}\AgdaSpace{}%
\AgdaSymbol{(λ}\AgdaSpace{}%
\AgdaBound{t₂}\AgdaSpace{}%
\AgdaSymbol{→}\AgdaSpace{}%
\AgdaFunction{″}\AgdaSpace{}%
\AgdaBound{t₂}\AgdaSpace{}%
\AgdaOperator{\AgdaDatatype{≡}}\AgdaSpace{}%
\AgdaFunction{″}\AgdaSpace{}%
\AgdaBound{t₁₂}\AgdaSpace{}%
\AgdaSymbol{)))}\<%
\\
%
\>[4]\AgdaFunction{go}\AgdaSpace{}%
\AgdaKeyword{with}\AgdaSpace{}%
\AgdaFunction{evalExp}\AgdaSpace{}%
\AgdaBound{exp}\AgdaSpace{}%
\AgdaBound{s}\<%
\\
%
\>[4]\AgdaFunction{go}%
\>[1696I]\AgdaSymbol{|}\AgdaSpace{}%
\AgdaBound{f}\AgdaSymbol{@(}\AgdaInductiveConstructor{just}\AgdaSpace{}%
\AgdaBound{v}\AgdaSymbol{)}\<%
\\
\>[.][@{}l@{}]\<[1696I]%
\>[7]\AgdaKeyword{with}\AgdaSpace{}%
\AgdaFunction{⌊ᵗ⌋-split'}\AgdaSpace{}%
\AgdaBound{ℱ}\AgdaSpace{}%
\AgdaSymbol{(}\AgdaField{updateState}\AgdaSpace{}%
\AgdaBound{id}\AgdaSpace{}%
\AgdaBound{v}\AgdaSpace{}%
\AgdaBound{s}\AgdaSymbol{)}\AgdaSpace{}%
\AgdaBound{Q₁'}\AgdaSpace{}%
\AgdaBound{Q₂}\AgdaSpace{}%
\AgdaBound{t₁₂}\<%
\\
%
\>[4]\AgdaFunction{go}\AgdaSpace{}%
\AgdaSymbol{|}\AgdaSpace{}%
\AgdaBound{f}\AgdaSymbol{@(}\AgdaInductiveConstructor{just}\AgdaSpace{}%
\AgdaBound{v}\AgdaSymbol{)}%
\>[1711I]\AgdaSymbol{|}%
\>[23]\AgdaBound{t₁}\AgdaSpace{}%
\AgdaOperator{\AgdaInductiveConstructor{,}}\AgdaSpace{}%
\AgdaBound{ℱ'}\AgdaSpace{}%
\AgdaOperator{\AgdaInductiveConstructor{,}}\AgdaSpace{}%
\AgdaBound{lt}%
\>[42]\AgdaOperator{\AgdaInductiveConstructor{,}}\AgdaSpace{}%
\AgdaBound{t₂}\AgdaSpace{}%
\AgdaOperator{\AgdaInductiveConstructor{,}}\AgdaSpace{}%
\AgdaBound{Δ}\<%
\\
\>[.][@{}l@{}]\<[1711I]%
\>[20]\AgdaSymbol{=}%
\>[23]\AgdaBound{t₁}\AgdaSpace{}%
\AgdaOperator{\AgdaInductiveConstructor{,}}\AgdaSpace{}%
\AgdaBound{ℱ'}\AgdaSpace{}%
\AgdaOperator{\AgdaInductiveConstructor{,}}\AgdaSpace{}%
\AgdaFunction{suc≤''}\AgdaSpace{}%
\AgdaBound{lt}\AgdaSpace{}%
\AgdaOperator{\AgdaInductiveConstructor{,}}\AgdaSpace{}%
\AgdaBound{t₂}\AgdaSpace{}%
\AgdaOperator{\AgdaInductiveConstructor{,}}\AgdaSpace{}%
\AgdaBound{Δ}\<%
\\
%
\>[2]\AgdaComment{--------------------------------------------------------------------}\<%
\\
%
\>[2]\AgdaComment{-- Q₁ = id := exp ;}\<%
\\
%
\>[2]\AgdaFunction{⌊ᵗ⌋-split'}\AgdaSpace{}%
\AgdaSymbol{(}\AgdaInductiveConstructor{suc}\AgdaSpace{}%
\AgdaBound{ℱ}\AgdaSymbol{)}\AgdaSpace{}%
\AgdaBound{s}\AgdaSpace{}%
\AgdaBound{Q₁}\AgdaSymbol{@(}\AgdaSpace{}%
\AgdaBound{id}\AgdaSpace{}%
\AgdaOperator{\AgdaInductiveConstructor{:=}}\AgdaSpace{}%
\AgdaBound{exp}\AgdaSpace{}%
\AgdaOperator{\AgdaInductiveConstructor{;}}\AgdaSymbol{)}\AgdaSpace{}%
\AgdaBound{Q₂}\AgdaSpace{}%
\AgdaBound{t₁₂}\AgdaSpace{}%
\AgdaSymbol{=}\AgdaSpace{}%
\AgdaFunction{go}\<%
\\
\>[2][@{}l@{\AgdaIndent{0}}]%
\>[4]\AgdaKeyword{where}\<%
\\
%
\>[4]\AgdaFunction{go}\AgdaSpace{}%
\AgdaSymbol{:}%
\>[1741I]\AgdaRecord{Σ}\AgdaSpace{}%
\AgdaOperator{\AgdaFunction{⌊ᵗ}}\AgdaSpace{}%
\AgdaInductiveConstructor{suc}\AgdaSpace{}%
\AgdaBound{ℱ}\AgdaSpace{}%
\AgdaOperator{\AgdaFunction{⸴}}\AgdaSpace{}%
\AgdaBound{Q₁}\AgdaSpace{}%
\AgdaOperator{\AgdaFunction{⸴}}\AgdaSpace{}%
\AgdaBound{s}\AgdaSpace{}%
\AgdaOperator{\AgdaFunction{ᵗ⌋}}\AgdaSpace{}%
\AgdaSymbol{(λ}\AgdaSpace{}%
\AgdaBound{t₁}\AgdaSpace{}%
\AgdaSymbol{→}\AgdaSpace{}%
\AgdaRecord{Σ}\AgdaSpace{}%
\AgdaDatatype{ℕ}\AgdaSpace{}%
\AgdaSymbol{(λ}\AgdaSpace{}%
\AgdaBound{ℱ'}\AgdaSpace{}%
\AgdaSymbol{→}\AgdaSpace{}%
\AgdaBound{ℱ'}\AgdaSpace{}%
\AgdaOperator{\AgdaRecord{≤''}}\AgdaSpace{}%
\AgdaInductiveConstructor{suc}\AgdaSpace{}%
\AgdaBound{ℱ}\AgdaSpace{}%
\AgdaOperator{\AgdaFunction{×}}\<%
\\
\>[.][@{}l@{}]\<[1741I]%
\>[9]\AgdaRecord{Σ}\AgdaSpace{}%
\AgdaOperator{\AgdaFunction{⌊ᵗ}}\AgdaSpace{}%
\AgdaBound{ℱ'}\AgdaSpace{}%
\AgdaOperator{\AgdaFunction{⸴}}\AgdaSpace{}%
\AgdaBound{Q₂}\AgdaSpace{}%
\AgdaOperator{\AgdaFunction{⸴}}\AgdaSpace{}%
\AgdaFunction{″}\AgdaSpace{}%
\AgdaBound{t₁}\AgdaSpace{}%
\AgdaOperator{\AgdaFunction{ᵗ⌋}}\AgdaSpace{}%
\AgdaSymbol{(λ}\AgdaSpace{}%
\AgdaBound{t₂}\AgdaSpace{}%
\AgdaSymbol{→}\AgdaSpace{}%
\AgdaFunction{″}\AgdaSpace{}%
\AgdaBound{t₂}\AgdaSpace{}%
\AgdaOperator{\AgdaDatatype{≡}}\AgdaSpace{}%
\AgdaFunction{″}\AgdaSpace{}%
\AgdaBound{t₁₂}\AgdaSpace{}%
\AgdaSymbol{)))}\<%
\\
%
\>[4]\AgdaFunction{go}\AgdaSpace{}%
\AgdaKeyword{with}\AgdaSpace{}%
\AgdaFunction{evalExp}\AgdaSpace{}%
\AgdaBound{exp}\AgdaSpace{}%
\AgdaBound{s}\<%
\\
%
\>[4]\AgdaSymbol{...}%
\>[1784I]\AgdaSymbol{|}\AgdaSpace{}%
\AgdaBound{f}\AgdaSymbol{@(}\AgdaInductiveConstructor{just}\AgdaSpace{}%
\AgdaSymbol{\AgdaUnderscore{})}\<%
\\
\>[.][@{}l@{}]\<[1784I]%
\>[8]\AgdaSymbol{=}\AgdaSpace{}%
\AgdaSymbol{(}\AgdaInductiveConstructor{Any.just}\AgdaSpace{}%
\AgdaInductiveConstructor{tt}\AgdaSymbol{)}\AgdaSpace{}%
\AgdaOperator{\AgdaInductiveConstructor{,}}\AgdaSpace{}%
\AgdaBound{ℱ}\AgdaSpace{}%
\AgdaOperator{\AgdaInductiveConstructor{,}}\AgdaSpace{}%
\AgdaInductiveConstructor{≤with}\AgdaSpace{}%
\AgdaSymbol{(}\AgdaFunction{+-comm}\AgdaSpace{}%
\AgdaBound{ℱ}\AgdaSpace{}%
\AgdaNumber{1}\AgdaSymbol{)}\AgdaSpace{}%
\AgdaOperator{\AgdaInductiveConstructor{,}}\AgdaSpace{}%
\AgdaBound{t₁₂}\AgdaSpace{}%
\AgdaOperator{\AgdaInductiveConstructor{,}}\AgdaSpace{}%
\AgdaInductiveConstructor{refl}\<%
\\
%
\>[2]\AgdaComment{--------------------------------------------------------------------}\<%
\end{code}}
\end{figure}



\subsection{Using the System to Reason about Programs}

Reasoning about swap was done for constants but a similar proof could be implemented with any expression as long as the variables to be swapped were also swapped across the whole expression. For example:


\begin{figure}
  \caption{SWAP: Using the library to formalise the correctness of the SWAP program:}
  \small
  {\centering \begin{code}
  \>[2]\AgdaFunction{SWAP}\AgdaSpace{}%
\AgdaSymbol{:}\AgdaSpace{}%
\AgdaSymbol{∀}\AgdaSpace{}%
\AgdaBound{𝑿}\AgdaSpace{}%
\AgdaBound{𝒀}%
\>[83I]\AgdaSymbol{→}\<%
\\
\>[83I][@{}l@{\AgdaIndent{0}}]%
\>[17]\AgdaOperator{\AgdaFunction{⟪}}\AgdaSpace{}%
\AgdaFunction{𝒙}\AgdaSpace{}%
\AgdaOperator{\AgdaFunction{==}}%
\>[86I]\AgdaSymbol{(}\AgdaInductiveConstructor{𝑐𝑜𝑛𝑠𝑡}\AgdaSpace{}%
\AgdaBound{𝑿}\AgdaSymbol{)}\AgdaSpace{}%
\AgdaOperator{\AgdaFunction{∧}}\AgdaSpace{}%
\AgdaFunction{𝒚}\AgdaSpace{}%
\AgdaOperator{\AgdaFunction{==}}\AgdaSpace{}%
\AgdaSymbol{(}\AgdaInductiveConstructor{𝑐𝑜𝑛𝑠𝑡}\AgdaSpace{}%
\AgdaBound{𝒀}\AgdaSymbol{)}\AgdaSpace{}%
\AgdaOperator{\AgdaFunction{⟫}}%
\>[66]\AgdaComment{-- Precondition}\<%
\\
\>[0]\<%
\\
\>[86I][@{}l@{\AgdaIndent{0}}]%
\>[28]\AgdaFunction{𝒛}\AgdaSpace{}%
\AgdaOperator{\AgdaInductiveConstructor{:=}}\AgdaSpace{}%
\AgdaInductiveConstructor{𝑣𝑎𝑙}\AgdaSpace{}%
\AgdaFunction{𝒙}\AgdaSpace{}%
\AgdaOperator{\AgdaInductiveConstructor{;}}\<%
\\
%
\>[28]\AgdaFunction{𝒙}\AgdaSpace{}%
\AgdaOperator{\AgdaInductiveConstructor{:=}}\AgdaSpace{}%
\AgdaInductiveConstructor{𝑣𝑎𝑙}\AgdaSpace{}%
\AgdaFunction{𝒚}\AgdaSpace{}%
\AgdaOperator{\AgdaInductiveConstructor{;}}\<%
\\
%
\>[28]\AgdaFunction{𝒚}\AgdaSpace{}%
\AgdaOperator{\AgdaInductiveConstructor{:=}}\AgdaSpace{}%
\AgdaInductiveConstructor{𝑣𝑎𝑙}\AgdaSpace{}%
\AgdaFunction{𝒛}\AgdaSpace{}%
\AgdaOperator{\AgdaInductiveConstructor{;}}\<%
\\
%
\\[\AgdaEmptyExtraSkip]%
%
\>[17]\AgdaOperator{\AgdaFunction{⟪}}\AgdaSpace{}%
\AgdaFunction{𝒙}\AgdaSpace{}%
\AgdaOperator{\AgdaFunction{==}}\AgdaSpace{}%
\AgdaSymbol{(}\AgdaInductiveConstructor{𝑐𝑜𝑛𝑠𝑡}\AgdaSpace{}%
\AgdaBound{𝒀}\AgdaSymbol{)}\AgdaSpace{}%
\AgdaOperator{\AgdaFunction{∧}}\AgdaSpace{}%
\AgdaFunction{𝒚}\AgdaSpace{}%
\AgdaOperator{\AgdaFunction{==}}\AgdaSpace{}%
\AgdaSymbol{(}\AgdaInductiveConstructor{𝑐𝑜𝑛𝑠𝑡}\AgdaSpace{}%
\AgdaBound{𝑿}\AgdaSymbol{)}\AgdaSpace{}%
\AgdaOperator{\AgdaFunction{⟫}}%
\>[65]\AgdaComment{-- Postcondition}\<%
\\
%
\>[2]\AgdaFunction{SWAP}\AgdaSpace{}%
\AgdaBound{𝑿}\AgdaSpace{}%
\AgdaBound{𝒀}\AgdaSpace{}%
\AgdaSymbol{=}\AgdaSpace{}%
\AgdaFunction{∎}\<%
\\
\>[2][@{}l@{\AgdaIndent{0}}]%
\>[5]\AgdaKeyword{where}\<%
\\
\>[0]\<%
\\
%
\>[5]\AgdaComment{-- Reasoning backwards from Postcondition Q to Precondition P}\<%
\\
\>[0]\<%
\\
%
\>[5]\AgdaFunction{PRE}\AgdaSpace{}%
\AgdaSymbol{:}\AgdaSpace{}%
\AgdaFunction{Assertion}\<%
\\
%
\>[5]\AgdaFunction{PRE}\AgdaSpace{}%
\AgdaSymbol{=}\AgdaSpace{}%
\AgdaFunction{𝒙}\AgdaSpace{}%
\AgdaOperator{\AgdaFunction{==}}\AgdaSpace{}%
\AgdaSymbol{(}\AgdaInductiveConstructor{𝑐𝑜𝑛𝑠𝑡}\AgdaSpace{}%
\AgdaBound{𝑿}\AgdaSymbol{)}\AgdaSpace{}%
\AgdaOperator{\AgdaFunction{∧}}\AgdaSpace{}%
\AgdaFunction{𝒚}\AgdaSpace{}%
\AgdaOperator{\AgdaFunction{==}}\AgdaSpace{}%
\AgdaSymbol{(}\AgdaInductiveConstructor{𝑐𝑜𝑛𝑠𝑡}\AgdaSpace{}%
\AgdaBound{𝒀}\AgdaSymbol{)}\<%
\\
%
\\[\AgdaEmptyExtraSkip]%
%
\>[5]\AgdaFunction{POST}\AgdaSpace{}%
\AgdaSymbol{:}\AgdaSpace{}%
\AgdaFunction{Assertion}\<%
\\
%
\>[5]\AgdaFunction{POST}\AgdaSpace{}%
\AgdaSymbol{=}\AgdaSpace{}%
\AgdaFunction{𝒙}\AgdaSpace{}%
\AgdaOperator{\AgdaFunction{==}}\AgdaSpace{}%
\AgdaSymbol{(}\AgdaInductiveConstructor{𝑐𝑜𝑛𝑠𝑡}\AgdaSpace{}%
\AgdaBound{𝒀}\AgdaSymbol{)}\AgdaSpace{}%
\AgdaOperator{\AgdaFunction{∧}}\AgdaSpace{}%
\AgdaFunction{𝒚}\AgdaSpace{}%
\AgdaOperator{\AgdaFunction{==}}\AgdaSpace{}%
\AgdaSymbol{(}\AgdaInductiveConstructor{𝑐𝑜𝑛𝑠𝑡}\AgdaSpace{}%
\AgdaBound{𝑿}\AgdaSymbol{)}\<%
\\
\>[0]\<%
\\
%
\>[5]\AgdaFunction{A₁}\AgdaSpace{}%
\AgdaSymbol{:}\AgdaSpace{}%
\AgdaFunction{Assertion}\<%
\\
%
\>[5]\AgdaFunction{A₁}\AgdaSpace{}%
\AgdaSymbol{=}\AgdaSpace{}%
\AgdaSymbol{((}\AgdaFunction{sub}\AgdaSpace{}%
\AgdaSymbol{(}\AgdaInductiveConstructor{𝑣𝑎𝑙}\AgdaSpace{}%
\AgdaFunction{𝒛}\AgdaSymbol{)}\AgdaSpace{}%
\AgdaFunction{𝒚}\AgdaSpace{}%
\AgdaSymbol{(}\AgdaInductiveConstructor{𝑣𝑎𝑙}\AgdaSpace{}%
\AgdaFunction{𝒙}\AgdaSymbol{))}\AgdaSpace{}%
\AgdaOperator{\AgdaFunction{==}}\AgdaSpace{}%
\AgdaSymbol{(}\AgdaInductiveConstructor{𝑐𝑜𝑛𝑠𝑡}\AgdaSpace{}%
\AgdaBound{𝒀}\AgdaSymbol{))}\AgdaSpace{}%
\AgdaOperator{\AgdaFunction{∧}}\AgdaSpace{}%
\AgdaSymbol{(}\AgdaSpace{}%
\AgdaFunction{𝒛}\AgdaSpace{}%
\AgdaOperator{\AgdaFunction{==}}\AgdaSpace{}%
\AgdaSymbol{(}\AgdaInductiveConstructor{𝑐𝑜𝑛𝑠𝑡}\AgdaSpace{}%
\AgdaBound{𝑿}\AgdaSymbol{))}\<%
\\
\>[0]\<%
\\
%
\>[5]\AgdaFunction{s₁}\AgdaSpace{}%
\AgdaSymbol{:}\AgdaSpace{}%
\AgdaOperator{\AgdaFunction{⟪}}\AgdaSpace{}%
\AgdaFunction{A₁}\AgdaSpace{}%
\AgdaOperator{\AgdaFunction{⟫}}\AgdaSpace{}%
\AgdaFunction{𝒚}\AgdaSpace{}%
\AgdaOperator{\AgdaInductiveConstructor{:=}}\AgdaSpace{}%
\AgdaInductiveConstructor{𝑣𝑎𝑙}\AgdaSpace{}%
\AgdaFunction{𝒛}\AgdaSpace{}%
\AgdaOperator{\AgdaInductiveConstructor{;}}\AgdaSpace{}%
\AgdaOperator{\AgdaFunction{⟪}}\AgdaSpace{}%
\AgdaFunction{POST}\AgdaSpace{}%
\AgdaOperator{\AgdaFunction{⟫}}\<%
\\
%
\>[5]\AgdaFunction{s₁}\AgdaSpace{}%
\AgdaSymbol{=}%
\>[175I]\AgdaKeyword{let}\AgdaSpace{}%
\AgdaBound{Ψ}\AgdaSpace{}%
\AgdaSymbol{=}\AgdaSpace{}%
\AgdaFunction{D0-Axiom-of-Assignment}\AgdaSpace{}%
\AgdaFunction{𝒚}\AgdaSpace{}%
\AgdaSymbol{(}\AgdaInductiveConstructor{𝑣𝑎𝑙}\AgdaSpace{}%
\AgdaFunction{𝒛}\AgdaSymbol{)}\AgdaSpace{}%
\AgdaFunction{POST}\AgdaSpace{}%
\AgdaKeyword{in}\AgdaSpace{}%
\AgdaFunction{go}\AgdaSpace{}%
\AgdaBound{Ψ}\<%
\\
\>[.][@{}l@{}]\<[175I]%
\>[10]\AgdaKeyword{where}\<%
\\
%
\>[10]\AgdaFunction{go}\AgdaSpace{}%
\AgdaSymbol{:}%
\>[187I]\AgdaOperator{\AgdaFunction{⟪}}\AgdaSpace{}%
\AgdaSymbol{((}\AgdaFunction{sub}\AgdaSpace{}%
\AgdaSymbol{(}\AgdaInductiveConstructor{𝑣𝑎𝑙}\AgdaSpace{}%
\AgdaFunction{𝒛}\AgdaSymbol{)}\AgdaSpace{}%
\AgdaFunction{𝒚}\AgdaSpace{}%
\AgdaSymbol{(}\AgdaInductiveConstructor{𝑣𝑎𝑙}\AgdaSpace{}%
\AgdaFunction{𝒙}\AgdaSymbol{))}\AgdaSpace{}%
\AgdaOperator{\AgdaFunction{==}}\AgdaSpace{}%
\AgdaSymbol{(}\AgdaInductiveConstructor{𝑐𝑜𝑛𝑠𝑡}\AgdaSpace{}%
\AgdaBound{𝒀}\AgdaSymbol{))}\<%
\\
\>[.][@{}l@{}]\<[187I]%
\>[15]\AgdaOperator{\AgdaFunction{∧}}\AgdaSpace{}%
\AgdaSymbol{((}\AgdaFunction{sub}\AgdaSpace{}%
\AgdaSymbol{(}\AgdaInductiveConstructor{𝑣𝑎𝑙}\AgdaSpace{}%
\AgdaFunction{𝒛}\AgdaSymbol{)}\AgdaSpace{}%
\AgdaFunction{𝒚}\AgdaSpace{}%
\AgdaSymbol{(}\AgdaInductiveConstructor{𝑣𝑎𝑙}\AgdaSpace{}%
\AgdaFunction{𝒚}\AgdaSymbol{))}\AgdaSpace{}%
\AgdaOperator{\AgdaFunction{==}}\AgdaSpace{}%
\AgdaSymbol{(}\AgdaInductiveConstructor{𝑐𝑜𝑛𝑠𝑡}\AgdaSpace{}%
\AgdaBound{𝑿}\AgdaSymbol{))}\AgdaSpace{}%
\AgdaOperator{\AgdaFunction{⟫}}\<%
\\
%
\>[15]\AgdaFunction{𝒚}\AgdaSpace{}%
\AgdaOperator{\AgdaInductiveConstructor{:=}}\AgdaSpace{}%
\AgdaInductiveConstructor{𝑣𝑎𝑙}\AgdaSpace{}%
\AgdaFunction{𝒛}\AgdaSpace{}%
\AgdaOperator{\AgdaInductiveConstructor{;}}\AgdaSpace{}%
\AgdaOperator{\AgdaFunction{⟪}}\AgdaSpace{}%
\AgdaFunction{POST}\AgdaSpace{}%
\AgdaOperator{\AgdaFunction{⟫}}\AgdaSpace{}%
\AgdaSymbol{→}\<%
\\
%
\>[15]\AgdaOperator{\AgdaFunction{⟪}}\AgdaSpace{}%
\AgdaFunction{A₁}\AgdaSpace{}%
\AgdaOperator{\AgdaFunction{⟫}}\AgdaSpace{}%
\AgdaFunction{𝒚}\AgdaSpace{}%
\AgdaOperator{\AgdaInductiveConstructor{:=}}\AgdaSpace{}%
\AgdaInductiveConstructor{𝑣𝑎𝑙}\AgdaSpace{}%
\AgdaFunction{𝒛}\AgdaSpace{}%
\AgdaOperator{\AgdaInductiveConstructor{;}}\AgdaSpace{}%
\AgdaOperator{\AgdaFunction{⟪}}\AgdaSpace{}%
\AgdaFunction{POST}\AgdaSpace{}%
\AgdaOperator{\AgdaFunction{⟫}}\<%
\\
%
\>[10]\AgdaFunction{go}\AgdaSpace{}%
\AgdaBound{t}\AgdaSpace{}%
\AgdaKeyword{with}\AgdaSpace{}%
\AgdaFunction{𝒚}\AgdaSpace{}%
\AgdaOperator{\AgdaFunction{?id=}}\AgdaSpace{}%
\AgdaFunction{𝒙}\<%
\\
%
\>[10]\AgdaFunction{go}\AgdaSpace{}%
\AgdaBound{t}\AgdaSpace{}%
\AgdaSymbol{|}\AgdaSpace{}%
\AgdaInductiveConstructor{yes}\AgdaSpace{}%
\AgdaBound{p}\AgdaSpace{}%
\AgdaKeyword{rewrite}\AgdaSpace{}%
\AgdaBound{p}\AgdaSpace{}%
\AgdaKeyword{with}\AgdaSpace{}%
\AgdaFunction{𝒙}\AgdaSpace{}%
\AgdaOperator{\AgdaFunction{?id=}}\AgdaSpace{}%
\AgdaFunction{𝒙}\<%
\\
%
\>[10]\AgdaFunction{go}\AgdaSpace{}%
\AgdaBound{t}\AgdaSpace{}%
\AgdaSymbol{|}\AgdaSpace{}%
\AgdaInductiveConstructor{yes}\AgdaSpace{}%
\AgdaBound{p}\AgdaSpace{}%
\AgdaSymbol{|}\AgdaSpace{}%
\AgdaInductiveConstructor{yes}\AgdaSpace{}%
\AgdaBound{q}\AgdaSpace{}%
\AgdaSymbol{=}\AgdaSpace{}%
\AgdaBound{t}\<%
\\
%
\>[10]\AgdaFunction{go}\AgdaSpace{}%
\AgdaBound{t}\AgdaSpace{}%
\AgdaSymbol{|}\AgdaSpace{}%
\AgdaInductiveConstructor{yes}\AgdaSpace{}%
\AgdaBound{p}\AgdaSpace{}%
\AgdaSymbol{|}\AgdaSpace{}%
\AgdaInductiveConstructor{no}\AgdaSpace{}%
\AgdaBound{¬q}\AgdaSpace{}%
\AgdaSymbol{=}\AgdaSpace{}%
\AgdaFunction{⊥-elim}\AgdaSpace{}%
\AgdaSymbol{(}\AgdaBound{¬q}\AgdaSpace{}%
\AgdaInductiveConstructor{refl}\AgdaSymbol{)}\<%
\\
%
\>[10]\AgdaFunction{go}\AgdaSpace{}%
\AgdaBound{t}\AgdaSpace{}%
\AgdaSymbol{|}\AgdaSpace{}%
\AgdaInductiveConstructor{no}\AgdaSpace{}%
\AgdaBound{¬p}\AgdaSpace{}%
\AgdaKeyword{with}\AgdaSpace{}%
\AgdaFunction{𝒚}\AgdaSpace{}%
\AgdaOperator{\AgdaFunction{?id=}}\AgdaSpace{}%
\AgdaFunction{𝒚}\<%
\\
%
\>[10]\AgdaFunction{go}\AgdaSpace{}%
\AgdaBound{t}\AgdaSpace{}%
\AgdaSymbol{|}\AgdaSpace{}%
\AgdaInductiveConstructor{no}\AgdaSpace{}%
\AgdaBound{¬p}\AgdaSpace{}%
\AgdaSymbol{|}\AgdaSpace{}%
\AgdaInductiveConstructor{yes}\AgdaSpace{}%
\AgdaBound{q}\AgdaSpace{}%
\AgdaSymbol{=}\AgdaSpace{}%
\AgdaBound{t}\<%
\\
%
\>[10]\AgdaFunction{go}\AgdaSpace{}%
\AgdaBound{t}\AgdaSpace{}%
\AgdaSymbol{|}\AgdaSpace{}%
\AgdaInductiveConstructor{no}\AgdaSpace{}%
\AgdaBound{¬p}\AgdaSpace{}%
\AgdaSymbol{|}\AgdaSpace{}%
\AgdaInductiveConstructor{no}\AgdaSpace{}%
\AgdaBound{¬q}\AgdaSpace{}%
\AgdaSymbol{=}\AgdaSpace{}%
\AgdaFunction{⊥-elim}\AgdaSpace{}%
\AgdaSymbol{(}\AgdaBound{¬q}\AgdaSpace{}%
\AgdaInductiveConstructor{refl}\AgdaSymbol{)}\<%
\end{code}}
  {\centering \hfill \Huge{\vdots} \hfill }
\end{figure}

\begin{figure}\ContinuedFloat
  \caption{SWAP: Using the library to formalise the correctness of the SWAP program:}
  \small
  \vspace{-0.5cm}
  \begin{center}\!\!\!\small{cont.}\end{center}
  {\centering \hfill \Huge{\vdots} \hfill }
  {\centering \begin{code}
\>[5]\AgdaFunction{A₂}\AgdaSpace{}%
\AgdaSymbol{:}\AgdaSpace{}%
\AgdaFunction{Assertion}\<%
\\
%
\>[5]\AgdaFunction{A₂}\AgdaSpace{}%
\AgdaSymbol{=}\AgdaSpace{}%
\AgdaSymbol{((}\AgdaFunction{sub}\AgdaSpace{}%
\AgdaSymbol{(}\AgdaInductiveConstructor{𝑣𝑎𝑙}\AgdaSpace{}%
\AgdaFunction{𝒚}\AgdaSymbol{)}\AgdaSpace{}%
\AgdaFunction{𝒙}\AgdaSpace{}%
\AgdaSymbol{(}\AgdaFunction{sub}\AgdaSpace{}%
\AgdaSymbol{(}\AgdaInductiveConstructor{𝑣𝑎𝑙}\AgdaSpace{}%
\AgdaFunction{𝒛}\AgdaSymbol{)}\AgdaSpace{}%
\AgdaFunction{𝒚}\AgdaSpace{}%
\AgdaSymbol{(}\AgdaInductiveConstructor{𝑣𝑎𝑙}\AgdaSpace{}%
\AgdaFunction{𝒙}\AgdaSymbol{)))}\AgdaSpace{}%
\AgdaOperator{\AgdaFunction{==}}\AgdaSpace{}%
\AgdaSymbol{(}\AgdaInductiveConstructor{𝑐𝑜𝑛𝑠𝑡}\AgdaSpace{}%
\AgdaBound{𝒀}\AgdaSymbol{))}\AgdaSpace{}%
\AgdaOperator{\AgdaFunction{∧}}\AgdaSpace{}%
\AgdaSymbol{(}\AgdaSpace{}%
\AgdaFunction{𝒛}\AgdaSpace{}%
\AgdaOperator{\AgdaFunction{==}}\AgdaSpace{}%
\AgdaSymbol{(}\AgdaInductiveConstructor{𝑐𝑜𝑛𝑠𝑡}\AgdaSpace{}%
\AgdaBound{𝑿}\AgdaSymbol{))}\<%
\\
%
\\[\AgdaEmptyExtraSkip]%
%
\>[5]\AgdaFunction{s₂}\AgdaSpace{}%
\AgdaSymbol{:}\AgdaSpace{}%
\AgdaOperator{\AgdaFunction{⟪}}\AgdaSpace{}%
\AgdaFunction{A₂}\AgdaSpace{}%
\AgdaOperator{\AgdaFunction{⟫}}\AgdaSpace{}%
\AgdaFunction{𝒙}\AgdaSpace{}%
\AgdaOperator{\AgdaInductiveConstructor{:=}}\AgdaSpace{}%
\AgdaInductiveConstructor{𝑣𝑎𝑙}\AgdaSpace{}%
\AgdaFunction{𝒚}\AgdaSpace{}%
\AgdaOperator{\AgdaInductiveConstructor{;}}\AgdaSpace{}%
\AgdaOperator{\AgdaFunction{⟪}}\AgdaSpace{}%
\AgdaFunction{A₁}\AgdaSpace{}%
\AgdaOperator{\AgdaFunction{⟫}}\<%
\\
%
\>[5]\AgdaFunction{s₂}\AgdaSpace{}%
\AgdaSymbol{=}%
\>[323I]\AgdaKeyword{let}\AgdaSpace{}%
\AgdaBound{Ψ}\AgdaSpace{}%
\AgdaSymbol{=}\AgdaSpace{}%
\AgdaFunction{D0-Axiom-of-Assignment}\AgdaSpace{}%
\AgdaFunction{𝒙}\AgdaSpace{}%
\AgdaSymbol{(}\AgdaInductiveConstructor{𝑣𝑎𝑙}\AgdaSpace{}%
\AgdaFunction{𝒚}\AgdaSymbol{)}\AgdaSpace{}%
\AgdaFunction{A₁}\AgdaSpace{}%
\AgdaKeyword{in}\AgdaSpace{}%
\AgdaFunction{go}\AgdaSpace{}%
\AgdaBound{Ψ}\<%
\\
\>[.][@{}l@{}]\<[323I]%
\>[10]\AgdaKeyword{where}\<%
\\
%
\>[10]\AgdaFunction{go}\AgdaSpace{}%
\AgdaSymbol{:}%
\>[335I]\AgdaOperator{\AgdaFunction{⟪}}\AgdaSpace{}%
\AgdaSymbol{((}\AgdaFunction{sub}\AgdaSpace{}%
\AgdaSymbol{(}\AgdaInductiveConstructor{𝑣𝑎𝑙}\AgdaSpace{}%
\AgdaFunction{𝒚}\AgdaSymbol{)}\AgdaSpace{}%
\AgdaFunction{𝒙}\AgdaSpace{}%
\AgdaSymbol{(}\AgdaFunction{sub}\AgdaSpace{}%
\AgdaSymbol{(}\AgdaInductiveConstructor{𝑣𝑎𝑙}\AgdaSpace{}%
\AgdaFunction{𝒛}\AgdaSymbol{)}\AgdaSpace{}%
\AgdaFunction{𝒚}\AgdaSpace{}%
\AgdaSymbol{(}\AgdaInductiveConstructor{𝑣𝑎𝑙}\AgdaSpace{}%
\AgdaFunction{𝒙}\AgdaSymbol{)))}\AgdaSpace{}%
\AgdaOperator{\AgdaFunction{==}}\AgdaSpace{}%
\AgdaSymbol{(}\AgdaInductiveConstructor{𝑐𝑜𝑛𝑠𝑡}\AgdaSpace{}%
\AgdaBound{𝒀}\AgdaSymbol{))}\<%
\\
\>[.][@{}l@{}]\<[335I]%
\>[15]\AgdaOperator{\AgdaFunction{∧}}\AgdaSpace{}%
\AgdaSymbol{((}\AgdaFunction{sub}\AgdaSpace{}%
\AgdaSymbol{(}\AgdaInductiveConstructor{𝑣𝑎𝑙}\AgdaSpace{}%
\AgdaFunction{𝒚}\AgdaSymbol{)}\AgdaSpace{}%
\AgdaFunction{𝒙}\AgdaSpace{}%
\AgdaSymbol{(}\AgdaInductiveConstructor{𝑣𝑎𝑙}\AgdaSpace{}%
\AgdaFunction{𝒛}\AgdaSymbol{))}\AgdaSpace{}%
\AgdaOperator{\AgdaFunction{==}}\AgdaSpace{}%
\AgdaSymbol{(}\AgdaInductiveConstructor{𝑐𝑜𝑛𝑠𝑡}\AgdaSpace{}%
\AgdaBound{𝑿}\AgdaSymbol{))}\AgdaSpace{}%
\AgdaOperator{\AgdaFunction{⟫}}\<%
\\
%
\>[15]\AgdaFunction{𝒙}\AgdaSpace{}%
\AgdaOperator{\AgdaInductiveConstructor{:=}}\AgdaSpace{}%
\AgdaInductiveConstructor{𝑣𝑎𝑙}\AgdaSpace{}%
\AgdaFunction{𝒚}\AgdaSpace{}%
\AgdaOperator{\AgdaInductiveConstructor{;}}\AgdaSpace{}%
\AgdaOperator{\AgdaFunction{⟪}}\AgdaSpace{}%
\AgdaFunction{A₁}\AgdaSpace{}%
\AgdaOperator{\AgdaFunction{⟫}}\AgdaSpace{}%
\AgdaSymbol{→}\<%
\\
%
\>[15]\AgdaOperator{\AgdaFunction{⟪}}\AgdaSpace{}%
\AgdaFunction{A₂}\AgdaSpace{}%
\AgdaOperator{\AgdaFunction{⟫}}\AgdaSpace{}%
\AgdaFunction{𝒙}\AgdaSpace{}%
\AgdaOperator{\AgdaInductiveConstructor{:=}}\AgdaSpace{}%
\AgdaInductiveConstructor{𝑣𝑎𝑙}\AgdaSpace{}%
\AgdaFunction{𝒚}\AgdaSpace{}%
\AgdaOperator{\AgdaInductiveConstructor{;}}\AgdaSpace{}%
\AgdaOperator{\AgdaFunction{⟪}}\AgdaSpace{}%
\AgdaFunction{A₁}\AgdaSpace{}%
\AgdaOperator{\AgdaFunction{⟫}}\<%
\\
%
\>[10]\AgdaFunction{go}\AgdaSpace{}%
\AgdaBound{t}\AgdaSpace{}%
\AgdaKeyword{with}\AgdaSpace{}%
\AgdaFunction{𝒙}\AgdaSpace{}%
\AgdaOperator{\AgdaFunction{?id=}}\AgdaSpace{}%
\AgdaFunction{𝒛}\<%
\\
%
\>[10]\AgdaFunction{go}\AgdaSpace{}%
\AgdaBound{t}\AgdaSpace{}%
\AgdaSymbol{|}\AgdaSpace{}%
\AgdaInductiveConstructor{yes}\AgdaSpace{}%
\AgdaBound{p}\AgdaSpace{}%
\AgdaSymbol{=}\AgdaSpace{}%
\AgdaFunction{⊥-elim}\AgdaSpace{}%
\AgdaSymbol{(}\AgdaFunction{𝒙≢𝒛}\AgdaSpace{}%
\AgdaBound{p}\AgdaSymbol{)}\<%
\\
%
\>[10]\AgdaFunction{go}\AgdaSpace{}%
\AgdaBound{t}\AgdaSpace{}%
\AgdaSymbol{|}\AgdaSpace{}%
\AgdaInductiveConstructor{no}%
\>[21]\AgdaSymbol{\AgdaUnderscore{}}\AgdaSpace{}%
\AgdaSymbol{=}\AgdaSpace{}%
\AgdaBound{t}\<%
\\
%
\\[\AgdaEmptyExtraSkip]%
%
\>[5]\AgdaFunction{A₃}\AgdaSpace{}%
\AgdaSymbol{:}\AgdaSpace{}%
\AgdaFunction{Assertion}\<%
\\
%
\>[5]\AgdaFunction{A₃}%
\>[397I]\AgdaSymbol{=}%
\>[12]\AgdaSymbol{((}\AgdaFunction{sub}\AgdaSpace{}%
\AgdaSymbol{(}\AgdaInductiveConstructor{𝑣𝑎𝑙}\AgdaSpace{}%
\AgdaFunction{𝒙}\AgdaSymbol{)}\AgdaSpace{}%
\AgdaFunction{𝒛}\AgdaSpace{}%
\AgdaSymbol{(}\AgdaFunction{sub}\AgdaSpace{}%
\AgdaSymbol{(}\AgdaInductiveConstructor{𝑣𝑎𝑙}\AgdaSpace{}%
\AgdaFunction{𝒚}\AgdaSymbol{)}\AgdaSpace{}%
\AgdaFunction{𝒙}\AgdaSpace{}%
\AgdaSymbol{(}\AgdaFunction{sub}\AgdaSpace{}%
\AgdaSymbol{(}\AgdaInductiveConstructor{𝑣𝑎𝑙}\AgdaSpace{}%
\AgdaFunction{𝒛}\AgdaSymbol{)}\AgdaSpace{}%
\AgdaFunction{𝒚}\AgdaSpace{}%
\AgdaSymbol{(}\AgdaInductiveConstructor{𝑣𝑎𝑙}\AgdaSpace{}%
\AgdaFunction{𝒙}\AgdaSymbol{))))}\AgdaSpace{}%
\AgdaOperator{\AgdaFunction{==}}\AgdaSpace{}%
\AgdaSymbol{(}\AgdaInductiveConstructor{𝑐𝑜𝑛𝑠𝑡}\AgdaSpace{}%
\AgdaBound{𝒀}\AgdaSymbol{))}\<%
\\
\>[397I][@{}l@{\AgdaIndent{0}}]%
\>[10]\AgdaOperator{\AgdaFunction{∧}}\AgdaSpace{}%
\AgdaSymbol{(}\AgdaSpace{}%
\AgdaFunction{𝒙}\AgdaSpace{}%
\AgdaOperator{\AgdaFunction{==}}\AgdaSpace{}%
\AgdaSymbol{(}\AgdaInductiveConstructor{𝑐𝑜𝑛𝑠𝑡}\AgdaSpace{}%
\AgdaBound{𝑿}\AgdaSymbol{)}\AgdaSpace{}%
\AgdaSymbol{)}\<%
\\
%
\\[\AgdaEmptyExtraSkip]%
%
\>[5]\AgdaFunction{s₃}\AgdaSpace{}%
\AgdaSymbol{:}\AgdaSpace{}%
\AgdaOperator{\AgdaFunction{⟪}}\AgdaSpace{}%
\AgdaFunction{A₃}\AgdaSpace{}%
\AgdaOperator{\AgdaFunction{⟫}}\AgdaSpace{}%
\AgdaFunction{𝒛}\AgdaSpace{}%
\AgdaOperator{\AgdaInductiveConstructor{:=}}\AgdaSpace{}%
\AgdaInductiveConstructor{𝑣𝑎𝑙}\AgdaSpace{}%
\AgdaFunction{𝒙}\AgdaSpace{}%
\AgdaOperator{\AgdaInductiveConstructor{;}}\AgdaSpace{}%
\AgdaOperator{\AgdaFunction{⟪}}\AgdaSpace{}%
\AgdaFunction{A₂}\AgdaSpace{}%
\AgdaOperator{\AgdaFunction{⟫}}\<%
\\
%
\>[5]\AgdaFunction{s₃}\AgdaSpace{}%
\AgdaSymbol{=}%
\>[433I]\AgdaKeyword{let}\AgdaSpace{}%
\AgdaBound{Ψ}\AgdaSpace{}%
\AgdaSymbol{=}\AgdaSpace{}%
\AgdaFunction{D0-Axiom-of-Assignment}\AgdaSpace{}%
\AgdaFunction{𝒛}\AgdaSpace{}%
\AgdaSymbol{(}\AgdaInductiveConstructor{𝑣𝑎𝑙}\AgdaSpace{}%
\AgdaFunction{𝒙}\AgdaSymbol{)}\AgdaSpace{}%
\AgdaFunction{A₂}\AgdaSpace{}%
\AgdaKeyword{in}\AgdaSpace{}%
\AgdaFunction{go}\AgdaSpace{}%
\AgdaBound{Ψ}\<%
\\
\>[.][@{}l@{}]\<[433I]%
\>[10]\AgdaKeyword{where}\<%
\\
%
\>[10]\AgdaFunction{go}\AgdaSpace{}%
\AgdaSymbol{:}%
\>[445I]\AgdaOperator{\AgdaFunction{⟪}}\AgdaSpace{}%
\AgdaSymbol{((}\AgdaFunction{sub}\AgdaSpace{}%
\AgdaSymbol{(}\AgdaInductiveConstructor{𝑣𝑎𝑙}\AgdaSpace{}%
\AgdaFunction{𝒙}\AgdaSymbol{)}\AgdaSpace{}%
\AgdaFunction{𝒛}\AgdaSpace{}%
\AgdaSymbol{(}\AgdaFunction{sub}\AgdaSpace{}%
\AgdaSymbol{(}\AgdaInductiveConstructor{𝑣𝑎𝑙}\AgdaSpace{}%
\AgdaFunction{𝒚}\AgdaSymbol{)}\AgdaSpace{}%
\AgdaFunction{𝒙}\AgdaSpace{}%
\AgdaSymbol{(}\AgdaFunction{sub}\AgdaSpace{}%
\AgdaSymbol{(}\AgdaInductiveConstructor{𝑣𝑎𝑙}\AgdaSpace{}%
\AgdaFunction{𝒛}\AgdaSymbol{)}\AgdaSpace{}%
\AgdaFunction{𝒚}\AgdaSpace{}%
\AgdaSymbol{(}\AgdaInductiveConstructor{𝑣𝑎𝑙}\AgdaSpace{}%
\AgdaFunction{𝒙}\AgdaSymbol{))))}\AgdaSpace{}%
\AgdaOperator{\AgdaFunction{==}}\AgdaSpace{}%
\AgdaSymbol{(}\AgdaInductiveConstructor{𝑐𝑜𝑛𝑠𝑡}\AgdaSpace{}%
\AgdaBound{𝒀}\AgdaSymbol{))}\<%
\\
\>[.][@{}l@{}]\<[445I]%
\>[15]\AgdaOperator{\AgdaFunction{∧}}\AgdaSpace{}%
\AgdaSymbol{((}\AgdaFunction{sub}\AgdaSpace{}%
\AgdaSymbol{(}\AgdaInductiveConstructor{𝑣𝑎𝑙}\AgdaSpace{}%
\AgdaFunction{𝒙}\AgdaSymbol{)}\AgdaSpace{}%
\AgdaFunction{𝒛}\AgdaSpace{}%
\AgdaSymbol{(}\AgdaInductiveConstructor{𝑣𝑎𝑙}\AgdaSpace{}%
\AgdaFunction{𝒛}\AgdaSymbol{))}\AgdaSpace{}%
\AgdaOperator{\AgdaFunction{==}}\AgdaSpace{}%
\AgdaSymbol{(}\AgdaInductiveConstructor{𝑐𝑜𝑛𝑠𝑡}\AgdaSpace{}%
\AgdaBound{𝑿}\AgdaSymbol{))}\AgdaSpace{}%
\AgdaOperator{\AgdaFunction{⟫}}\<%
\\
%
\>[15]\AgdaFunction{𝒛}\AgdaSpace{}%
\AgdaOperator{\AgdaInductiveConstructor{:=}}\AgdaSpace{}%
\AgdaInductiveConstructor{𝑣𝑎𝑙}\AgdaSpace{}%
\AgdaFunction{𝒙}\AgdaSpace{}%
\AgdaOperator{\AgdaInductiveConstructor{;}}\AgdaSpace{}%
\AgdaOperator{\AgdaFunction{⟪}}\AgdaSpace{}%
\AgdaFunction{A₂}\AgdaSpace{}%
\AgdaOperator{\AgdaFunction{⟫}}\AgdaSpace{}%
\AgdaSymbol{→}\<%
\\
%
\>[15]\AgdaOperator{\AgdaFunction{⟪}}\AgdaSpace{}%
\AgdaFunction{A₃}\AgdaSpace{}%
\AgdaOperator{\AgdaFunction{⟫}}\AgdaSpace{}%
\AgdaFunction{𝒛}\AgdaSpace{}%
\AgdaOperator{\AgdaInductiveConstructor{:=}}\AgdaSpace{}%
\AgdaInductiveConstructor{𝑣𝑎𝑙}\AgdaSpace{}%
\AgdaFunction{𝒙}\AgdaSpace{}%
\AgdaOperator{\AgdaInductiveConstructor{;}}\AgdaSpace{}%
\AgdaOperator{\AgdaFunction{⟪}}\AgdaSpace{}%
\AgdaFunction{A₂}\AgdaSpace{}%
\AgdaOperator{\AgdaFunction{⟫}}\<%
\\
%
\>[10]\AgdaFunction{go}\AgdaSpace{}%
\AgdaBound{t}\AgdaSpace{}%
\AgdaKeyword{with}\AgdaSpace{}%
\AgdaFunction{𝒛}\AgdaSpace{}%
\AgdaOperator{\AgdaFunction{?id=}}\AgdaSpace{}%
\AgdaFunction{𝒛}\<%
\\
%
\>[10]\AgdaFunction{go}\AgdaSpace{}%
\AgdaBound{t}\AgdaSpace{}%
\AgdaSymbol{|}\AgdaSpace{}%
\AgdaInductiveConstructor{yes}\AgdaSpace{}%
\AgdaSymbol{\AgdaUnderscore{}}\AgdaSpace{}%
\AgdaSymbol{=}\AgdaSpace{}%
\AgdaBound{t}\<%
\\
%
\>[10]\AgdaFunction{go}\AgdaSpace{}%
\AgdaBound{t}\AgdaSpace{}%
\AgdaSymbol{|}\AgdaSpace{}%
\AgdaInductiveConstructor{no}\AgdaSpace{}%
\AgdaBound{¬p}\AgdaSpace{}%
\AgdaSymbol{=}\AgdaSpace{}%
\AgdaFunction{⊥-elim}\AgdaSpace{}%
\AgdaSymbol{(}\AgdaBound{¬p}\AgdaSpace{}%
\AgdaInductiveConstructor{refl}\AgdaSymbol{)}\<%
\end{code}}
  {\centering \hfill \Huge{\vdots} \hfill }
\end{figure}

\begin{figure}\ContinuedFloat
  \caption{SWAP: Using the library to formalise the correctness of the SWAP program:}
  \small
  \vspace{-0.5cm}
  \begin{center}\!\!\!\small{cont.}\end{center}
  {\centering \hfill \Huge{\vdots} \hfill }
  {\centering \begin{code}
  \>[5]\AgdaFunction{s₄}\AgdaSpace{}%
\AgdaSymbol{:}\AgdaSpace{}%
\AgdaFunction{A₃}\AgdaSpace{}%
\AgdaOperator{\AgdaDatatype{≡}}\AgdaSpace{}%
\AgdaSymbol{(}\AgdaSpace{}%
\AgdaFunction{𝒚}\AgdaSpace{}%
\AgdaOperator{\AgdaFunction{==}}\AgdaSpace{}%
\AgdaSymbol{(}\AgdaInductiveConstructor{𝑐𝑜𝑛𝑠𝑡}\AgdaSpace{}%
\AgdaBound{𝒀}\AgdaSymbol{)}\AgdaSpace{}%
\AgdaOperator{\AgdaFunction{∧}}\AgdaSpace{}%
\AgdaFunction{𝒙}\AgdaSpace{}%
\AgdaOperator{\AgdaFunction{==}}\AgdaSpace{}%
\AgdaSymbol{(}\AgdaInductiveConstructor{𝑐𝑜𝑛𝑠𝑡}\AgdaSpace{}%
\AgdaBound{𝑿}\AgdaSymbol{)}\AgdaSpace{}%
\AgdaSymbol{)}\<%
\\
%
\>[5]\AgdaFunction{s₄}\AgdaSpace{}%
\AgdaKeyword{with}\AgdaSpace{}%
\AgdaFunction{𝒚}\AgdaSpace{}%
\AgdaOperator{\AgdaFunction{?id=}}\AgdaSpace{}%
\AgdaFunction{𝒙}\<%
\\
%
\>[5]\AgdaFunction{s₄}\AgdaSpace{}%
\AgdaSymbol{|}\AgdaSpace{}%
\AgdaInductiveConstructor{yes}\AgdaSpace{}%
\AgdaSymbol{\AgdaUnderscore{}}\AgdaSpace{}%
\AgdaKeyword{with}\AgdaSpace{}%
\AgdaFunction{𝒙}\AgdaSpace{}%
\AgdaOperator{\AgdaFunction{?id=}}\AgdaSpace{}%
\AgdaFunction{𝒛}\<%
\\
%
\>[5]\AgdaFunction{s₄}\AgdaSpace{}%
\AgdaSymbol{|}\AgdaSpace{}%
\AgdaInductiveConstructor{yes}\AgdaSpace{}%
\AgdaSymbol{\AgdaUnderscore{}}\AgdaSpace{}%
\AgdaSymbol{|}\AgdaSpace{}%
\AgdaInductiveConstructor{yes}\AgdaSpace{}%
\AgdaBound{q}\AgdaSpace{}%
\AgdaSymbol{=}\AgdaSpace{}%
\AgdaFunction{⊥-elim}\AgdaSpace{}%
\AgdaSymbol{(}\AgdaFunction{𝒙≢𝒛}\AgdaSpace{}%
\AgdaBound{q}\AgdaSymbol{)}\<%
\\
%
\>[5]\AgdaFunction{s₄}\AgdaSpace{}%
\AgdaSymbol{|}\AgdaSpace{}%
\AgdaInductiveConstructor{yes}\AgdaSpace{}%
\AgdaSymbol{\AgdaUnderscore{}}\AgdaSpace{}%
\AgdaSymbol{|}\AgdaSpace{}%
\AgdaInductiveConstructor{no}%
\>[22]\AgdaSymbol{\AgdaUnderscore{}}\AgdaSpace{}%
\AgdaKeyword{with}\AgdaSpace{}%
\AgdaFunction{𝒛}\AgdaSpace{}%
\AgdaOperator{\AgdaFunction{?id=}}\AgdaSpace{}%
\AgdaFunction{𝒛}\<%
\\
%
\>[5]\AgdaFunction{s₄}\AgdaSpace{}%
\AgdaSymbol{|}\AgdaSpace{}%
\AgdaInductiveConstructor{yes}\AgdaSpace{}%
\AgdaBound{p}\AgdaSpace{}%
\AgdaSymbol{|}\AgdaSpace{}%
\AgdaInductiveConstructor{no}%
\>[22]\AgdaSymbol{\AgdaUnderscore{}}\AgdaSpace{}%
\AgdaSymbol{|}\AgdaSpace{}%
\AgdaInductiveConstructor{yes}\AgdaSpace{}%
\AgdaSymbol{\AgdaUnderscore{}}\AgdaSpace{}%
\AgdaKeyword{rewrite}\AgdaSpace{}%
\AgdaBound{p}\AgdaSpace{}%
\AgdaSymbol{=}\AgdaSpace{}%
\AgdaInductiveConstructor{refl}\<%
\\
%
\>[5]\AgdaFunction{s₄}\AgdaSpace{}%
\AgdaSymbol{|}\AgdaSpace{}%
\AgdaInductiveConstructor{yes}\AgdaSpace{}%
\AgdaSymbol{\AgdaUnderscore{}}\AgdaSpace{}%
\AgdaSymbol{|}\AgdaSpace{}%
\AgdaInductiveConstructor{no}%
\>[22]\AgdaSymbol{\AgdaUnderscore{}}\AgdaSpace{}%
\AgdaSymbol{|}\AgdaSpace{}%
\AgdaInductiveConstructor{no}%
\>[30]\AgdaBound{w}\AgdaSpace{}%
\AgdaSymbol{=}\AgdaSpace{}%
\AgdaFunction{⊥-elim}\AgdaSpace{}%
\AgdaSymbol{(}\AgdaBound{w}\AgdaSpace{}%
\AgdaInductiveConstructor{refl}\AgdaSymbol{)}\<%
\\
%
\>[5]\AgdaFunction{s₄}\AgdaSpace{}%
\AgdaSymbol{|}\AgdaSpace{}%
\AgdaInductiveConstructor{no}\AgdaSpace{}%
\AgdaBound{¬p}\AgdaSpace{}%
\AgdaKeyword{with}\AgdaSpace{}%
\AgdaFunction{𝒙}\AgdaSpace{}%
\AgdaOperator{\AgdaFunction{?id=}}\AgdaSpace{}%
\AgdaFunction{𝒙}\<%
\\
%
\>[5]\AgdaFunction{s₄}\AgdaSpace{}%
\AgdaSymbol{|}\AgdaSpace{}%
\AgdaInductiveConstructor{no}%
\>[14]\AgdaSymbol{\AgdaUnderscore{}}\AgdaSpace{}%
\AgdaSymbol{|}\AgdaSpace{}%
\AgdaInductiveConstructor{no}\AgdaSpace{}%
\AgdaBound{¬q}\AgdaSpace{}%
\AgdaSymbol{=}\AgdaSpace{}%
\AgdaFunction{⊥-elim}\AgdaSpace{}%
\AgdaSymbol{(}\AgdaBound{¬q}\AgdaSpace{}%
\AgdaInductiveConstructor{refl}\AgdaSymbol{)}\<%
\\
%
\>[5]\AgdaFunction{s₄}\AgdaSpace{}%
\AgdaSymbol{|}\AgdaSpace{}%
\AgdaInductiveConstructor{no}%
\>[14]\AgdaSymbol{\AgdaUnderscore{}}\AgdaSpace{}%
\AgdaSymbol{|}\AgdaSpace{}%
\AgdaInductiveConstructor{yes}\AgdaSpace{}%
\AgdaSymbol{\AgdaUnderscore{}}\AgdaSpace{}%
\AgdaKeyword{with}\AgdaSpace{}%
\AgdaFunction{𝒛}\AgdaSpace{}%
\AgdaOperator{\AgdaFunction{?id=}}\AgdaSpace{}%
\AgdaFunction{𝒚}\<%
\\
%
\>[5]\AgdaFunction{s₄}\AgdaSpace{}%
\AgdaSymbol{|}\AgdaSpace{}%
\AgdaInductiveConstructor{no}%
\>[14]\AgdaSymbol{\AgdaUnderscore{}}\AgdaSpace{}%
\AgdaSymbol{|}\AgdaSpace{}%
\AgdaInductiveConstructor{yes}\AgdaSpace{}%
\AgdaSymbol{\AgdaUnderscore{}}\AgdaSpace{}%
\AgdaSymbol{|}\AgdaSpace{}%
\AgdaInductiveConstructor{yes}\AgdaSpace{}%
\AgdaBound{w}\AgdaSpace{}%
\AgdaSymbol{=}\AgdaSpace{}%
\AgdaFunction{⊥-elim}\AgdaSpace{}%
\AgdaSymbol{(}\AgdaFunction{𝒚≢𝒛}\AgdaSpace{}%
\AgdaSymbol{(}\AgdaFunction{sym}\AgdaSpace{}%
\AgdaBound{w}\AgdaSymbol{))}\<%
\\
%
\>[5]\AgdaFunction{s₄}\AgdaSpace{}%
\AgdaSymbol{|}\AgdaSpace{}%
\AgdaInductiveConstructor{no}%
\>[14]\AgdaSymbol{\AgdaUnderscore{}}\AgdaSpace{}%
\AgdaSymbol{|}\AgdaSpace{}%
\AgdaInductiveConstructor{yes}\AgdaSpace{}%
\AgdaSymbol{\AgdaUnderscore{}}\AgdaSpace{}%
\AgdaSymbol{|}\AgdaSpace{}%
\AgdaInductiveConstructor{no}%
\>[30]\AgdaSymbol{\AgdaUnderscore{}}\AgdaSpace{}%
\AgdaSymbol{=}\AgdaSpace{}%
\AgdaInductiveConstructor{refl}\<%
\\
  \>[5]\AgdaFunction{s₅}\AgdaSpace{}%
\AgdaSymbol{:}\AgdaSpace{}%
\AgdaOperator{\AgdaFunction{⟪}}\AgdaSpace{}%
\AgdaFunction{A₂}\AgdaSpace{}%
\AgdaOperator{\AgdaFunction{⟫}}\AgdaSpace{}%
\AgdaFunction{𝒙}\AgdaSpace{}%
\AgdaOperator{\AgdaInductiveConstructor{:=}}\AgdaSpace{}%
\AgdaInductiveConstructor{𝑣𝑎𝑙}\AgdaSpace{}%
\AgdaFunction{𝒚}\AgdaSpace{}%
\AgdaOperator{\AgdaInductiveConstructor{;}}%
\>[31]\AgdaFunction{𝒚}\AgdaSpace{}%
\AgdaOperator{\AgdaInductiveConstructor{:=}}\AgdaSpace{}%
\AgdaInductiveConstructor{𝑣𝑎𝑙}\AgdaSpace{}%
\AgdaFunction{𝒛}\AgdaSpace{}%
\AgdaOperator{\AgdaInductiveConstructor{;}}\AgdaSpace{}%
\AgdaOperator{\AgdaFunction{⟪}}\AgdaSpace{}%
\AgdaFunction{POST}\AgdaSpace{}%
\AgdaOperator{\AgdaFunction{⟫}}\<%
\\
%
\>[5]\AgdaFunction{s₅}\AgdaSpace{}%
\AgdaSymbol{=}\AgdaSpace{}%
\AgdaFunction{D2-Rule-of-Composition}\AgdaSpace{}%
\AgdaSymbol{\{}\AgdaFunction{A₂}\AgdaSymbol{\}}\AgdaSpace{}%
\AgdaSymbol{\{}\AgdaFunction{A₁}\AgdaSymbol{\}}\AgdaSpace{}%
\AgdaSymbol{\{}\AgdaFunction{POST}\AgdaSymbol{\}}\AgdaSpace{}%
\AgdaFunction{s₂}\AgdaSpace{}%
\AgdaFunction{s₁}\<%
\\
%
\\[\AgdaEmptyExtraSkip]%
%
\\[\AgdaEmptyExtraSkip]%
%
\>[5]\AgdaFunction{s₆}\AgdaSpace{}%
\AgdaSymbol{:}\AgdaSpace{}%
\AgdaOperator{\AgdaFunction{⟪}}\AgdaSpace{}%
\AgdaFunction{A₃}\AgdaSpace{}%
\AgdaOperator{\AgdaFunction{⟫}}\AgdaSpace{}%
\AgdaFunction{𝒛}\AgdaSpace{}%
\AgdaOperator{\AgdaInductiveConstructor{:=}}\AgdaSpace{}%
\AgdaInductiveConstructor{𝑣𝑎𝑙}\AgdaSpace{}%
\AgdaFunction{𝒙}\AgdaSpace{}%
\AgdaOperator{\AgdaInductiveConstructor{;}}%
\>[31]\AgdaFunction{𝒙}\AgdaSpace{}%
\AgdaOperator{\AgdaInductiveConstructor{:=}}\AgdaSpace{}%
\AgdaInductiveConstructor{𝑣𝑎𝑙}\AgdaSpace{}%
\AgdaFunction{𝒚}\AgdaSpace{}%
\AgdaOperator{\AgdaInductiveConstructor{;}}%
\>[45]\AgdaFunction{𝒚}\AgdaSpace{}%
\AgdaOperator{\AgdaInductiveConstructor{:=}}\AgdaSpace{}%
\AgdaInductiveConstructor{𝑣𝑎𝑙}\AgdaSpace{}%
\AgdaFunction{𝒛}\AgdaSpace{}%
\AgdaOperator{\AgdaInductiveConstructor{;}}\AgdaSpace{}%
\AgdaOperator{\AgdaFunction{⟪}}\AgdaSpace{}%
\AgdaFunction{POST}\AgdaSpace{}%
\AgdaOperator{\AgdaFunction{⟫}}\<%
\\
%
\>[5]\AgdaFunction{s₆}\AgdaSpace{}%
\AgdaSymbol{=}\AgdaSpace{}%
\AgdaFunction{D2-Rule-of-Composition}\AgdaSpace{}%
\AgdaSymbol{\{}\AgdaFunction{A₃}\AgdaSymbol{\}}\AgdaSpace{}%
\AgdaSymbol{\{}\AgdaFunction{A₂}\AgdaSymbol{\}}\AgdaSpace{}%
\AgdaSymbol{\{}\AgdaFunction{POST}\AgdaSymbol{\}}\AgdaSpace{}%
\AgdaFunction{s₃}\AgdaSpace{}%
\AgdaFunction{s₅}\<%
\\
%
\\[\AgdaEmptyExtraSkip]%
%
\>[5]\AgdaFunction{∎}\AgdaSpace{}%
\AgdaSymbol{:}\AgdaSpace{}%
\AgdaOperator{\AgdaFunction{⟪}}\AgdaSpace{}%
\AgdaFunction{PRE}\AgdaSpace{}%
\AgdaOperator{\AgdaFunction{⟫}}\AgdaSpace{}%
\AgdaFunction{𝒛}\AgdaSpace{}%
\AgdaOperator{\AgdaInductiveConstructor{:=}}\AgdaSpace{}%
\AgdaInductiveConstructor{𝑣𝑎𝑙}\AgdaSpace{}%
\AgdaFunction{𝒙}\AgdaSpace{}%
\AgdaOperator{\AgdaInductiveConstructor{;}}%
\>[31]\AgdaFunction{𝒙}\AgdaSpace{}%
\AgdaOperator{\AgdaInductiveConstructor{:=}}\AgdaSpace{}%
\AgdaInductiveConstructor{𝑣𝑎𝑙}\AgdaSpace{}%
\AgdaFunction{𝒚}\AgdaSpace{}%
\AgdaOperator{\AgdaInductiveConstructor{;}}%
\>[45]\AgdaFunction{𝒚}\AgdaSpace{}%
\AgdaOperator{\AgdaInductiveConstructor{:=}}\AgdaSpace{}%
\AgdaInductiveConstructor{𝑣𝑎𝑙}\AgdaSpace{}%
\AgdaFunction{𝒛}\AgdaSpace{}%
\AgdaOperator{\AgdaInductiveConstructor{;}}\AgdaSpace{}%
\AgdaOperator{\AgdaFunction{⟪}}\AgdaSpace{}%
\AgdaFunction{POST}\AgdaSpace{}%
\AgdaOperator{\AgdaFunction{⟫}}\<%
\\
%
\>[5]\AgdaFunction{∎}\AgdaSpace{}%
\AgdaSymbol{=}%
\>[695I]\AgdaFunction{D1-Rule-of-Consequence-pre}\AgdaSpace{}%
\AgdaSymbol{\{}\AgdaFunction{A₃}\AgdaSymbol{\}}\AgdaSpace{}%
\AgdaSymbol{\{}\AgdaFunction{swap}\AgdaSymbol{\}}\AgdaSpace{}%
\AgdaSymbol{\{}\AgdaFunction{POST}\AgdaSymbol{\}}\AgdaSpace{}%
\AgdaSymbol{\{}\AgdaFunction{PRE}\AgdaSymbol{\}}%
\>[62]\AgdaFunction{s₆}\AgdaSpace{}%
\AgdaFunction{go}\<%
\\
\>[695I][@{}l@{\AgdaIndent{0}}]%
\>[10]\AgdaKeyword{where}\<%
\\
%
\>[10]\AgdaFunction{go}\AgdaSpace{}%
\AgdaSymbol{:}\AgdaSpace{}%
\AgdaFunction{PRE}\AgdaSpace{}%
\AgdaOperator{\AgdaFunction{⇒}}\AgdaSpace{}%
\AgdaFunction{A₃}\<%
\\
%
\>[10]\AgdaFunction{go}\AgdaSpace{}%
\AgdaBound{s}\AgdaSpace{}%
\AgdaBound{x}%
\>[707I]\AgdaKeyword{rewrite}%
\>[708I]\AgdaFunction{ConjunctionComm}\<%
\\
\>[708I][@{}l@{\AgdaIndent{0}}]%
\>[27]\AgdaSymbol{(}\AgdaFunction{evalExp}\AgdaSpace{}%
\AgdaSymbol{(}\AgdaFunction{𝒙}\AgdaSpace{}%
\AgdaOperator{\AgdaFunction{==}}\AgdaSpace{}%
\AgdaInductiveConstructor{𝑐𝑜𝑛𝑠𝑡}\AgdaSpace{}%
\AgdaBound{𝑿}\AgdaSymbol{)}\AgdaSpace{}%
\AgdaBound{s}\AgdaSpace{}%
\AgdaSymbol{)}\<%
\\
%
\>[27]\AgdaSymbol{(}\AgdaFunction{evalExp}\AgdaSpace{}%
\AgdaSymbol{(}\AgdaFunction{𝒚}\AgdaSpace{}%
\AgdaOperator{\AgdaFunction{==}}\AgdaSpace{}%
\AgdaInductiveConstructor{𝑐𝑜𝑛𝑠𝑡}\AgdaSpace{}%
\AgdaBound{𝒀}\AgdaSymbol{)}\AgdaSpace{}%
\AgdaBound{s}\AgdaSpace{}%
\AgdaSymbol{)}\<%
\\
\>[.][@{}l@{}]\<[707I]%
\>[17]\AgdaSymbol{=}\AgdaSpace{}%
\AgdaFunction{subst}\AgdaSpace{}%
\AgdaSymbol{(λ}\AgdaSpace{}%
\AgdaBound{p}\AgdaSpace{}%
\AgdaSymbol{→}\AgdaSpace{}%
\AgdaBound{s}\AgdaSpace{}%
\AgdaOperator{\AgdaFunction{⊨}}\AgdaSpace{}%
\AgdaBound{p}\AgdaSpace{}%
\AgdaSymbol{)}\AgdaSpace{}%
\AgdaSymbol{(}\AgdaFunction{sym}\AgdaSpace{}%
\AgdaFunction{s₄}\AgdaSymbol{)}\AgdaSpace{}%
\AgdaBound{x}\<%
\end{code}}
\end{figure}


\begin{verbatim}
< x == y + 1 > $\leftarrow$ sub x for z
  
z := x (z == y + 1) $\leftarrow$ sub y for x

x := y (z == x + 1) $\leftarrow$ sub z for y

y := z <y == x + 1 >

\end{verbatim}


Some maths: \texttt{$\ll$\!\,\,P\,\,\!$\gg$\!\,\,C\,\,\!$\ll$\!\,\,Q\,\,\!$\gg$} test test hello hello
 
\section{Evaluation}


\subsection{Using Agda}

Getting better at working with Agda --- thanks to unicode suport, psychological bias of aesthetic but incorrect signature.


\subsection{Missteps}


Downsides: Having to define all logical manipulations in the interface. Some mechanism for making this less painful would be nice. Equally however, said logical manipulations are not really of the main concern. Just because nothing can be postulated for the proof to have a computational meaning, doesn't mean we need be bound by this restriction. Indeed, there is little reason for us to care about whether or not our proofs have this computational context so long as we trust the parts we omit, and as these ommisions are oft simple logical manipulations, leaving them only in the record but not actually proved would be perfectly sensible and allow more focus to be on the manipulation of Hoare triples to reason about programs.

Actually, if I was doing it again, it would have been very sensible to not bother with implementing the interfaces at all. It was probably not a worthwhile use of my time to prove De Morgans law, or the commutativity of boolean and, in Agda when I could have instead focused on the more salient parts of the code base.

\subsection{Future Work}

Gries page 164 'a fine balance between the two' 
\ldots but! automation, Infer,


parse a C program and create formal proof in background. Complain if fail

If the Agda code is to work as a library, there had ought to be some functionality for allowing potential users to add to Data-Implementation.agda, to define their own logical rules.


Ref paper: tactics for separation logic and how some reworking of data-interface could allow full use of HOL in Agda when manipulating assertions.


\subsection{Conclusion}

Hoare's surprise at test case success (see retrospective)

Not all that useful in practice, other tools are far more sophisticated and far better suited to practical applications, whether that be the verified software toolchain for reasoning about embedded software that needs to be correct, or Infer for catching a litany of bugs before they make their way into production. There's not a lot of room to claim that this Agda library has any real practical purpose. But that doesn't mean no purpose.

Constructive mathematics and interactive theorem provers are not front and center in mainstream mathematics, and perhaps never will be, but one oft talked about benefits of constructive mathematics is quite simply that it is fun to do [link MHE blog post] and it is on that note that one finds the most compelling use for this Agda library; it's actually a lot of fun --- if you're of a particular sort --- to reason about even the most simple of programs. I certainly found it enjoyable to reason about the swap program and have Agda check my workings for me.

Never been so intimately aquainted with the three lines of code comprising the swap function.


\section{Appendix}


\nocite{*}

\bibliographystyle{plain}

\bibliography{report}
 
\end{document}

